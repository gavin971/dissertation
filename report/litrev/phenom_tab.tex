%        File: phenom_tab.tex
%     Created: Thu Aug 04 11:00 AM 2011 C
% Last Change: Thu Aug 04 11:00 AM 2011 C
%
\begin{table}[h!]
  \centering
  \footnotesize{
  \begin{tabular}{|l|l|l|l|l|}
    \multicolumn{5}{c}{\textbf{Geologlical Repository Concepts}}\\
    \hline
    Geology & Hydrology & Geochemistry & Design Concepts & Thermal Behavior \\ 
    \hline
    granite&low porosity (~0.01)&reducing&single WP&bentonite limti 100C\\
    &dual porosity fractures&saline&carbon-steel/copper overpack&\\
    &high velocities&pH&bentonite buffer/backfill&\\
    &&can be saturated or unsaturated&closed&\\
    &&&saturated bentonite = diffusive&\\
    &&&&\\
    &&&&\\
    &&&&\\
    &&&<(-100m)&\\
    clay&very low conductivity&reducing&&limit due to EDZ enlargement with heat\\
    &high primary porosity (up to 0.5)&saline&&potentially 100C limit if bentonite buffer\\
    &very low effective porosity&pH&no/bentonite/concrete backfill&\\
    &slow water movement&saturated&horizontal, vertical, or room emplacement&\\
    &diffusion dominates&&closed&\\
    &&&&\\
    &&&&\\
    &&&&\\
    &&&&\\
    &&&<(-100m)&heat causes creep, enhanced sealing\\
    salt&&reducing (less reducing in the far field)& multiple waste packages per room&heat limit based on brine?\\
    &&pH&horizontal alcove emplacement&very limited information .\\
    &&FF saturated&salt backfill&\\
    &&NF dry, driven by heat &closed&\\
    &&&&\\
    &&&&\\
    &&&&\\
    &&&&\\
    &&&&\\
    &&&<(-100m)&\\
    borehole&&reducing&5km deep&\\
    &&&emplacement in bottom 2km&affects the hydrology\\
    &&limited solubility (reducing)&1km betonite/concrete seal &conductivity: 3.0 Watts/meter-Kelvin\\
    &crystalline rock&enhanced sorption (reducing)&bentonite grout filling&specific heat 790 J/kg-Kelvin\\
    &permeability ($10^-19$)&high salinity&compressed bentonite plugs&density $2750 $kg/m$^3$\\
    &permeability ($10^-19$)&pH&400 packages per borehole&\\
    &porosity (0.01)&saturated&closed&\\
    &&&&\\
    &&&&\\
    &&&&\\
    &&&<(-1000m)&\\
    \hline
  \end{tabular}
  \caption[Distinct Geological phenomena.]{Various geological repository 
  concepts demonstrate various dominant physical phenomena. }
  \label{tab:phenom_tab}
  }
\end{table}


