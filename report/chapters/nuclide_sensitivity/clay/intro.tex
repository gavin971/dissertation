
\section{Parametric Analyses With The Clay GDSM}

These analyses were performed using the Clay \gls{GDSM} developed by the 
\gls{UFD} campaign\cite{clayton_generic_2011}. The Clay \gls{GDSM} is built on the 
GoldSim software and tracks the movement of key radionuclides through the 
natural system and engineered barriers \cite{golder_associates_goldsim_2010, 
golder_associates+goldsim_2010-1}.

The disposal concept modeled by the Clay \gls{GDSM} includes an \gls{EBS} which 
can undergo rate based dissolution and barrier failure. Releases from the \gls{EBS} enter 
near field and subsequently far field host rock regions in which diffusive and 
advective transport take place, attenuated by solubility limits as well as 
sorption and dispersion phenomena.  

The Clay \gls{GDSM} models a single waste form, a waste package, additional 
\glspl{EBS}, 
an \gls{EDZ}, and a far field zone using a batch reactor mixing cell framework. This waste unit cell is modeled 
with boundary conditions such that it may be repeated assuming an infinite 
repository configuration. The waste form and engineered barrier system are modeled as well-mixed volumes 
and radial transport away from the cylindrical base case unit cell is modeled as  
one dimensional. Two radionuclide release pathways are considered. One is the nominal, 
undisturbed case, while the other is a fast pathway capable of simulating a 
hypothetical disturbed case 
\cite{clayton_generic_2011}.

