
A benchmarking effort was conducted to determine the accuracy of the 
semi-analytic generic geology thermal repository model developed at 
\gls{LLNL}\cite{hardin_generic_2011,sutton_investigations_2011,greenberg_application_2012} 
relative to the more traditional, numerical, lumped parameter technique in the 
\gls{SINDAG} model from \gls{ANL}.

The fast-running analytical thermal transport model assumes uniform thermal 
properties throughout a homogenous storage medium. Arrays of time-dependent heat 
sources are included geometrically as arrays of line segments and points.  The 
solver uses a source-based linear superposition of closed form analytical 
functions from each contributing point or line to arrive at an estimate of the 
thermal evolution of a generic geologic repository.  Temperature rise throughout 
the storage medium is computed as a linear superposition of temperature rises.  
It is modeled using the MathCAD mathematical engine and is parameterized to 
allow myriad gridded repository geometries and geologic characteristics 
\cite{ptc_mathcad_2010}.

It was anticipated that the accuracy and utility of the temperature field
calculated with the \gls{LLNL} analytical model would provide an accurate 
``birds-eye''
view in regions that are many tunnel radii away from actual storage units;
i.e., at distances where tunnels and individual storage units could
realistically be approximated as physical lines or points. 
However, geometrically explicit storage units, waste packages, tunnel
walls and close-in rock are not included in the MathCad model. 
The present benchmarking effort therefore focuses on the ability of the 
analytical model to accurately represent the close-in
temperature field.

Specifically, close-in temperatures computed with the \gls{LLNL} MathCAD model 
were benchmarked against temperatures computed using geometrically-explicit 
lumped-parameter, repository thermal modeling technique developed over several 
years at \gls{ANL} using the \gls{SINDAG} thermal modeling code 
\cite{gaski_sinda_1987}. Application of this numerical modeling technique to 
underground storage of heat generating nuclear waste streams within the proposed 
\gls{YMR} Site has been widely reported \cite{wigeland_separations_2006}.  New 
\gls{SINDAG} thermal models presented here share this same basic modeling 
approach. 

\section{Description of the Comparisons}

The two models were compared for a single tunnel case with UOX spent fuel and a 
0.35 meter tunnel radius. Shared assumptions of the model benchmarks include a 
single UOX assembly fuel loading per $5m$ of tunnel, calculation radii, numbers 
of adjacent tunnels, and geological thermal parameters. The benchmarking cases 
run in this validation effort for the simplified single 
tunnel case are listed in Table \ref{tab:benchSingle}. 


\section{Results}
The benchmarking effort between the analytical MathCAD model and the 
\gls{SINDAG} numerical  model showed that the analytical model was sufficiently 
in agreement with the numerical model for its purpose, rapid evaluation of 
generic geology repository configurations.  The analytic model gave peak 
temperatures for all cases run which agreed with the numerical numerical model 
within $4^{\circ}C$ and, for calculation radii less than 5 meters, consistently 
reported peak temperature timing within 11 years of the \gls{SINDAG} numerical 
model. In light of the magnitude of uncertainties involved in generically 
modeling a non-site-specific geologic repository, this sufficiently validated 
the analytical model with respect to its goals.

Peak times agreed well for close radii, though peak values were consistently 
underestimated by the analytical model. However, the time of peak heat arrived 
consistently sooner and the peak temperature value was consistently lower in the 
homogeneous medium analytical model than in the \gls{SINDAG} model. 

The results from the single and multiple drift scenarios are summarized in 
Tables \ref{tab:benchSingle} and \ref{tab:benchMulti}, respectively. 

\begin{table}
  \centering
  \begin{tabular}{|l|l|l|l|l|l|l|}
    \multicolumn{7}{c}{\textbf{Benchmarking Results for Single Drift 
    Scenario}}\\
    \hline
    Material & \multicolumn{3}{|c|}{Clay} & \multicolumn{3}{|c|}{Salt}\\ & 
    \multicolumn{3}{|c|}{$K_{th}=2.5$} & \multicolumn{3}{|c|}{$K_{th}=4.2$}\\ & 
    \multicolumn{3}{|c|}{$\alpha=1.13\times10^{-6}$} & 
    \multicolumn{3}{|c|}{$\alpha=2.07\times10^{-6}$}\\ 
    \hline
    & \multicolumn{6}{|c|}{Peak Temperature Discrepancy}\\ 
    & \multicolumn{6}{|c|}{$T_{peak,numeric}-T_{peak,analytic}$ $[^{\circ}C]$} \\
    \hline
    Years Cooling  & 10     & 25      & 50      & 10     & 25     & 50\\
    \hline
     R=0.35m  & 3.0   & 2.3     & 1.6    & 2.0   & 1.7   & 1.2\\
     R=0.69m  & 3.1   & 2.4    & 1.6    & 2.2    & 1.8   & 1.3\\
     R=3.46m  & 2.1   & 1.9    & 1.5    & 2.2   & 1.7    & 1.3\\
     R=7.04m  & 3.1   & 2.4     & 1.8    & 2.5   & 2.1   & 2.2\\
     R=14.32m & 3.6   & 2.9    & 2.1    & 2.8   & 2.6   & 3.7\\
    \hline
    & \multicolumn{6}{|c|}{Peak Heat Timing Discrepancy}\\ 
    & \multicolumn{6}{|c|}{ $t_{peak,numeric}-t_{peak,analytic}$ [yr]} \\
    \hline
    Material & \multicolumn{3}{|c|}{Clay} & \multicolumn{3}{|c|}{Salt}\\ & 
    \multicolumn{3}{|c|}{$K_{th}=2.5$} & \multicolumn{3}{|c|}{$K_{th}=4.2$}\\ & 
    \multicolumn{3}{|c|}{$\alpha=1.13\times10^{-6}$} & 
    \multicolumn{3}{|c|}{$\alpha=2.07\times10^{-6}$}\\ \hline
    Years Cooling  & 10     & 25      & 50      & 10     & 25     & 50\\
    \hline
     R=0.35m  & 1    & 1       & 1   & 1      & 1      & 3\\
     R=0.69m  & 2    & 2       & 1    & 2      & 3      & 4\\
     R=3.46m  & 9    & 7       & 6    & 4      & 2      & 11\\
     R=7.04m  & 4    & 13      & 10    & 11     & 10     & 288\\
     R=14.32m & 16   & 14      & 21   & 17     & 285    & 282\\
    \hline
  \end{tabular}
  \caption{Benchmarking in the single tunnel case showed that the peak heat was 
  calculated to be lower and arrived consistently sooner in the analytic model. 
  }
  \label{tab:benchSingle}
\end{table}

\begin{table}
  \centering
  \begin{tabular}{|l|l|l|l|}
    \multicolumn{4}{c}{\textbf{Benchmarking Results for 101 Drift Scenario}}\\
    \hline
    Material & \multicolumn{3}{|c|}{Clay} \\
    & \multicolumn{3}{|c|}{$K_{th}=2.5$}\\ 
    & \multicolumn{3}{|c|}{$\alpha=1.13\times10^{-6}$}  \\
    \hline
    & \multicolumn{3}{|c|}{Peak Temperature Discrepancy} \\
    & \multicolumn{3}{|c|}{$T_{peak,numeric}-T_{peak,analytic}$ $[^{\circ}C]$} \\
    \hline
    Years Cooling  & 10  & 25 & 50 \\
    \hline
    R=0.35m   & 7 & 4.6 & 2.1 \\
    \hline
    &\multicolumn{3}{|c|}{Peak Heat Timing Discrepancy}\\
    &\multicolumn{3}{|c|}{ $t_{peak,numeric}-t_{peak,analytic}$ [yr]} \\
    \hline
    R=0.35m       & -13.5   & 2   & -6  \\
    \hline
  \end{tabular}
  \caption{Benchmarking in the multiple tunnel case showed that the peak heat was 
  calculated to be consistently lower in the analytic model and deviated further
  from the numerical model than did the single tunnel case.
  }
  \label{tab:benchMulti}
\end{table}

\clearpage

