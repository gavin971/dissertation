
% DOE is thinking about various geologies

In addition to reconsideration of domestic fuel cycle policy, the uncertain 
future of the \gls{YMR} has driven the expansion of the option space of 
potential repository host geologies to include, at the very least, granite, 
clay/shale, salt, and deep borehole concepts \cite{nutt_used_2010}. 

% Various waste forms, packages, etc. are being considered.

In accordance with various fuel cycle options, corresponding waste form, waste 
package, and other engineered barrier systems are being considered.  
Specifically, current considerations include ceramic (e.g.  Uranium Oxide), 
glass (e.g.  borosilicate glasses), and metallic (e.g.  hydride fuels) waste 
forms. Waste packages may be copper, steel, or other alloys. Similarly, buffer 
and backfill materials vary from the crushed salt recommended for a salt 
repository to bentonite or concrete in other geologies. For this reason, \Cyder
was designed to be capable of modular substitution of engineered barrier 
components and data in order to analyze the full option space.

% Various geologies, WFs, WPs, EBSs have various physics

The physical, hydrologic, and geochemical mechanisms that dictate 
radionuclide and heat transport vary among the geological and engineered 
containment systems in the domestic disposal system option space.  Therefore, 
to support the system level simulation effort, a disposal system model must
capture the salient physics of these geological options and quantify associated 
disposal metrics.  Furthermore, in the same way that system level 
modularity facilitates analysis, so too does modular linkage between subcomponent 
process modules. The subcomponent models and repository environmental model in 
this work therefore provide a cohesively integrated disposal system simulator.


\subsubsection{Thermal Modeling Needs}
% repository loading limits
% optimization of layout
% necessary decay cooling time before emplacement.
The decay heat from nuclear material generates a significant heat source within 
a repository. This decay heat varies among fuel cycles, constrains repository 
loading capacity differently in respective geologic media, and affects the 
resilience of certain engineered barrier choices. In order, therefore, to 
distinguish among the performance of fuel cycle, geologic media, and \gls{EBS} 
choices, an accordingly capable thermal model has been included in the repository 
model. 

% The capability to model decay heat variations among fuel cycles is necessary
First, to distinguish among the repository decay heat burdens associated with 
various fuel cycles, the repository analysis model must, at the very least, 
capture the decay heat behavior of dominant heat contributors. In particular, 
partitioning and transmutation of heat generating radionuclides within  some 
fuel cycles will alter the isotopic composition of materials sent to the 
repository \cite{swift_applying_2010}, and the dominant heat contributing 
isotopes will therefore vary.  Isotopes of plutonium, americium, curium, and their 
decay daughters dominate long term decay heat contribution within directly 
deposited nuclear fuels.  Other high heat contributing radionuclides that may 
dominate shorter term decay heat include fission products such as isotopes of 
cesium and strontium \cite{piet_which_2007}. 

% The capability to model heat in each component is necessary
Second, the capability to model thermal evolution in each component is also 
necessary as the repository capacity may be constrained by thermal limits in 
\glspl{EBS} components. Such thermal limits have their technical basis in the 
temperature dependence of isolation integrity of the waste forms, waste 
packages, and buffer materials. Alteration, corrosion, degradation, and 
dissolution behaviors are often function of heat, in addition to redox 
conditions, and constrain loading density within the engineered barriers. 

% Finally, the capability to model whole repository thermal capacity is 
% necessary 
In addition, the capability to model far field thermal evolution is necessary 
in order to capture differences in thermal loading sensitivity among host rock 
choices. Thermal limits in the geologic environment can be based on the 
mechanical integrity of the rock as well as mineralogical, hydrologic and 
geochemical phenomena. The isolating characteristics of a geological 
environment are most sensitive to hydrologic and geochemical effects of thermal 
loading. Thus, heat load constraints are typically chosen to control hydrologic 
and geochemical response to thermal loading. In the United States, current 
regulations necessitate thermal limits in order to passively steward the 
repository's hydrologic and geochemical integrity against radionuclide  release 
for the first million years of the repository.  Such constraints affect the 
repository waste package spacing and repository footprint among other 
parameters. 

Finally, since some material and hydrologic phenomena affecting radionuclide 
transport are thermally coupled, future advanced implementation goals include 
dynamically informing those parameters with the thermal modeling capability 
developed here. Those temperature coupled phenomena include corrosion 
processes, dissolution rates, diffusion, solubility, and partition 
coefficients.  Only a coarse time resolution will be necessary to capture that 
coupling however, since time evolution of repository heat is such that thermal 
coupling can typically be treated as quasi static for long time scales 
\cite{andra_argile:_2005}. 

%andra, clay, evaluation, page 195)

%``In order to remain in an operational range in which phenomena are known and, 
%thus, reduce any damage to the argillite, the objective is to restrict 
%argillite temperature to these values. Basically, it means that the thermal 
%dimensioning of the cells and the architecture of the C waste repository zone 
%aim to restrict the temperature to 90°C at the interface “disposal cell – 
%argillite” and to ensure that the temperature will be below 70°C, in the 
%geological medium on the cell boundary, before a thousand years, which provides 
%a safety margin with respect to thermal effects.''


\subsubsection{Radionuclide Transport Modeling Needs}

Domestically, the \gls{EPA} has defined a limit on  human 
exposure due to the repository. This regulation places important limitations on 
capacity, design, and loading techniques for repository concepts under 
consideration. Repository concepts developed in this work must therefore 
quantify radionuclide transport through the geological environment in order to 
calculate repository capacity and other postprocessed performance metrics. 

In particular, the radionuclides in need of containment vary among fuel cycles 
and travel differently through various geologic environments and engineered 
barrer choices. Thus, in order to distinguish among the performance of fuel 
cycle, geologic media, and \gls{EBS} options, a hydrologic radionuclide 
contaminant transport modeling capability has been included in the repository 
model.  These capabilities focused on hydrogeologic, geochemical, and 
mechanical modeling behaviors identified as especially important to repository 
performance by sensitivity analyses conducted with a detailed geologic 
repository performance tool.  

Particular hydrogeologic modeling needs identified in Appendix sections 
\ref{sec:appendix_adv_vel, sec:appendix_diff_coeff} include the need to capture 
differing behavior in advectively dominated and diffusively dominated transport 
regimes. As advection and diffusion parameters vary among geologic media, and 
performance is sensitive to the dominant mode of transport, this distinction is 
necessary when modeling the far field in particular.

Particular geochemical modeling needs identified in Appendix sections 
\ref{sec:appendix_sol, sec:appendix_kd} include the need for solubility and 
sorption behaviors which were identified as varying greatly among host rock 
options and in some cases significantly affecting contaminant transport 
behaviors. 

Particular mechanical modeling needs identified in Appendix sections 
\ref{sec:appendix_deg, sec:appendix_fail} include the need for \gls{EBS} 
degradation based failure modeling. Waste package and waste form degradation 
and failure were shown to be important in geologic settings where transport is 
dominated by a fast advective pathway. 

Furthermore, to support varying modeling simplifications and assumptions common 
in performance assessment, modeling capabilities are necessary at varying 
levels of detail.  Accordingly, four interchangeable radionuclide transport 
models have been implemented. The simplest of these captures only the simplest 
mechanical modeling needs while the most detailed additionally captures 
geochemical and hydrogeologic behaviors.

%The exposure limit set by the \gls{EPA} is based on a `reasonably maximally 
%exposed individual.' For the \gls{YMR}, the limiting case is a person who lives, 
%grows food, drinks water and breathes air 18 km downstream from the repository. 
%The Yucca Mountain Repository \gls{EPA} regulations limit total dose from the 
%repository to 15 mrem/yr, and limit dose from drinking water to 4 mrem/yr for 
%the first 10,000 years. 
%Predictions of that dose rate depend on an enormous variety of factors, most 
%important of which is the primary pathway for release. In the \gls{YMR} primary 
%pathway of radionuclides from an accidental release will be from cracking aged 
%canisters. Subsequently, transport of the radionuclides to the water table 
%requires that the radionuclides come in contact with water and travel through 
%the rock to the water table. This results in contamination of drinking water 
%downstream.  
%
%Source term is a measure of the quantity of a radionuclide released into the 
%environment whereas radiotoxicity is a measure of the hazardous effect of that 
%particular radionuclide upon human ingestion or inhalation.  In particular, 
%radiotoxicity is measured in terms of the volume of water dilution required to 
%make it safe to ingest. Studies of source term and radiotoxicity therefore make 
%probabilistic assessments of radionuclide release, transport, and human 
%exposure.  
%
%Importantly, due to the long time scale and intrinsic uncertainties required in 
%such probabilistic assessments it is in general not advisable to base any 
%maximum repository capacity estimates on source term alone. This is due to the fact 
%that in order to give informative values for the risk associated with transport of 
%particular radionuclides, for example, it is necessary to make highly uncertain  
%predictions concerning waste form degradation, water flow, and other parameters 
%during the long repository evolution time scale.  However, source term remains a 
%pertinent metric for the comparison of alternative separations and fuel cycle
%scenarios as it is a fundamental factor in the calculation of risk.
%%The probabilistic nature of these assessments mean a direct dependence of 
%%source term on repository capacity can be difficult to arrive at.  
%
%Arriving at a generalized metric of probabilistic risk is fairly difficult. For 
%example, the \gls{PEI} metric from Berkeley (ref.  
%\cite{bouvier_comparison_2007}) is a multifaceted function of spent fuel 
%composition, waste conditioning, vitrification method, and radionuclide 
%transport through the repository walls and rock.  Also, it makes the assumption 
%that the waste canisters have been breached at $t=0$. Furthermore, reported in 
%$m^3$, PEI is a measure of radiotoxicity in the environment in the event of 
%total breach. While informative, this model on its own does not completely 
%determine a source-term limited maximum repository capacity.  Additional waste 
%package failure and a dose pathway model must be incorporated into it.

