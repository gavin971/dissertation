
% DOE is thinking about various geologies

In addition to reconsideration of the domestic fuel cycle policy, the uncertain 
future of the \gls{YMR} has driven the expansion of the option space of 
potential repository host geologies to include, at the very least, granite, 
clay/shale, salt, and deep borehole concepts \cite{nutt_used_2010}. 

% Various waste forms, packages, etc. are being considered.

In accordance with various fuel cycle options, corresponding waste form, waste 
package, and other engineered barrier systems are being considered.  
Specifically, current considerations include ceramic (e.g.  Uranium Oxide), 
glass (e.g.  borosilicate glasses), and metallic (e.g.  hydride fuels) waste 
forms. Waste packages may be copper, steel, or other alloys. Similarly, buffer 
and backfill materials vary from the crushed salt recommended for a salt 
repository to bentonite or concrete in other geologies. Therefore, this repository  
model was designed to be capable of modular substitution of waste form models and data
in order to analyze the full option space.

% Various geologies, WFs, WPs, EBSs have various physics

The physical, hydrologic, and geochemical mechanisms that dictate 
radionuclide and heat transport vary between the geological and engineered 
containment systems in the domestic disposal system option space.  Therefore, 
to support the system level simulation effort, models must
capture the salient physics of these geological options and quantify associated 
disposal metrics.  Furthermore, in the same way that system level 
modularity facilitates analysis, so too does modular linkage between subcomponent 
process modules. These subcomponent models and the repository environmental model 
must provide a cohesively integrated disposal system model such as is 
acheived in this work. 



\subsubsection{Thermal Modeling Needs}
% repository loading limits
% optimization of layout
% necessary decay cooling time before emplacement.
The decay heat from nuclear material generates a significant heat source within a 
repository. In order to arrive at loading strategies that comply with thermal 
limits in the engineered barrier system and the geological medium, a thermal 
modeling capability must be included in the repository model. Such a model is 
also necessary to inform material and hydrologic phenomena that affect 
radionuclide transport and are thermally coupled. 

Partitioning and transmutation of heat generating radionuclides within  
some fuel cycles will alter the heat evolution of the repository 
\cite{swift_applying_2010}. Thus, to distinguish  between the repository heat 
evolution associated with various fuel cycles involving partitioning and 
transmutation, a repository analysis model, must at the very least, 
capture the decay heat behavior of dominant heat contributors.  Plutonium, 
Americium, and their decay daughters dominate decay heat contribution within 
used nuclear fuels. Other contributing radionuclides include Cesium, Strontium, 
and Curium \cite{piet_which_2007}. 

Thermal limits within a used nuclear fuel disposal system are waste form, 
package, and geology dependent. The heat evolution of the repository 
constrains waste form loadings and package loadings as heat 
generated in the waste form is transported through the package. It 
also places requirements on the size, design, and loading strategy in a 
potential geological repository as that heat is deposited in the engineered 
barrier system and host geology.

Thermal limits of various waste forms have their technical basis in the 
temperature dependence of isolation integrity of the waste form. Waste form 
alteration, degradation, and dissolution behavior is a function of heat in 
addition to redox conditions and constrains loading
density within the waste form. 
 
Thermal limits of various engineered barrier systems similarly have a technical 
basis in the temperature dependent alteration, corrosion, degradation, and 
dissolution rates of the materials from whence they are constructed.  

Thermal limits of the geologic environment can be based on the mechanical 
integrity of the rock as well as mineralogical, hydrologic and geochemical 
phenomena. The isolating characteristics of a geological environment are most 
sensitive to hydrologic 
and geochemical effects of thermal loading. Thus, heat load constraints are 
typically chosen to control hydrologic and geochemical response to thermal 
loading. In the United States, current regulations necessitate thermal limits in 
order to passively steward the repository's hydrologic and geochemical integrity 
against radionuclide  release for the first 10,000 years of the repository.

Constraints for a broad set of possible geological environments 
depend on heat transport properties and geochemical behaviors of the rock matrix 
as well as its hydrologic state.  Such constraints affect the  
repository waste package spacing and repository footprint among 
other parameters. 

%Clay repositories should have a ~70 degrees C limit, because temps higher than 
%100 degrees can cause irreversible minerological damage.
%From ANDRA:
%``In order to remain in an operational range in which phenomena are known and, 
%thus, reduce any damage to the argillite, the objective is to restrict 
%argillite temperature to these values. Basically, it means that the thermal 
%dimensioning of the cells and the architecture of the C waste repository zone 
%aim to restrict the temperature to 90°C at the interface “disposal cell – 
%argillite” and to ensure that the temperature will be below 70°C, in the 
%geological medium on the cell boundary, before a thousand years, which provides 
%a safety margin with respect to thermal effects.''

In addition to development of a concept of heat transport within the repository 
in order to meet heat load limitations, it is also necessary to model 
temperature gradients in the repository in  order to support modeling of 
thermally dependent hydrologic and material phenomena.  As mentioned above, 
waste form corrosion processes, waste form
dissolution rates, diffusion coefficients, and the mechanical integrity of 
engineered barriers and geologic environment are coupled with temperature 
behavior. 
Only a coarse time resolution is necessary to capture that coupling 
however, since time evolution of repository heat is
such that thermal coupling can typically be treated as quasi static for long 
time scales.
\cite{andra_argile:_2005}. %andra, clay, evaluation, page 195)

\subsubsection{Source Term Modeling Needs}

Domestically, the \gls{EPA} has defined a limit on  human 
exposure due to the repository. This regulation places important limitations on 
capacity, design, and loading techniques for repository concepts under 
consideration. Repository concepts developed in this work must therefore 
quantify radionuclide transport through the geological environment in order to 
calculate repository capacity and other metrics. 

The exposure limit set by the \gls{EPA} is based on a `reasonably maximally 
exposed individual.' For the \gls{YMR}, the limiting case is a person who lives, 
grows food, drinks water and breathes air 18 km downstream from the repository. 
The Yucca Mountain Repository \gls{EPA} regulations limit total dose from the 
repository to 15 mrem/yr, and limit dose from drinking water to 4 mrem/yr for 
the first 10,000 years. 
Predictions of that dose rate depend on an enormous variety of factors, most 
important of which is the primary pathway for release. In the \gls{YMR} primary 
pathway of radionuclides from an accidental release will be from cracking aged 
canisters. Subsequently, transport of the radionuclides to the water table 
requires that the radionuclides come in contact with water and travel through 
the rock to the water table. This results in contamination of drinking water 
downstream.  

Source term is a measure of the quantity of a radionuclide released into the 
environment whereas radiotoxicity is a measure of the hazardous effect of that 
particular radionuclide upon human ingestion or inhalation.  In particular, 
radiotoxicity is measured in terms of the volume of water dilution required to 
make it safe to ingest. Studies of source term and radiotoxicity therefore make 
probabilistic assessments of radionuclide release, transport, and human 
exposure.  

Importantly, due to the long time scale and intrinsic uncertainties required in 
such probabilistic assessments it is in general not advisable to base any 
maximum repository capacity estimates on source term. This is due to the fact 
that in order to give informative values for the risk associated with transport of 
particular radionuclides, for example, it is necessary to make highly uncertain  
predictions concerning waste form degradation, water flow, and other parameters 
during the long repository evolution time scale.  However, source term remains a 
pertinent metric for the comparison of alternative separations and fuel cycle
scenarios as it is a fundamental factor in the calculation of risk.
%The probabilistic nature of these assessments mean a direct dependence of 
%source term on repository capacity can be difficult to arrive at.  

Arriving at a generalized metric of probabilistic risk is fairly difficult. For 
example, the \gls{PEI} metric from Berkeley (ref.  
\cite{bouvier_comparison_2007}) is a multifaceted function of spent fuel 
composition, waste conditioning, vitrification method, and radionuclide 
transport through the repository walls and rock.  Also, it makes the assumption 
that the waste canisters have been breached at $t=0$. Furthermore, reported in 
$m^3$, PEI is a measure of radiotoxicity in the environment in the event of 
total breach. While informative, this model on its own does not completely 
determine a source-term limited maximum repository capacity.  Additional waste 
package failure and a dose pathway model must be incorporated into it.

