% Scope

In this work, concise dominant physics thermal and radionuclide transport models 
were developed by comparison of analytical models with more detailed repository 
modeling tools. The result of this work is a software library capable of 
assessing a broad combinatoric domain of potential engineered barriers, 
repository design concepts, and geologic settings in the context of fuel cycle 
alternatives. Within this software library, a suite of basic capabilities have 
been demonstrated and validated and a few advanced features are operational. 

\subsection{Identification of the Modeling Domain}
% Selection of base set of subcomponents to model

An appropriate modeling domain was identified according to a review of current 
domestic and international candidate repository concepts, analytic models, and 
computational performance assessment techniques.  

Specifically, in order to capture the current domestic option space, three 
candidate geologic settings and four corresponding repository concepts under 
consideration by the \gls{UFD} campaign were selected for modeling in 
this work. The three geologic environments selected were clay, granite and salt.  
The four concepts included three enclosed, saturated, $500 m$ deep horizontal 
plane concepts in each of the three geologic settings and a fourth $5 km$ deep 
borehole concept conceived in crystalline basement rock.

% Review of available analytical models
% Review of available current detailed tools

Analytic models and detailed computational tools for repository analysis were 
investigated and categorized within the literature review.  Of these, candidate 
computational tools appropriate for performing abstraction and regression 
analyses were identified. Specifically, a suite of \gls{GDSM} tools developed by
\gls{DOE} \gls{UFD} campaign was selected to inform radionuclide transport model 
abstraction\cite{clayton_generic_2011} and two thermal analysis tools, also 
developed by the \gls{UFD} 
campaign, were selected to inform the thermal abstraction process 
\cite{huff_benchmarking_2012, huff_numerical_2012, greenberg_application_2012}.


\subsection{Identification of Dominant Physics}

% Sensitivity analyses sought to determine key parameters and expectations
Sensitivity analyses characterized the importance of physical mechanisms of 
repository performance  within the materials and media under consideration with 
respect to thermal and radionuclide transport.

Sensitivity analysis concerning thermal parameters was undertaken with available 
detailed models. These models, a 2D finite element thermal performance 
assessment model \cite{huff_benchmarking_2012, huff_numerical_2012} and a 
semi-analytic model \cite{greenberg_application_2012} were used to characterize 
the parametric dependence of thermal loading in a specific geologic environment.  
In the conductive thermal transport regime found in enclosed saturated 
repository concepts, the primary parameters distinguishing clay, salt, and 
granite geologic settings included thermal conductivity, thermal diffusivity, 
waste package spacing, and near field thermal limits. The results of these 
analyses provided the database powering \Cyder's thermal capability. A 
benchmarking effort between these models was conducted (see Appendix 
\ref{ch:thermal_accuracy}) and informed the level of detail with which 
repository capacity and thermal evolution are modeled in this work.  

Sensitivity analysis utilizing a \gls{UFD} \gls{GDSM} tool built on the 
GoldSim simulation framework informed performance sensitivity to engineered 
barrier system failure parameters as well as hydrologic and geochemical 
radionuclide transport phenomena. In saturated geologic settings, key parameters 
included dominant transport mechanism (diffusive or 
advective), isotopic spent fuel inventory and composition, engineered barrier 
failure, and redox state (insofar as it affected sorption behavior and 
solubility limitation).

Particular hydrologic modeling needs identified in Appendix section 
\ref{sec:AdvVelDiffCoeff} include the need to capture 
differing behavior in advectively dominated and diffusively dominated transport 
regimes. As advection and diffusion parameters vary among geologic media, and 
performance is sensitive to the dominant mode of transport, this distinction is 
necessary when modeling the far field in particular.

Particular geochemical modeling needs identified in Appendix sections 
\ref{sec:solubility} and \ref{sec:sorption} include the need for solubility and 
sorption behaviors which vary greatly among host rock options and were 
identified to, in some cases, significantly affect contaminant transport 
behaviors. 

Particular mechanical modeling needs identified in Appendix sections 
\ref{sec:wfdeginv} and \ref{sec:wpfail} include the need for \gls{EBS} 
degradation based failure modeling. Waste package and waste form degradation 
and failure were shown in those sensitivity analyses to be important in 
geologic settings where transport is dominated by a fast advective pathway. 

\subsection{Abstraction and Implementation}
% Analytic Models were implemented with respect to these sensitivity results

% Empirical data was incorporated

% For each component, abstraction of current tools and analytical models
Iterative abstraction between analytic models and sensitivity analysis results 
produced concise models that capture performance among candidate geologic media 
as a function of radionuclide inventory and heat generation over long time 
scales. Supporting data are derived primarily from the \gls{UFD} campaign 
\glspl{GDSM} and data, as well as European efforts such as the RED-IMPACT 
assessment and \gls{ANDRA} Dossier efforts \cite{von_lensa_red-impact_2008, 
andra_argile:_2005, clayton_generic_2011} . 

% Abstraction = Analytic Solutions + Regression Approximation

These concise models are a combination of two components: 
semi-analytic mathematical models that represent a simplified description of 
the most important physical phenomena, and semi-empirical models that reproduce 
the results of detailed models.  By combining the complexity of the analytic 
models and regression against numerical experiments, variations were limited 
between two models for the same system.  Different approaches have been 
compared in this work, with final modeling choices balancing the accuracy and 
efficiency of the possible implementations.  

In addition, the concise models are capable of roughly adjusting release 
pathways according to the characteristics of the natural system (both the host 
geologic setting and the site in general) and the engineered system (such as package 
loading arrangements, tunnel spacing, and engineered barriers).

The full abstraction process was iterated to achieve a balance between 
calculation speed and simulation detail. Model improvements during this stage 
sought a level of detail appropriate for informative comparison of subcomponents, but 
with sufficient speed to enable systems analysis. 
By varying input parameters and comparing with corresponding results from 
detailed tools, the behavior of each model on its parametric domain was 
explored.

In the \Cyder library, the resulting concise models support performance 
estimation within components of a robust disposal system simulation module. It 
was designed as a dynamically loadable module appropriate to represent a 
facility within the \Cyclus fuel cycle simulator. Furthermore, the robust 
architecture implemented within the repository module allows for 
interchangeable loading of components and model in support of simulations at 
various levels of fidelity.

% Abstraction in repository environment

% Abstraction for WFs, WPs, EBS, etc. will be the same

% Example, heat from WPs along repo drifts.

%For example, analytic models and semi-empirical models are available (i.e.  
%specific temperature integrals in ref. \cite{li_methodology_2006} and the 
%specific temperature change index in ref. \cite{radel_effect_2007}) that 
%approximate thermal response from heat generation in waste packages as linear
%along repository drifts. These analyses arrive at a thermal evolution over time 
%at any location in the rock by superpositioning and integration. Subsequently, 
%their results can be converted easily into conventional line loading and areal 
%power density metrics.  Such analytic models will first be assessed to determine
%likely parameters upon which thermal response will rely (e.g. tunnel spacing, 
%radionuclide inventories, etc.).
%
% Regression.


% Incorporation of empirical data

%For instance, in the case of radionuclide release from waste packages, analytic 
%models of radionuclide release (e.g.  congruent or solubility limited) will 
%first be assessed to determine likely parameters upon which radionuclide release 
%will rely (e.g.  radionuclide concentration, water flow rates, etc.) 
%\cite{kawasaki_congruent_2004}.  Again, regression analysis concerning those 
%parameters will be undertaken with available detailed models to further 
%characterize the parametric dependence of radionuclide release from specific 
%waste packages.  Finally, the radionuclide release model so developed will 
%depend on empirical data (e.g. the waste form dissolution rate).  Determination 
%of representative values to make available within the dominant physics model 
%will rely on existing empirical data concerning the specific waste form  
%materials being modeled (i.e. long time scale extrapolations of known glass 
%degradation rates).  

% Iterate
% Test using canonical data for those subcomponents 


