%        File: litrev/wf_tab.tex
%     Created: Fri Aug 05 09:00 AM 2011 C
% Last Change: Fri Aug 05 09:00 AM 2011 C
%
\begin{table}[h!]
  \centering
  \footnotesize{
    \begin{tabularx}{\textwidth}{|l|X|X|X|}
    \multicolumn{4}{c}{\textbf{Waste Form Types}}\\
    \hline
    WF Type & SubTypes & Contents & Release Drivers  \\
    \hline
    \hline
    Once Through & \gls{CSNF} Ceramic Oxide & Nominal Burnup UOx \& MOX & redox reactions \\
                 & \gls{CSNF} Ceramic Oxide & High Burnup  & redox reactions, heat  \\
                 & \gls{HTGR} TRISO Graphite & High Burnup & graphite reactions\\
                 & \gls{DSNF} Metal  & High Burnup N Reactor Fuel & metal reactions,  heat\\
                 & \gls{DSNF} Carbides  & Fast Reactor Fuels & carbide reactions,  heat\\
                 & \gls{DSNF} Ceramic Oxides  & Research Reactor Fuels & redox reactions,  heat\\
    \hline
    Borosilicate Glass & Current & \glspl{MA} Cs/Sr & heat, glass alteration \\
                       & Future & Mo, no \gls{MA} no Cs/Sr & glass alteration  \\
    \hline
    Glass Ceramic & Glass Bonded Sodalite & Echem processed oxide fuels & ceramic, redox, glass reactions  \\
    \hline
    Metal Alloy & From Echem & Cladding, noble metals & metal reactions, heat \\
                & From Aqueous & transition metals & metal reactions, heat  \\
    \hline
    Advance Ceramic &  & volatized iodine  & ceramic reactions, redox \\
    \hline
    Salt  & Cementitious Sodium  & separated streams  & alkaline reactions, dissolution \\
    \hline
  \end{tabularx}
  \caption[Waste form types.]{An array of waste forms developed for nuclear 
  wastes will have a corresponding array of dominant release mechanisms 
  \cite{blink_disposal_2010}.}
  \label{tab:wf}
  }
\end{table}


