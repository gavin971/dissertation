 \begin{table}[h!]
    \centering
    \footnotesize{
    \begin{tabularx}{\textwidth}{|X|c|c|X|}
      \multicolumn{4}{c}{\textbf{Models of Heat Load for Various Geologic Media}}\\
      \hline
      Source & Nation & Medium & Methodology \\  
      (Who) & (Where) & (What) & (How) \\  
      \hline
      Enresa \cite{von_lensa_red-impact_2008}           & Spain       & Granite       &  CODE\_BRIGHT 3D Finite Element \\ 
      NRI   \cite{von_lensa_red-impact_2008}            & Czech Rep.  & Granite       &  Specific Temperature Integral   \\
      ANDRA \cite{andra_granite:_2005}                  & France      & Granite       &  3D Finite Element CGM code   \\
      SKB \cite{ab_long-term_2006}                      & Sweden      & metagranite   &  1D-3D Site  Descriptive Models \\
      SCK$\cdot$CEN   \cite{von_lensa_red-impact_2008}  & Belgium     & Clay          &  Specific Temperature Integral   \\ 
      ANDRA \cite{andra_argile:_2005}                   & France      & Argile Clay   &  3D Finite Element CGM code   \\
      NAGRA \cite{johnson_project_2002, johnson_calculations_2002}  & Switzerland  & Opalinus Clay &  3D Finite Element CGM code \\
      GRS \cite{von_lensa_red-impact_2008}              & Germany     & Salt          &  HEATING (3D finite difference)   \\ 
      NCSU(Li)   \cite{li_examining_2007}               & USA         & Yucca Tuff    &  Specific Temperature Integral \\        
      NCSU(Nicholson) \cite{nicholson_thermal_2007}     & USA         & Yucca Tuff    &  COSMOL 3D Finite Element\\
      Radel \& Wilson \cite{radel_repository_2007}      & USA         & Yucca Tuff    &  Specific Temperature Change \\ 
      \hline
    \end{tabularx}
    \caption[International heat transport modeling methods in various geologic host media.]{Methods by which to calculate heat 
    load are independent of geologic medium. Maximum heat load constraints, however, vary among host formations. }
    \label{tab:heat}
    }
  \end{table}
