% Provide a summary of the work conducted:
%      Describe the technical problem clearly
%      support it with a method

An algorithm and supporting database for rapid thermal repository capacity 
calculation was implemented in \Cyder.  This algorithm employs a \gls{STC} 
method \cite{radel_effect_2007, radel_repository_2007} and has resulted from 
combining detailed spent nuclear fuel composition data \cite{carter_fuel_2011} 
with a detailed thermal repository performance analysis tool from \gls{LLNL} 
and the \gls{UFD} campaign \cite{greenberg_application_2012}. By abstraction of 
and benchmarking against these detailed thermal models, \Cyder captures the 
dominant physics of thermal phenomena affecting repository capacity in various 
geologic media and as a function of spent fuel composition.

Abstraction based on detailed computational thermal repository performance 
calculations with the \gls{LLNL} analytic model has resulted in implementation 
of the \gls{STC} estimation algorithm and a supporting reference dataset.  This 
method is capable of rapid estimation of temperature increase near emplacement 
tunnels as a function of waste composition, limiting radius, $r_{lim}$, waste 
package spacing, $S$, near field thermal conductivity, $K_{th}$, and near field 
thermal diffusivity, $\alpha_{th}$.
