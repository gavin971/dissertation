% Provide a summary of the work conducted:
%      Describe the technical problem clearly
%      support it with a method

An algorithm and supporting database for rapid thermal repository loading 
calculation was implemented in \Cyder.  This algorithm employs a \gls{STC} 
method \cite{radel_effect_2007, radel_repository_2007} and has resulted from 
combining a number of resources provided by the \gls{UFD} campaign.  These 
resources include detailed spent nuclear fuel composition data from Carter 
\cite{carter_fuel_2011} and a detailed thermal repository performance analysis 
tool from Greenberg at \gls{LLNL} and previously discussed in Section 
\ref{sec:llnl_background} \cite{greenberg_application_2012}. 
Benchmarking was conducted using an additional tool from Bauer 
at \gls{ANL}, previously discussed in Section \ref{sec:SINDA} \cite{huff_numerical_2012}. By abstraction of and benchmarking against 
these detailed thermal models, \Cyder captures the dominant physics of thermal 
phenomena affecting repository capacity in various geologic media and as a 
function of spent fuel composition.

Abstraction based on detailed computational thermal repository performance 
calculations with the \gls{LLNL} semi-analytic model has resulted in implementation 
of the \gls{STC} estimation algorithm and a supporting reference dataset.  This 
method is capable of rapid estimation of temperature increase near emplacement 
tunnels as a function of waste composition, limiting radius, $r_{lim}$, waste 
package spacing, $S$, near field thermal conductivity, $K_{th}$, and near field 
thermal diffusivity, $\alpha_{th}$.
