

\subsection{Specific Temperature Change Method}
Introduced by Radel, Wilson et al., the \gls{STC} method uses 
a linear approximation to arrive at the thermal loading density limit 
\cite{radel_repository_2007, radel_effect_2007}.  
Since the thermal response in a system with a long term transient response is strong function of the 
transient decay power, it is also a strong function of the isotopic 
composition of the waste. Thus, the time dependent temperature change, $\Delta 
T$, at the limiting radius, $r_{lim}$, can be approximated as proportional to the 
mass loading density. First, $\Delta T$ is determined for a limiting loading density 
of the particular material composition then it is normalized to a single 
kilogram of that material, $\Delta t$, the so called \gls{STC}. 

\begin{align}
 \Delta T(r_{lim}) &= m \cdot \Delta t(r_{lim})
 \label{STC}
 \intertext{where}
 \Delta T &= \mbox{ Temperature change due to m }[K]\nonumber\\
 m &= \mbox{ Mass of heat generating material }[kg]\nonumber \\
 \Delta t &= \mbox{ Temperature change due to 1 kg }[K]\nonumber\\
 r_{lim} &= \mbox{ Limiting radius } [m].\nonumber
\end{align}

For an arbitrary waste stream composition, scaled curves, $\Delta t_i$, calculated in this 
manner for individual isotopes can be superimposed for each isotope to arrive at an 
approximate total temperature change.

\begin{align}
 \Delta T (r_{lim}) &\sim \sum_{i} m_i \Delta t_i(r_{lim})
 \label{superposition}
\intertext{where}
 i &= \mbox{ An isotope in the material } [-]\nonumber\\
 m_i &= \mbox{ mass of isotope i  } [kg]\nonumber\\
 \Delta t_i &= \mbox{ Specifc temperature change due to \textsl{i} } [K].\nonumber
\end{align}


