\subsubsection{Explicit Advection Dominated Mass Transfer}\label{sec:adv_mass_transfer}

The first, specified-concentration or Dirichlet type boundary conditions define 
a specified species concentration on some section of the boundary of the 
representative volume, 

    \begin{align}
      C(\vec{r},t)\Big|_{\vec{r} \in \Gamma} &= C_0(t)
      \intertext{where}
      \vec{r} &= \mbox{ position vector }\nonumber\\
      \Gamma &= \mbox{ domain boundary }.\nonumber
    \end{align}

The right hand side of the Dirichlet boundary condition can be provided by any 
mass balance model, $j$, at its external boundary, $r_j$, based on the 
concentration profile it calculates (see Section \ref{sec:mass_balance}),

\begin{align}
C(z,t_n)|_{z=r_j} &= \mbox{ fixed concentration in j at }r_j\mbox{ and }t_n [kg/m^3].\nonumber\\ 
                  &= \begin{cases} 
                         \frac{m_{d}(t_n)}{V_{d}(t_n)}, & \mbox{Degradation Rate}\\
                         \frac{m_{df}(t_n)}{V_{df}(t_n)}, & \mbox{Mixed Cell}\\
                         C_{out}(t_n), & \mbox{Lumped Parameter}\\
                         C(r_j,t_n), & \mbox{One Dimensional PPM}.
                      \end{cases}
\end{align}

In the Degradation Rate and Mixed Cell models, the Dirichlet boundary condition can 
be chosen to enforce an advective flux on the inner boundary. This choice is 
appropriate when the user expects a primarily advective interface between two 
components. The advective flux across the boundary between two components $j$ 
and $k$, 

\begin{align}
J_{adv,jk}(t_n) &= \mbox{ potential advective flux from j to k at }t_n[kg/m^2/s]\nonumber\\
               &= \theta_k v C(z,t_n)|_{z=r_j}
\end{align}
relies on the fixed concentration Dirichlet boundary condition at the 
interface, provided by the internal component.

The resulting mass transfer into the Degradation Rate or Mixed Cell model
is, therefore, 
\begin{align}
m_{jk}(t_n) &= A\Delta t \theta_k v C(z,t_n)|_{z=r_j}
\intertext{where}
A &= \mbox{ surface area normal to the flow direction }[m^2]\nonumber\\
\Delta t &= \mbox{ length of the time step }[s].\nonumber
\end{align}


