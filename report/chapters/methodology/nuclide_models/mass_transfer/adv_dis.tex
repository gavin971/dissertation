\subsubsection{Coupled Advective Dispersive Mass Transfer}\label{sec:adv_dif_mass_transfer}

The third, head-dependent mixed boundary condition or Cauchy type, defines a 
solute flux along a boundary.  The fixed concentration flux Cauchy boundary 
condition can be provided at the external boundary of any mass balance model.  
For a vertically oriented system with advective velocity in the $\hat{k}$ 
direction,

    \begin{align}
      -D\frac{\partial C(z, t)}{\partial z}\Big|_{z \in \Gamma} + v_zC(z, t) &= v_zC(t) 
      \intertext{where}
      C(t) &= \mbox{ a known concentration function }[kg/m^{3}].\nonumber
    \end{align}  

On its inner boundary, the One Dimensional PPM model uses this fixed flux 
boundary condition directly in its solution such that, 
\begin{align}
C_{0,j}(t_n) &=  
\end{align}

The resulting mass transfer into the external component containing the Lumped 
Parameter model is 

\begin{align}
m_{ij}(t_n) &=\int C(z,t_n)dV - \int C(z, t_{n-1})dV.
\end{align}

In the Degradation Rate and Mixed Cell models, the Cauchy boundary condition 
can be selected to enforce coupled advective and dispersive flow.  The 
resulting mass transfer into the Degradation Rate or Mixed Cell model is then, 

\begin{align}
m_{ij}(t_n) &= A\Delta t \left( \theta_j v C(z,t_n)|_{z=r_i} - D \frac{\partial C(z,t_n)}{\partial z}|_{z=r_i} \right).
\end{align}

