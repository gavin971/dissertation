\subsection{Degradation Rate Radionuclide Transport Model}\label{sec:deg_rate}
The degradation rate model, simulating the fractional degradation of the 
material containment properties, is the simplest of implemented models and is most 
appropriate for simplistic waste package failure modeling. The fundamental 
concept is depicted in Figure \ref{fig:deg_volumes}.

\begin{figure}[h!]
  \begin{center}
    \def\svgwidth{.7\textwidth}
    \begin{figure}[h!]
  \begin{center}
    \def\svgwidth{.7\textwidth}
    \begin{figure}[h!]
  \begin{center}
    \def\svgwidth{.7\textwidth}
    \input{./chapters/methodology/nuclide_models/mass_balance/deg_rate/deg_volumes.eps_tex}
  \end{center}
  \caption[Constituents of a Degradation Rate Control Volume]{The control volume contains an 
  intact volume $V_i$ and a degraded volume, $V_d$. Contaminants in $V_d$ are 
  available for transport, while contaminants in $V_i$ are contained.}
  \label{fig:deg_volumes}
\end{figure}


  \end{center}
  \caption[Constituents of a Degradation Rate Control Volume]{The control volume contains an 
  intact volume $V_i$ and a degraded volume, $V_d$. Contaminants in $V_d$ are 
  available for transport, while contaminants in $V_i$ are contained.}
  \label{fig:deg_volumes}
\end{figure}


  \end{center}
  \caption[Constituents of a Degradation Rate Control Volume]{The control volume contains an 
  intact volume $V_i$ and a degraded volume, $V_d$. Contaminants in $V_d$ are 
  available for transport, while contaminants in $V_i$ are contained.}
  \label{fig:deg_volumes}
\end{figure}



The materials that constitute the engineered barriers in a saturated 
repository environment degrade over time. The implemented model of this nuclide 
release behavior is based solely on a fractional degradation rate. 
The degraded volume is a simple fraction, $d$, of the total volume, $V_T$,
\begin{align}
V_d &= dV_T
\intertext{where}
V_T &= V_i + V_d
V_T &= \mbox{ total volume }[m^3]\\
V_i &= \mbox{ intact volume }[m^3]\\
\intertext{and}
V_d &= \mbox{ degraded volume }[m^3].
\end{align}

In this model, the contaminants in the degraded fraction of the control volume 
are available to adjacent components. The available contaminants
$m_{ij}(t)$, at the boundary between cell $i$ to cell $j$ at time $t$ are thus

\begin{align}
\dot{m}_{ij}(t) &= f_i(\cdots)m_i(t)
\label{deg_rate_source_cont}
\intertext{where}
\dot{m}_{ij} &= \mbox{ the rate of mass transfer from i to j }[kg/s]\nonumber\\
f_i &= \mbox{ fractional degradation rate in cell i }[1/s] \nonumber\\
m_i &= \mbox{ mass in cell i }[kg] \nonumber\\
t &= \mbox{ time  }[s].\nonumber
\end{align}

For a situation as in \Cyder and \Cyclus, with discrete timesteps, the timesteps are 
assumed to be small enough to assume a constant rate $m_{ij}$ over the course 
of the timestep. This model incorporates the source term made available on the 
inner boundary into its available mass and defines the resulting boundary 
conditions at the outer boundary as solely a function of the degradation rate 
of that component.  The mass transferred between discrete times $t_{n-1}$ and 
$t_n$ is thus a simple linear function of the transfer rate in 
\eqref{deg_rate_source_cont}, 

\begin{align}
m_{ij}^n &= \int_{t_{n-1}}^{t_n}\dot{m}_{ij}(t')dt' \nonumber\\
         &= f_i(\cdots)m_i^{n-1}\left(t_n - t_{n-1}\right).
\label{deg_rate_source_discrete}
\end{align}

The concentration boundary condition must also be defined at the outer boundary 
to support parent components that utilize the Dirichlet boundary condition. For 
the degradation rate model, which incorporates no diffusion or advection, the 
concentration, $C_{ij}$ at the boundary between cells $i$ and $j$ is the average 
concentration in the saturated pore volume,

\begin{align}
C_{ij}^n &= \frac{m_i^n}{V_{vi}}
\label{deg_rate_dirichlet}\\
&= \frac{\mbox{ solute mass in cell i }}{\mbox{ void volume in cell i}}.\nonumber 
\end{align}

For the case in which all engineered barrier components are represented by 
degradation rate models, the source term at the outermost edge will be solely
a function of the original central source and the degradation rates of the 
components. 

To support parent components that utilize the Cauchy boundary condition, the 
degradation rate model assumes that the fluid velocity is constant across the cell 
as is the concentration. Thus, 

\begin{align}
-\theta_i D_{zz}\frac{\partial C}{\partial z}\hat{k} + \theta_iv_zC\hat{k} &= \theta_{i0} v_0 C_0\hat{k}\nonumber\\
\theta_i &= \mbox{ porosity in cell i }[-] \nonumber\\
D_{zz} &= \mbox{ $\hat{k}$ component diffusion tensor component in $\hat{k}$ direction }[m^2/s] \nonumber\\
C_i &= \mbox{ concentration in cell i }[kg/m^3] \nonumber\\
v_z &= \mbox{ velocity in $\hat{k}$ direction }[m/s] \nonumber
\end{align}
reduces to
\begin{align}
\theta_i^n v_z^n C_i^n &= \theta_i^{n-1} v_z^{n-1} C_i^{n-1}.
\label{deg_rate_cauchy}
\end{align}


%<++> &= \mbox{ <++> }[<++>] \nonumber\\

