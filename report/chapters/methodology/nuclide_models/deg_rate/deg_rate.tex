\subsection{Degradation Rate Radionuclide Transport Model}\label{sec:deg_rate}
The degradation rate model, simulating the fractional degradation of the 
material containment properties, is the simplest of implemented models and is most 
appropriate for simplistic waste package failure modeling. 

The materials that constitute the engineered barriers in a saturated 
repository environment degrade over time. The implemented model of this nuclide 
release behavior is based solely on a fractional degradation rate. This 
model incorporates the source term made available on the inner boundary into its 
available mass and defines the resulting boundary conditions at the outer boundary 
as solely a function of the degradation rate of that component.

This results in the following expression for the mass transfer, 
$m_{ij}(t)$, from cell $i$ to cell $j$ at time $t$ :

\begin{align}
\dot{m}_{ij}(t) &= f_i(\cdots)m_i(t)
\label{deg_rate_source_cont}
\intertext{where}
\dot{m}_{ij} &= \mbox{ the rate of mass transfer from i to j }[kg/s]\nonumber\\
f_i &= \mbox{ fractional degradation rate in cell i }[1/s] \nonumber\\
m_i &= \mbox{ mass in cell i }[kg] \nonumber\\
t &= \mbox{ time  }[s].\nonumber
\end{align}

For a situation as in Cyder and Cyclus, with discrete timesteps, the timesteps are 
assumed to be small enough to assume a constant rate $m_{ij}$ over the course of 
the timestep. The mass transferred between discrete times $t_{n-1}$ and $t_n$ is 
thus a simple linear function of the transfer rate in \eqref{deg_rate_source_cont}, 

\begin{align}
m_{ij}^n &= \int_{t_{n-1}}^{t_n}\dot{m}_{ij}(t')dt' \nonumber\\
         &= f_i(\cdots)m_i^{n-1}\left(t_n - t_{n-1}\right).
\label{deg_rate_source_discrete}
\end{align}

The concentration boundary condition must also be defined at the outer boundary 
to support parent components that utilize the Dirichlet boundary condition. For 
the degradation model, which incorporates no diffusion or advection, the 
concentration, $C_{ij}$ at the boundary between cells $i$ and $j$ is the average 
concentration in the saturated pore volume,

\begin{align}
C_{ij}^n &= \frac{m_i^n}{V_{vi}}
\label{deg_rate_dirichlet}\\
&= \frac{\mbox{ solute mass in cell i }}{\mbox{ void volume in cell i}}.\nonumber 
\end{align}

For the case in which all engineered barrier components are represented by 
degradation rate models, the source term at the outermost edge will be solely
a function of the original central source and the degradation rates of the 
components. 

To support parent components that utilize the Cauchy boundary condition, the 
degradation model assumes that the fluid velocity is constant across the cell 
as is the concentration. Thus, 

\begin{align}
-\theta_i D_{zz}\frac{\partial C}{\partial z}\hat{k} + \theta_iv_zC\hat{k} &= \theta_{i0} v_0 C_0\hat{k}\nonumber\\
\theta_i &= \mbox{ porosity in cell i }[-] \nonumber\\
D_{zz} &= \mbox{ $\hat{k}$ component diffusion tensor component in $\hat{k}$ direction }[m^2/s] \nonumber\\
C_i &= \mbox{ concentration in cell i }[kg/m^3] \nonumber\\
v_z &= \mbox{ velocity in $\hat{k}$ direction }[m/s] \nonumber
\end{align}
reduces to
\begin{align}
\theta_i^n v_z^n C_i^n &= \theta_i^{n-1} v_z^{n-1} C_i^{n-1}.
\label{deg_rate_cauchy}
\end{align}


%<++> &= \mbox{ <++> }[<++>] \nonumber\\

