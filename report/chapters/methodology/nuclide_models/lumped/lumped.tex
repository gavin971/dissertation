\subsection{Lumped Parameter Radionuclide Transport Model}\label{sec:lumped}

The response function model implemented interchangeable Piston Flow, 
Exponential, and Dispersion response functions \cite{maloszewski_lumped_1996}.
For systems in which the flow is sufficiently slow to be assumed constant over a 
timestep, it is possible to model a 
system of volumes as a connected lumped parameter models (Figure 
\ref{fig:lumpedseries}).


\begin{figure}[htbp!]
  \begin{center}
    \def\svgwidth{.8\textwidth}
    \input{./chapters/methodology/nuclide_models/lumped/lumpedseries.eps_tex}
  \end{center}
  \caption{A system of volumes can be modeled as lumped parameter models in 
  series.}
  \label{fig:lumpedseries}
\end{figure}

The method by which each lumped parameter component is modeled is according to a 
relationship between the incoming concentration, $C_{in}(t)$, and the outgoing 
concentration, $C_{out}(t)$, \begin{align}
  C_{out}(t) &= \int_{-\infty}^t C_{in}(t')g(t-t')e^{-\lambda(t-t')}dt'
  \label{lumped1}
  \intertext{equivalently}
  C_{out}(t) &= \int_0^\infty C_{in}(t-t')g(t')e^{-\lambda t'}dt'
  \label{lumped2}
  \intertext{where}
  t'  &= \mbox{ time of entry }[s]\nonumber\\
  t-t'  &= \mbox{ transit time }[s]\nonumber\\
  g(t-t')  &= \mbox{ response function, a.k.a. transit time 
  distribution}[-]\nonumber]\\
  \lambda &= \mbox{ radioactive decay constant }[s^{-1}].\nonumber
\end{align}

Selection of the response function is usually based on experimental tracer 
results in the medium at hand. If such data is available, functions used commonly in chemical 
engineering applications \cite{maloszewski_lumped_1996} include the Piston Flow Model (PFM), 
\begin{align}
  g(t') &= \delta{(t'-t_t))}
  \intertext{ the Exponential Model (EM) }
  g(t') &= \frac{1}{t_t}e^{-\frac{t'}{t_t}}
  \intertext{ and the Dispersion Model (DM), }
  g(t') &= \left( \frac{\emph{Pe }t_t}{4\pi t'} \right)^{\frac{1}{2}}
  \frac{1}{t'}e^{- \frac{\emph{Pe }t_t\left( 1- \frac{t'}{t_t}  \right)^2} 
  {4t'}}, \intertext{where}
  \emph{Pe}  &= \mbox{ Peclet number for mass diffusion }[-]\nonumber\\
  t_t  &= \mbox{ mean tracer age }[s]\nonumber\\
    &= t_w \mbox{ if there are no stagnant areas }\nonumber\\
  t_w  &= \mbox{ mean residence time of water }[s]\nonumber\\
       &= \frac{V_m}{Q}\nonumber\\
       &= \frac{z}{v_z}\nonumber\\
       &= \frac{z\theta_e}{q}\nonumber
  \intertext{in which}
  V_m  &= \mbox{ mobile water volume }[m^3]\nonumber\\
  Q    &= \mbox{ volumetric flow rate }[m^3/s]\nonumber\\
  z    &= \mbox{ average travel distance in flow direction }[m]\nonumber\\
  v_z  &= \mbox{ mean water velocity}[m/s]\nonumber\\
  q    &= \mbox{ Darcy Flux }[m/s]\nonumber\\
  \theta_e &= \mbox{ effective (connected) porosity }[\%].\nonumber
\end{align}
The latter of these, the Dispersion Model satisfies the one dimensional 
advection-dispersion equation, and is therefore the most physically relevant for 
this application. The solutions to these for constant concentration at the 
source boundary are given in Maloszewski and Zuber \cite{maloszewski_lumped_1996}, 
\begin{align}
  C(t) &=\begin{cases}
    PFM & C_0e^{-\lambda t_t}\\
    EM  & \frac{C_0}{1+\lambda t_t}\\
    DM & C_0e^{\frac{\emph{Pe}}{2}\left(1-\sqrt{1+\frac{4\lambda 
    t_t}{\emph{Pe}}}\right)}.
  \end{cases}
  \label{lumpedsolns}
\end{align}


\subsubsection{Calculation of Mass Transfer}

The LumpedNuclide model requires a specified internal concentration, so the 
Dirichlet boundary condition is queried at the internal boundary of the lumped 
parameter nuclide model. To calculate the resulting mass transfer over a 
timestep, the response function is applied and a linear concentration profile 
is made across the cell. The concentration profile combined with 
information about the initial state and the water volume in the cell, can be 
integrated over the volume to arrive at a mass change,

\begin{align}
m_j(t_n) &= \int_0^V \theta C(t_n, r) dV \nonumber\\
         &= \int_{r_i}^{r_j} \int_0^{2\pi} \int_0^h \theta C(t_n, r) r dr d\phi dh\nonumber \\
         &= 2\pi h\theta \int_{r_i}^{r_j} C(t_n, r)rdr \nonumber\\
         &= 2\pi h\theta \int_{r_i}^{r_j}\left( \frac{C_j(t_n) - C_i(t_n)}{r_j - r_i}r^2 + C_j(t_n)r \right) dr\nonumber
\intertext{such that}
m_j(t_n) &= 2\pi h \theta \left[ \frac{C_j(t_n) - C_i(t_n)}{3\left(r_j - r_i\right)}r^3 + C_j(t_n-1) \frac{r^2}{2}\right]_{r_i}^{r_j} \nonumber\\
         &= 2\pi h \theta \left[ \frac{C_j(t_n) - C_i(t_n)}{3\left(r_j - r_i\right)}(r_j-r_i)^3 + C_j(t_n) \frac{(r_j-r_i)^2}{2}\right]\nonumber\\ 
         &= 2\pi h \theta (r_j-r_i)^2 \left[ \frac{C_j(t_n) - C_i(t_n)}{3} + \frac{C_j(t_n)}{2} \right]\nonumber\\ 
         &= 2\pi h \theta (r_j-r_i)^2 \left[ \frac{5C_j(t_n)}{6} - \frac{C_i(t_n)}{3} \right]. 
\end{align}

Using this expression for $m_j$, the necessary mass transfer from $m_i$ is 
simply

\begin{align}
m_{ij} &= m_j(t_n) - m_j(t_{n-1}).
\end{align}

\subsubsection{Boundary Interfaces}
The external source term and boundary conditions are found exactly similarly to the method by 
which the Degradation Rate model finds external boundary conditions in Section 
\ref{sec:dr_bc}. This method is entirely based on contained mass and component 
volume.  
