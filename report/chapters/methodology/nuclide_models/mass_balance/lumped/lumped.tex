\subsubsection{Lumped Parameter Radionuclide Mass Balance Model}\label{sec:lumped}

For systems in which the flow is sufficiently slow to be assumed constant over 
a timestep, it is possible to model a system of volumes as a connected lumped 
parameter models (Figure \ref{fig:lumpedseries}). Based on this lumped paramter 
interperetation, the Lumped Parameter mass balance model implemented a response 
function model capable of Piston Flow, Exponential, and Dispersion response 
functions from Maloszewski and Zuber \cite{maloszewski_lumped_1996}.

\begin{figure}[htbp!]
  \begin{center}
    \def\svgwidth{.8\textwidth}
    \input{./chapters/methodology/nuclide_models/mass_balance/lumped/lumpedseries.eps_tex}
  \end{center}
  \caption{A system of volumes can be modeled as lumped parameter models in 
  series.}
  \label{fig:lumpedseries}
\end{figure}

Each lumped parameter component is modeled according to a 
relationship between the incoming concentration, $C_{in}(t)$, and the outgoing 
concentration, $C_{out}(t)$, \begin{align}
  C_{out}(t) &= \int_{-\infty}^t C_{in}(t')g(t-t')e^{-\lambda(t-t')}dt'
  \label{lumped1}
  \intertext{equivalently}
  C_{out}(t) &= \int_0^\infty C_{in}(t-t')g(t')e^{-\lambda t'}dt'
  \label{lumped2}
  \intertext{where}
  t'  &= \mbox{ time of entry }[s]\nonumber\\
  t-t'  &= \mbox{ transit time }[s]\nonumber\\
  g(t-t')  &= \mbox{ response function, a.k.a. transit time 
  distribution}[-]\nonumber]\\
  \lambda &= \mbox{ radioactive decay constant }[s^{-1}].\nonumber
\end{align}

Selection of the response function is usually based on experimental tracer 
results in the medium at hand. If such data is available, functions used 
commonly in chemical engineering applications \cite{maloszewski_lumped_1996} 
include the Piston Flow Model (PFM), 

\begin{align}
  g(t') &= \delta{(t'-t_t))}
  \intertext{ the Exponential Model (EM) }
  g(t') &= \frac{1}{t_t}e^{-\frac{t'}{t_t}}
  \intertext{ and the Dispersion Model (DM), }
  g(t') &= \left( \frac{\emph{Pe }t_t}{4\pi t'} \right)^{\frac{1}{2}}
  \frac{1}{t'}e^{- \frac{\emph{Pe }t_t\left( 1- \frac{t'}{t_t}  \right)^2} 
  {4t'}}, \intertext{where}
  \emph{Pe}  &= \mbox{ Peclet number for mass diffusion }[-]\nonumber\\
  t_t  &= \mbox{ mean tracer age }[s]\nonumber\\
    &= t_w \mbox{ if there are no stagnant areas }\nonumber\\
  t_w  &= \mbox{ mean residence time of water }[s]\nonumber\\
       &= \frac{V_m}{Q}\nonumber\\
       &= \frac{z}{v_z}\nonumber\\
       &= \frac{z\theta_e}{q}\nonumber
  \intertext{in which}
  V_m  &= \mbox{ mobile water volume }[m^3]\nonumber\\
  Q    &= \mbox{ volumetric flow rate }[m^3/s]\nonumber\\
  z    &= \mbox{ average travel distance in flow direction }[m]\nonumber\\
  v_z  &= \mbox{ mean water velocity}[m/s]\nonumber\\
  q    &= \mbox{ Darcy Flux }[m/s]\nonumber\\
  \theta_e &= \mbox{ effective (connected) porosity }[\%].\nonumber
\end{align}

The latter of these, the Dispersion Model satisfies the one dimensional 
advection-dispersion equation, and is therefore the most physically relevant for 
this application. The solutions to these for constant concentration at the 
source boundary are given in Maloszewski and Zuber \cite{maloszewski_lumped_1996}, 

\begin{align}
  C(t) &=\begin{cases}
    PFM & C_0e^{-\lambda t_t}\\
    EM  & \frac{C_0}{1+\lambda t_t}\\
    DM & C_0e^{\frac{\emph{Pe}}{2}\left(1-\sqrt{1+\frac{4\lambda 
    t_t}{\emph{Pe}}}\right)}.
  \end{cases}
  \label{lumpedsolns}
\end{align}

Since \Cyclus handles decay outside of \Cyder, the use of these models relies on a 
reference transit time and decay constant supplied by the user. The behavior of 
the reference isotope, in this way, fully defines the behavior of all isotopes.

For calculations later in this chapter concerning this model, it is important to 
note that a linear concentration profile is assumed between the inlet and the 
outlet,

\begin{align}
  C(z,t) = C_{in}(t)  + C_{out}(t)z.
\end{align}

