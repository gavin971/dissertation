\subsubsection{Degradation Rate Radionuclide Mass Balance Model}\label{sec:deg_rate}
Many barrier materials in a repository environment degrade over time.  The 
Degradation Rate mass balance model is the simplest implemented model and is 
most appropriate for simplistic modeling of a degrading barrier volume.  The 
Degradation Rate mass balance model does not attempt to model the physical 
mechanisms responsible for this degradation. Rather, it generically captures 
this behavior as a simple fractional degradation rate.  The fundamental concept 
is depicted in Figure \ref{fig:deg_volumes}.

\begin{figure}[h!]
  \begin{center}
    \def\svgwidth{.7\textwidth}
    \begin{figure}[h!]
  \begin{center}
    \def\svgwidth{.7\textwidth}
    \begin{figure}[h!]
  \begin{center}
    \def\svgwidth{.7\textwidth}
    \input{./chapters/methodology/nuclide_models/mass_balance/deg_rate/deg_volumes.eps_tex}
  \end{center}
  \caption[Constituents of a Degradation Rate Control Volume]{The control volume contains an 
  intact volume $V_i$ and a degraded volume, $V_d$. Contaminants in $V_d$ are 
  available for transport, while contaminants in $V_i$ are contained.}
  \label{fig:deg_volumes}
\end{figure}


  \end{center}
  \caption[Constituents of a Degradation Rate Control Volume]{The control volume contains an 
  intact volume $V_i$ and a degraded volume, $V_d$. Contaminants in $V_d$ are 
  available for transport, while contaminants in $V_i$ are contained.}
  \label{fig:deg_volumes}
\end{figure}


  \end{center}
  \caption[Constituents of a Degradation Rate Control Volume]{The control volume contains an 
  intact volume $V_i$ and a degraded volume, $V_d$. Contaminants in $V_d$ are 
  available for transport, while contaminants in $V_i$ are contained.}
  \label{fig:deg_volumes}
\end{figure}



For a situation as in \Cyder and \Cyclus, with discrete time steps, the time 
steps are assumed to be small enough to assume a constant rate of degradation over 
the course of the time step.  The degraded volume, then, is a simple fraction, 
$d$, of the total volume, $V_T$, such that 

\begin{align}
V_T &= V_i + V_d
\label{deg_volumes}
\intertext{where}
V_d(t) &= d(t)V_T\nonumber\\
V_i(t) &= (1-d(t))V_T\nonumber\\
V_T &= \mbox{ total volume }[m^3]\nonumber\\
V_i(t) &= \mbox{ intact volume at time t }[m^3]\nonumber\\
V_d(t) &= \mbox{ degraded volume at time t }[m^3]\nonumber
\intertext{and}
d(t) &= \mbox{ the fraction that has been degraded by time t }[-]\nonumber\\
     &= \sum_{n=0}^{N}f_n\Delta t\nonumber\\
\intertext{where}
f_n &= \mbox{ the constant rate over a time step }[1/s]\nonumber\\ 
\Delta t &= \mbox{ the length of a timestep }[s].\nonumber
\end{align}

In this model, all contaminants in the degraded fraction of the control volume 
are available to adjacent components such that,

\begin{align}
m_{jk}(t_n) &= m_{j,d}(t_n)
\intertext{where}
m_{jk} &= \mbox{ maximum contaminant mass available for transfer from j to k }[kg]\nonumber\\
m_{j,d} &= \mbox{ mass in degraded volume of cell j }[kg] \nonumber\\
t_n &= \mbox{ time }[s].\nonumber
\end{align}

The total contaminants $m_{d,j}(t_n)$, available in the degraded volume
at time $t_n$ are calculated based on mass flux from the
inner boundary, the updated mass in the degraded volume at the previous 
timestep, and the mass released by degradation during the 
current timestep. Specifically, 

\begin{align}
m_{d,j}(t_n) &= m_{ij}(t_n) + m_{d,j}^*(t_n) + m_{j,i}^*(t_{n-1})f_n\Delta t
\label{deg_rate_source_cont}
\intertext{where}
m_{ij}(t_n) &= \mbox{ incoming mass from the inner boundary }[kg]\nonumber\\
m_{j,d}^*(t_{n-1}) &= \mbox{ mass in the degraded volume at the end of }t_{n-1}\mbox{ }[kg]\nonumber\\
m_{j,i}^*(t_{n-1})f_n\Delta t &= \mbox{ mass in the intact volume at the end of }t_{n-1}\mbox{ }[kg]\nonumber\\
f_n&= \mbox{ degradation rate during the timestep }t_n\mbox{ }[kg]\nonumber\\
\Delta t &= t_n-t_{n-1}\mbox{ }[s].\nonumber
\end{align}

The concentration calculation results from the mass balance calculation in 
\eqref{deg_rate_source_cont} 
to support parent components that utilize the Dirichlet boundary condition. 
For 
the degradation rate model, which incorporates no diffusion or advection, the 
concentration, $C_j$ at $r_j$, the boundary between cells $j$ and $k$, is the average 
concentration in degraded volume, 

\begin{align}
C_{d} &= \frac{m_{d}(t_n)}{V_{d}(t_n)}\\
\label{deg_rate_conc}\\
&= \frac{\mbox{ solute mass in degraded fluid in cell j }}{\mbox{ degraded fluid volume in cell j}}.\nonumber 
\end{align}

