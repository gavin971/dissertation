\subsection{Timestep Algorithm}\label{sec:timestepping}

In \Cyder, radionuclide contaminant flow is assumed to travel outward from 
the central Component (and up, in the $\hat{k}$ direction). In order to conduct 
a mass balance in each Component at each time step, the mass flow and mass 
balance calculations proceed from the innermost Component to the outermost 
Component. As mass flows from inner components to outer components, the mass 
balances in inner components are updated.  Thus, nuclide release information is 
passed radially outward from the waste stream sequentially through each 
containment layer to the geosphere.  This implicit timestepping method 
arrives at the updated state of each Component, radially outward, as a function 
of both the past state and the current state of the system. 

To illustrate the algorithm by which mass flow calculations are conducted 
through the system of components at each timestep, we will walk through the 
phases of a single timestep for a simple pair of components. The source, $i$, 
is the innermost and the sink, $j$, is the outermost component. 

\subsubsection{Phase 1: Initial Conditions}

The initial conditions in both the source and the sink at the beginning of a 
timestep are equal to the final updated state of the previous timestep. If this 
is the first timestep, the global intial state of the repository system is used. 

\subsubsection{Phase 2: Interior Mass Balance}

The mass distrubution and concentration profile in the interior source volume 
$i$ is solved based on the initial condition, any influxes, and the physics of 
its mass balance model.  This calculation results in a contaminant mass 
distribution and concentration profile within the volume $i$ at time $t_n$.  
For each of the models, this mass balance is calculated thus :

This mass distribution and concentration profile fully inform 
the conditions on the boundary at $r_i$ and this information is made available 
to the external component, $j$.


\subsubsection{Phase 3: Mass Transfer Calculation}

The mass transfer from the source volume $i$ to the sink volume $j$ is 
calculated next, based on the up to date boundary conditions at $r_i$ 
determined in Phase 2 and the initial conditions in volume $j$. The mass 
transfer is calculated according to the mass transfer mode preference of the 
mass balance model of volume $j$.  

The Degradation Rate and Mixed Cell models can be parameterized to utilize an 
explicit mass transfer mode that captures either advection, dispersion, or coupled flow. 

The Lumped Parameter and One Dimensional PPM models, on the other hand, use an 
implcit method by which the incoming mass flux is determined based on the 
expected concentration profile resulting from the internal 
Dirichlet boundary condition at $r_i$. 

\subsubsection{Phase 4: Exterior Mass Balance}

When a mass flux $m_{ij}(t_n)$ is determined between volumes $i$ and $j$, the 
mass is added to the exterior sink volume $j$. Accordingly, necessary updates 
are made to the mass balance and concentration profile. 

For each of the models, this mass balance is calculated thus :



\subsubsection{Phase 5: Interior Mass Balance Update}
When a mass flux $m_{ij}(t_n)$ is determined between volumes $i$ and $j$, the 
mass is simultanously added to the exterior sink volume $j$ (as in phase 4) and 
extracted from the interior source volume $i$.

When the material is extracted from the interior source volume, an update is 
made to the mass balance and concentration profile.

\begin{align}
\end{align}

\subsubsection{old text}

At each component interface where mass transfer occurs and within each 
component where mass balances take place, the flow model is solved with the 
most up to date information available. 

That is, in Component $j$, some Component in a nested series, the mass flux 
entering the Component at time $t_n$ is found from the initial state of the cell 
at time $t_n$, the inner boundary 
condition at time $t_n$ and the outer boundary condition at $t_{n-1}$.  

\begin{align}
  \dot{m}_{ij}^n &= f( m_j(t_{n-1}) , BC_i(t_n) , BC_j(t_{n-1}) . . . ) \nonumber\\
  \intertext{where}
  m_{ij}(t_n) &= \mbox{ contaminant mass flux from component i to j }[kg/timestep]\nonumber\\
  BC_i(t_n)  &= \mbox{ inner conditions at }r_i\mbox{, and time }t_n \nonumber \\
  BC_j(t_{n-1})  &= \mbox{ outer conditions at }r_j\mbox{, and time }t_{n-1} \nonumber\\
  f &= \mbox{ functional form of contaminant transport into j. }\nonumber
\end{align}

Once the mass flux into the component is found, the mass is removed from the 
inner cell, updating its state in preparation for the next time step.

\begin{align}
  m_i^\dagger(t_n)  &= m_i(t_n)  - m_{ij}(t_n) 
  \intertext{where}
  m_i^\dagger(t_n)  &= \mbox{ updated mass in component i }[kg]
\end{align}

In this way, the contained mass in the component is described as
\begin{align}
  m_j(t_n)  &= m_j(t_{n-1})  + \dot{m}_j(t_n) . \nonumber
\end{align}

Resulting concentration profiles across the component can then be calculated 
and one can solve, numerically, for the outer boundary condition at $t_n$ 

\begin{align}
  BC_j(t_n) &= g\left( m_j(t_n) , C_j(t_n) \right)\nonumber\\
  g &= \mbox{functional form of contaminant transport across j}\nonumber
\end{align}

This boundary condition can, in turn, be used by the component external to it, $k$ as the $t_n$ 
inner boundary condition of its own solution and so on.

