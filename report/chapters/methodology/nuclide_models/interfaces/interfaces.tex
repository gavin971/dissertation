\subsection{Mass Transfer Modes}\label{sec:mass_transfer}

The results here provide an overview of the relative importance of thermal
parameters that affect the repository capacity of simplified generic
disposal concept in various geologic media where conduction is the dominant
heat transfer mode. The applicability of this sensitivity analysis is thus
restricted to enclosed, backfilled concepts.  

\subsection{Parametric Domain}

Sensitivity analyses were conducted which span the parametric range of values 
generated by the reference specific temperature change database and described 
in Table \ref{tab:thermal_cases}.  

\begin{table}[ht!]
\centering
\footnotesize{
\begin{tabular}{|l|l|l|r|}
\multicolumn{4}{c}{\textbf{Thermal Cases}}\\
\hline
\textbf{Parameter} & \textbf{Symbol} & \textbf{Units} & \textbf{Value Range} \\
\hline
Diffusivity & $\alpha_{th}$ & $[m^2\cdot s^{-1}]$ & $1.0\times10^{-7}-3.0\times10^{-6}$\\
\hline
Conductivity & $K_{th}$     & $[W\cdot m^{-1} \cdot K^{-1}]$ & $0.1 - 4.5$ \\
\hline
Spacing & $S$ & $[m]$ & 2, 5, 10, 15, 20, 25, 50 \\
\hline
Radius & $r_{lim}$ & $[m]$ & 0.1, 0.25, 0.5, 1, 2, 5 \\
\hline
Isotope & $i$ & $[-]$ & $^{241,243}Am,$  \\
        & & & $^{242,243,244,245,246}Cm,$  \\
        & & & $^{238,240,241,242}Pu$  \\
        & & & $^{134,135,137}Cs$  \\
        & & & $^{90}Sr$  \\
\hline
\end{tabular}
\caption{A thermal reference dataset of \gls{STC} values as a function of each of these parameters was generated by repeated parameterized runs of the LLNL 
MathCAD model\cite{greenberg_application_2012, greenberg_investigations_2012}.}
\label{tab:thermal_cases}
}
\end{table}



These values were selected to provide detail in the near field and at values of
$\alpha_{th}$ and $K_{th}$ in the three host media under consideration in this
work.

\subsection{Approach}

% used existing gdsms 
This analysis utilized the \gls{LLNL} semi-analytic MathCAD model
discussed in Section \ref{sec:llnl_background}.  It performs detailed
calculations of the conductive thermal transport in a generic repository
concept with a gridded layout.  

It relies on the thermal diffusivity, $\alpha_{th}$ and conductivity $K_{th}$ of 
the material as well as the waste package spacing, $S$, and thermally limiting 
radius, $r_{lim}$. Finally, it relies on the \gls{STC} data calculated with the 
semi-analytic model based on the decay heat profiles of the emplaced wastes, $Q$. 
The essential decay heat profiles, $Q$, were retrieved from a \gls{UFD} database 
provided by Carter et al. \cite{carter_fuel_2011}.



%\subsubsection{Advection Dominated Mass Transfer}\label{sec:adv_mass_transfer}
\subsubsection{Advection Dominated Mass Transfer}\label{sec:adv_mass_transfer}

The first, specified-concentration or Dirichlet type boundary conditions define 
a specified species concentration on some section of the boundary of the 
representative volume, 

    \begin{align}
      C(\vec{r},t)\Big|_{\vec{r} \in \Gamma} &= C_0(t)
      \intertext{where}
      \vec{r} &= \mbox{ position vector }\nonumber\\
      \Gamma &= \mbox{ domain boundary }.\nonumber
    \end{align}

The Dirichlet boundary condition can be provided by any mass balance model, 
based on the concentration profile calculated in the previous section,

\begin{align}
C(z,t_n)|_{z=r_j} &= \mbox{ fixed concentration in j at }r_j\mbox{ and }t_n [kg/m^3].\nonumber\\ 
                  &= \begin{cases} 
                         \frac{m_{d}(t_n)}{V_{d}(t_n)}, & \mbox{Degradation Rate}\\
                         \frac{m_{df}(t_n)}{V_{df}(t_n)}, & \mbox{Mixed Cell}\\
                         C_{out}(t_n), & \mbox{Lumped Parameter}\\
                         C(r_j,t_n), & \mbox{One Dimensional PPM}.
                      \end{cases}
\end{align}

In the Degradation Rate and Mixed Cell models, the Dirichlet boundary condition can 
be chosen to enforce an advective flux on the inner boundary. This choice is 
appropriate when the user expects a primarily advective interface between two 
components. The advective flux across the boundary between two components $i$ 
and $j$, 

\begin{align}
J_{adv,ij}(t_n) &= \mbox{ potential advective flux from i to j at }t_n[kg/m^2/s]\nonumber\\
               &= \theta_j v C(z,t_n)|_{z=r_i}
\end{align}
relies on the fixed concentration Dirichlet boundary condition at the 
interface, provided by the internal component.

The resulting mass transfer into the Degradation Rate or Mixed Cell model
is, therefore, 
\begin{align}
m_{ij}(t_n) &= A\Delta t \theta_j v C(z,t_n)|_{z=r_i}
\intertext{where}
A &= \mbox{ surface area normal to the flow direction }[m^2]\nonumber\\
\Delta t &= \mbox{ length of the time step }[s].\nonumber
\end{align}



%\subsubsection{Dispersion Dominated Mass Transfer}\label{sec:dis_mass_transfer}
\subsubsection{Explicit Dispersion Dominated Mass Transfer}\label{sec:diff_mass_transfer}


The second type, specified-flow or Neumann type boundary conditions describe a full set of 
concentration gradients at the boundary of the domain,
    \begin{align}
      \frac{\partial C(\vec{r},t)}{\partial r}\Big|_{\vec{r}\in\Gamma} &= f(t)\\
      f(t) &= \mbox{ known function }.\nonumber
    \end{align}

The Neumann boundary condition can be provided at the external boundary of any 
mass balance model,
\begin{align}
\frac{\partial C}{\partial z}\Bigg|_{z=r_j} &= \mbox{ fixed concentration gradient in j at }r_j\mbox{ and } t_n [kg/m^3/s].\nonumber
\end{align}


For mass balance models that are 0-dimensional in space (i.e. the Degradation 
Rate model and the Mixed Cell model), the center-to-center difference between 
adjacent components is taken,

\begin{align}
\frac{\partial C(z,t_n)}{\partial z}\Bigg|_{z=r_j} &= \frac{C_k(r_{k-1/2},t_{n-1}) - C_j(r_{j-1/2}, t_n)}{r_{k-1/2} - r_{j-1/2}}
\intertext{where}
r_{j-1/2} &= r_{j} - \frac{r_{j} - r_i}{2}\nonumber\\
r_{k-1/2} &= r_{k} - \frac{r_{k} - r_j}{2}.\nonumber
\end{align}

However, for mass balance models that are 1-dimensional in space (i.e. the Lumped Parameter model 
and the One Dimensional PPM model), the derivative is taken based on the 
concentration profile in the internal component as it approaches the boundary. 
For the lumped parameter model, where the profile is assumed to be a linear 
relationship between $C_{in}$ and $C_{out}$, the gradient is

\begin{align} 
\frac{\partial C(z,t_n)}{\partial z}\Bigg|_{z=r_j} &= \frac{C_{out} - C_{in}}{r_{j} - r_{i}}.
\end{align}

For the one dimensional permeable porous medium model, the analytical 
derivative of equation \eqref{simple_genuchten} is evaluated at $r_j$.

%\begin{align}
%derivative..
%\end{align}


In the Degradation Rate and Mixed Cell models, the Neumann boundary condition 
can be chosen to enforce a dispersive flux on the inner boundary. This choice 
is appropriate when the user expects a primarily dispersive flow across the 
boundary.
The dispersive flux in one dimension, 
\begin{align}
      J_{dis} &= \mbox{ Total Dispersive Mass Flux }[kg/m^2/s]\nonumber\\
      &= -\theta D\frac{\partial C}{\partial z}\Big|_{z=r_j} \nonumber
\end{align}
relies on the fixed gradient Neumann boundary condition at the interface, 
provided by the internal component. 

The resulting mass transfer into the Degradation Rate or Mixed Cell model is, 
therefore, 

\begin{align}
m_{ij}(t_n) &= - A\Delta t D \frac{\partial C(z,t_n)}{\partial z}|_{z=r_i}.
\end{align}

%\subsubsection{Coupled Advective Dispersive Mass Transfer}\label{sec:adv_dis_mass_transfer}
\subsubsection{Explicit Coupled Advective Dispersive Mass Transfer}\label{sec:adv_dif_mass_transfer}

The third Cauchy type mixed boundary condition defines a 
solute flux along a boundary.  The fixed concentration flux Cauchy boundary 
condition can be provided at the external boundary of any mass balance model.  
For a vertically oriented system with advective velocity in the $\hat{k}$ 
direction,

    \begin{align}
      -D\frac{\partial C(z, t)}{\partial z}\Big|_{z \in \Gamma} + v_zC(z, t) &= v_zC(t) 
      \intertext{where}
      C(t) &= \mbox{ a known concentration function }[kg/m^{3}].\nonumber
    \end{align}  

In the Degradation Rate and Mixed Cell models, the Cauchy boundary condition 
can be selected to enforce coupled advective and dispersive flow,

\begin{align}
  J_{coupled} &= J_{adv} + J_{dis} \nonumber\\
  &= \theta vC(z,t_n) -\theta D\frac{\partial C}{\partial z}.
\end{align}

The resulting mass transfer into the Degradation Rate or Mixed Cell model is then, 

\begin{align}
m_{jk}(t_n) &= A\Delta t \left( \theta_k v C(z,t_n)|_{z=r_j} - \theta_k D \frac{\partial C(z,t_n)}{\partial z}|_{z=r_j} \right).
\end{align}




\subsubsection{Maximum Flow Mass Transfer}\label{sed:max_flow_mass_transfer}

For debugging and testing purposes, the maximum flow mode transports all 
available material in a component into the component external to it. 

The total available mass for each mass balance model can be expressed,
\begin{align}
m_{ij}(t_n) &= \begin{cases}
                         m_{d}(t_n), & \mbox{Degradation Rate}\\
                         m_{df}(t_n), & \mbox{Mixed Cell}\\
                         \int C(z,t_n)dV, & \mbox{Lumped Parameter}\\
                         \int C(z,t_n)dV, & \mbox{One Dimensional PPM}.
               \end{cases}
\end{align}

The integrals for the Lumped Parameter model and the One Dimensional PPM model 
are calculated numerically.

\subsection{Implicit Timestepping}\label{sec:timestepping}
\subsection{Time Stepping Algorithm}\label{sec:time stepping}

In \Cyder, radionuclide contaminant flow is assumed to travel outward from 
the central Component (and up, in the $\hat{k}$ direction). In order to conduct 
a mass balance in each Component at each time step, the mass flow and mass 
balance calculations proceed from the innermost Component to the outermost 
Component. As mass flows from inner components to outer components, the mass 
balances in both components are updated.  Thus, nuclide release information is 
passed radially outward from the waste stream sequentially through each 
containment layer to the geosphere.  This implicit time stepping method 
arrives at the updated state of each Component, radially outward, as a function 
of both the past state and the current state of the system. 

At each component interface where mass transfer occurs and within each component 
where mass balances take place, the flow model is solved with the most up to 
date information available.  To illustrate the algorithm by which mass flow 
calculations are conducted through the system of components at each time step, we 
will walk through the phases of a single time step for a simple pair of 
components. The source, $i$, is the inner and the sink, $j$, is the outer
component. 

\subsubsection{Phase 1: Initial Conditions}

The initial conditions in both the source and the sink at the beginning of a 
time step are equal to the final updated state of the previous time step. If this 
is the first time step, the global intial state of the repository system is used. 

\subsubsection{Phase 2: Interior Mass Balance}

The mass distrubution and concentration profile in the interior source volume 
$i$ is solved based on the initial condition, any influxes, and the physics of 
its mass balance model.  This calculation results in a contaminant mass 
distribution and concentration profile within the volume $i$ at time $t_n$.  
For each of the models, the calculation behind this mass distribution and 
concentration profile is disscussed in Section \ref{sec:mass_balance}.

This mass distribution and concentration profile fully inform 
the conditions on the boundary at $r_i$ and this information is made available 
to the external component, $j$.


\subsubsection{Phase 3: Mass Transfer Calculation}

The mass transfer from the source volume $i$ to the sink volume $j$ is 
calculated next, based on the up to date conditions at $0\le r \le r_{i}$ 
determined in Phase 2 and the initial conditions in volume $j$ where $r_i \le r 
\le r_j$. The mass transfer is calculated according to the mass transfer mode 
preference of the mass balance model of volume $j$.  

The Degradation Rate and Mixed Cell models can be parameterized to utilize an 
explicit mass transfer mode that captures either advection, dispersion, or 
coupled flow.  The Lumped Parameter and One Dimensional PPM models, on the 
other hand, use an implcit method by which the incoming mass flux is determined 
based on the expected concentration profile resulting from the internal 
Dirichlet boundary condition at $r_i$. 

\subsubsection{Phase 4: Exterior Mass Balance}

When a mass flux $m_{ij}(t_n)$ is determined between volumes $i$ and $j$, the 
mass is added to the exterior sink volume $j$. Accordingly, necessary updates 
are made to the mass balance and concentration profile.  For each of the 
models, the calculation behind this mass distribution and concentration profile 
is disscussed in Section \ref{sec:mass_balance}.

\subsubsection{Phase 5: Interior Mass Balance Update}

When a mass flux $m_{ij}(t_n)$ is determined between volumes $i$ and $j$, the 
mass is simultanously added to the exterior sink volume $j$ (as in phase 4) and 
extracted from the interior source volume $i$.  When the material is extracted 
from the interior source volume, the contained mass distribution and 
concentration profile are updated to reflect this change.  

\subsubsection{old text}


That is, in Component $j$, some Component in a nested series, the mass flux 
entering the Component at time $t_n$ is found from the initial state of the cell 
at time $t_n$, the inner boundary 
condition at time $t_n$ and the outer boundary condition at $t_{n-1}$.  

\begin{align}
  \dot{m}_{ij}^n &= f( m_j(t_{n-1}) , BC_i(t_n) , BC_j(t_{n-1}) . . . ) \nonumber\\
  \intertext{where}
  m_{ij}(t_n) &= \mbox{ contaminant mass flux from component i to j }[kg/time step]\nonumber\\
  BC_i(t_n)  &= \mbox{ inner conditions at }r_i\mbox{, and time }t_n \nonumber \\
  BC_j(t_{n-1})  &= \mbox{ outer conditions at }r_j\mbox{, and time }t_{n-1} \nonumber\\
  f &= \mbox{ functional form of contaminant transport into j. }\nonumber
\end{align}

Once the mass flux into the component is found, the mass is removed from the 
inner cell, updating its state in preparation for the next time step.

\begin{align}
  m_i^\dagger(t_n)  &= m_i(t_n)  - m_{ij}(t_n) 
  \intertext{where}
  m_i^\dagger(t_n)  &= \mbox{ updated mass in component i }[kg]
\end{align}

In this way, the contained mass in the component is described as
\begin{align}
  m_j(t_n)  &= m_j(t_{n-1})  + \dot{m}_j(t_n) . \nonumber
\end{align}

Resulting concentration profiles across the component can then be calculated 
and one can solve, numerically, for the outer boundary condition at $t_n$ 

\begin{align}
  BC_j(t_n) &= g\left( m_j(t_n) , C_j(t_n) \right)\nonumber\\
  g &= \mbox{functional form of contaminant transport across j}\nonumber
\end{align}

This boundary condition can, in turn, be used by the component external to it, $k$ as the $t_n$ 
inner boundary condition of its own solution and so on.



%
%\subsection{Flow Mode Selection}
%
%  \begin{figure}[htp!]
%    \begin{center}
%      \def\svgwidth{\textwidth}
%      \input{./chapters/methodology/nuclide_models/interfaces/flow.eps_tex}
%    \end{center}
%    \caption[\Cyder Component interfaces provide a source term  and three 
%    boundary condition types.]{The boundaries between components (e.g., waste form and waste 
%      package) are robust interfaces defined by boundary condition types.}
%    \label{fig:flow}
%  \end{figure}
%
%
%
%The spatial concentration throughout the volume is sufficient to fully describe 
%implementation of advective, dispersive, and couple mass flow modes within 
%\Cyder. 
%
%\begin{align}
%\mathcal{C}_j(t_n) &= \mbox{ fixed concentration flux from component j at }t_n [kg/m^2/s]\nonumber\\
%                   &= -D\frac{\partial C(t_n)}{\partial r}\Bigg|_{r=r_j} + v_zC(t_n)\Bigg|_{r=r_j} \\
%\label{deg_rate_cauchy}
%\end{align}
%
%%<++> &= \mbox{ <++> }[<++>] \nonumber\\
%% mixed
%\subsection{Mixed Cell Calculation of Mass Transfer}
%
%The MixedCell model can operate in four modes, each dictating the method by 
%which mass transfer across the inner boundary is calculated.  Those modes 
%utilize the source term, Dirichlet, and Neumann interfaces to model prescribed, 
%flow, advective flow, or diffusive flow.
%Once calculated, that material object is removed from the 
%internal component ($m_{ij} = -\dot{m}_i$) such that its internal state can be 
%queried accurately in future time steps.  
%
%For the case in which source term is used on the inner boundary, the mass 
%transferred from component $i$ to component $j$ in time step $t_{n}$ is simply
%
%\begin{align} 
%m_{ij} &=  \mathcal{S}_i(t_n) \nonumber
%\intertext{where}
%\mathcal{S}_i(t_n) &= \mbox{ source term provided by component i at }t_n [kg].
%\end{align}
%
%For the case in which Dirichlet is used, the mass transferred is determined by 
%advection. One dimensional mass flux due to advection is the speed of flowing 
%water, $v_z$, scaled by the concentration of contaminants fixed by the Dirichlet 
%boundary condition, $C$, all integrated over the porous, degraded area perpendicular to 
%the flow, $\theta dxdy$,
%
%\begin{align}
%  m_{ij} &= \int_{t_{n-1}}^{t_n}\int_0^y\int_0^x\theta d v_z \mathcal{D}_i(t_n) dxdydt \label{mixed_adv}
%\intertext{which, for the Cyder components, becomes }
%m_{ij} &= \theta d v_z C 2rl (t_n - t_{n-1})\nonumber\\
%\intertext{where}
%\mathcal{D}_i(t_n) &= \mbox{ fixed C from component i at }t_n [kg/m^3]\nonumber\\
%r &= \mbox{ radius of the cylinder }[m]\nonumber\\
%l &= \mbox{ length of the cylinder }[m].\nonumber
%\end{align}
%
%For the case in which Neumann is chosen, the mass transfer is taken to be 
%dispersive, 
%
%\begin{align}
%  m_{ij} &= \int_{t_{n-1}}^{t_n}\int_0^y\int_0^x -D \theta d \mathcal{N}_i(t_n) dxdydt \label{mixed_adv}\\
%         &= \int_{t_{n-1}}^{t_n}\int_0^y\int_0^x -D \theta d \frac{\partial C}{\partial z}\Bigg|_{z=r_j} dxdydt \nonumber
%\intertext{which, for the Cyder components, becomes }
%m_{ij} &= -D \theta d \frac{\partial C}{\partial z}\Bigg|_{z=r_j} 2rl(t_n - t_{n-1}).\nonumber
%\end{align}
%
%\subsection{Mixed Cell Boundary Interfaces}
%The source term of available contaminants is all mass in the available degraded fluid,
%\begin{align}
%\mathcal{S}_j(t_n) &= m_{df}(t_n). 
%\end{align}
%The desired boundary conditions can be expressed in terms of $m_{df}$. First, the 
%Dirichlet boundary condition is 
%\begin{align}
%\mathcal{D}_j(t_n) &= C_j(t_n)\nonumber\\ 
% &= \frac{m_{df}(t_n)}{V_{df}(t_n)}.
%\label{dirichlet_mixed}
%\end{align}
%
%From this boundary condition in combination with global advective velocity 
%data, porosity data,  and elemental dispersion coefficient data, all other 
%boundary conditions can be found. The Neumann boundary condition generated at 
%the external boundary of cell $j$ relies on up to date data from cell $k$ and 
%on internal state data from the previous time step, such that 
%
%\begin{align}
%\mathcal{N}_j(t_n)&= \frac{dC(t_n)}{dr}\Bigg|_{r=r_j}\nonumber\\ 
%                  &= \frac{C_k(r_{k-1/2},t_{n-1}) - C_i(r_{j-1/2}, t_n)}{r_{k-1/2} - r_{j-1/2}}
%\label{neumann_mixed}
%\intertext{where}
%r_{j-1/2} &= r_{j} - \frac{r_{j} - r_{i}}{2}.\nonumber\\
%r_{k-1/2} &= r_{k} - \frac{r_{k} - r_{j}}{2}.\nonumber
%\end{align}
%
%This expression for the concentration gradient can also be used in the Cauchy 
%boundary condition, which relies on the advective velocity and concentration 
%profile as well as the concentration gradient,
%
%\begin{align}
%v_z C_0 &= \frac{dC(t_n)}{dr}\Big|_{r=r_{j}} + v_{z}C_j(t_n).
%\label{cauchy_mixed}
%\end{align}
%
%
%% lumped
%\subsection{Lumped Parameter Calculation of Mass Transfer}
%
%The Lumped Parameter model requires a specified internal concentration, so the 
%Dirichlet boundary condition is queried at the internal boundary of the lumped 
%parameter nuclide model. To calculate the resulting mass transfer over a 
%time step, the response function is applied and a linear concentration profile 
%is made across the cell. The concentration profile combined with 
%information about the initial state and the water volume in the cell, can be 
%integrated over the volume to arrive at a resulting mass in the cell,
%
%\begin{align}
%m_j(t_n) &= \int_0^V \theta C(t_n, r) dV \\\label{lp_mass}
%         &= \int_{r_i}^{r_j} \int_0^{2\pi} \int_0^h \theta C(t_n, r) r dr d\phi dh\nonumber \\
%         &= 2\pi l\theta \int_{r_i}^{r_j} C(t_n, r)rdr \nonumber\\
%         &= 2\pi l\theta \int_{r_i}^{r_j}\left( \frac{C_j(t_n) - C_i(t_n)}{r_j - r_i}r^2 + C_j(t_n)r \right) dr\nonumber
%\intertext{such that}
%m_j(t_n) &= 2\pi l \theta \left[ \frac{C_j(t_n) - C_i(t_n)}{3\left(r_j - r_i\right)}r^3 + C_j(t_n-1) \frac{r^2}{2}\right]_{r_i}^{r_j} \nonumber\\
%         &= 2\pi l \theta \left[ \frac{C_j(t_n) - C_i(t_n)}{3\left(r_j - r_i\right)}(r_j-r_i)^3 + C_j(t_n) \frac{(r_j-r_i)^2}{2}\right]\nonumber\\ 
%         &= 2\pi l \theta (r_j-r_i)^2 \left[ \frac{C_j(t_n) - C_i(t_n)}{3} + \frac{C_j(t_n)}{2} \right]\nonumber\\ 
%         &= 2\pi l \theta (r_j-r_i)^2 \left[ \frac{5C_j(t_n)}{6} - \frac{C_i(t_n)}{3} \right]. 
%\end{align}
%
%Using this expression for $m_j$, the necessary mass transfer from $m_i$ is 
%simply
%
%\begin{align}
%m_{ij} &= m_j(t_n) - m_j(t_{n-1}).
%\end{align}
%
%\subsection{Lumped Parameter Boundary Interfaces}
%The external source term and boundary conditions are found exactly similarly to the method by 
%which the Degradation Rate model finds external boundary conditions in Section 
%\ref{sec:dr_bc}. This method is entirely based on contained mass and component 
%volume.  
