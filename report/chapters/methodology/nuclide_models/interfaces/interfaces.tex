\section{Interfaces}
The interfaces between the models are essential to the understanding of the 
models themselves. The interfaces define boundary conditions in a number of 
forms based on information available internally to the component implementation. 

In groundwater transport, contaminants are transported by dispersion 
and advection. It is customary to define the combination of molecular diffusion 
and mechanical mixing as the dispersion tensor, $D$, such that, for a 
conservative solute (infinitely soluble and non-sorbing), the mass conservation equation becomes 
\cite{schwartz_fundamentals_2004, wang_introduction_1982, 
van_genuchten_analytical_1982}:

    \begin{align}
      J &= J_{dis} + J_{adv}\nonumber\\
      &= -\theta(D_{mdis} + \tau D_m)\nabla C + \theta vC\nonumber\\ 
      &= -\theta D\nabla C + \theta vC \nonumber\\ 
      \intertext{which, for uniform flow in $\hat{k}$, is}
      &=\left(-\theta D_{xx} \frac{\partial C}{\partial x}
             \right)\hat{\imath}
             + \left( -\theta D_{yy} \frac{\partial C}{\partial y}
            \right)\hat{\jmath}
            + \left( -\theta D_{zz} \frac{\partial C}{\partial z}
             + \theta v_zC 
            \right)\hat{k},
      \label{unidirflow}
      \intertext{where}
      J_{dis} &= \mbox{ Total Dispersive Mass Flux }[kg/m^2/s]\nonumber\\
      J_{adv} &= \mbox{ Advective Mass Flux }[kg/m^2/s]\nonumber\\
      \tau &= \mbox{ Tortuosity }[-] \nonumber\\
      \theta &= \mbox{ Porosity }[-] \nonumber\\
      D_m &= \mbox{ Molecular diffusion coefficient }[m^2/s]\nonumber\\
      D_{mdis} &= \mbox{ Coefficient of mechanical dispersivity}[m^2/s]\nonumber\\
      D &= \mbox{ Effective Dispersion Coefficient }[m^2/s]\nonumber\\
      C &= \mbox{ Concentration }[kg/m^3]\nonumber\\
      v &= \mbox{ Fluid Velocity in the medium }[m/s].\nonumber
    \end{align}

Solutions to this equation can be categorized by their boundary conditions and 
those boundary conditions serve as the interfaces between components in the 
\Cyder library of nuclide transport models.

  \begin{figure}[htp!]
    \begin{center}
      \def\svgwidth{\textwidth}
      \input{./chapters/methodology/nuclide_models/interfaces/flow.eps_tex}
    \end{center}
    \caption[\Cyder Component interfaces provide a source term  and three 
    boundary condition types.]{The boundaries between components (e.g., waste form and waste 
      package) are robust interfaces defined by boundary condition types.}
    \label{fig:flow}
  \end{figure}

The first, specified-concentration or Dirichlet type boundary conditions define a specified species 
concentration on some section of the boundary of the representative volume, 

    \begin{align}
      C(\vec{r},t)\Big|_{\vec{r} \in \Gamma} &= C_0(t)
      \intertext{where}
      \vec{r} &= \mbox{ position vector }\nonumber\\
      \Gamma &= \mbox{ domain boundary }.\nonumber
    \end{align}

The second type, specified-flow or Neumann type boundary conditions describe a full set of 
concentration gradients at the boundary of the domain,

    \begin{align}
      \frac{\partial C(\vec{r},t)}{\partial r}\Big|_{\vec{r}\in\Gamma} &= f(t)\\
      f(t) &= \mbox{ known function }.\nonumber
    \end{align}
    

The third, head-dependent mixed boundary condition or Cauchy type, defines a solute 
flux along a boundary. For a vertically oriented system with advective velocity 
in the $\hat{z}$ direction,

    \begin{align}
      -D\frac{\partial C(z, t)}{\partial z}\Big|_{z \in \Gamma} + v_zC(z, t) &= v_zC(t) 
      \intertext{where}
      C(t) &= \mbox{ a known concentration function }[kg/m^{3}].\nonumber
    \end{align}  

The spatial concentration throughout the volume is sufficient to fully describe 
implementation of the following nuclide transport models within \Cyder. This is 
supported by implementation in which vertical advective velocity is uniform 
throughout the system and in which parameters such as the dispersion coefficient 
are known for each component. Since this is the case in \Cyder, description of 
the Dirichlet condition is sufficient to fully define calculation of the Neumann 
and Cauchy type conditions.


\subsection{Degradation Rate Mass Flow Calculations}
The concentration calculation results from the mass balance calculation in 
\eqref{deg_rate_source_cont} 
to support parent components that utilize the Dirichlet boundary condition. 
For 
the degradation rate model, which incorporates no diffusion or advection, the 
concentration, $C_j$ at $r_j$, the boundary between cells $j$ and $k$, is the average 
concentration in degraded volume, 

\begin{align}
\mathcal{D}_j(t_n) &= \mbox{ fixed concentration from component j at }t_n [kg/m^3].\nonumber\\ 
                   &= C_{j}(t_n)\nonumber\\ 
                   &= C_{df}\nonumber\\
                   &= \frac{m_{d}(t_n)}{V_{d}(t_n)}\\
\label{deg_rate_dirichlet}\\
&= \frac{\mbox{ solute mass in degraded fluid in cell j }}{\mbox{ degraded fluid volume in cell j}}.\nonumber 
\end{align}

To support parent components that utilize the Neumann boundary condition, the 
concentration gradient can be found if the concentration and the 
radial midpoint of the external component, $k$, are specified.

\begin{align}
\mathcal{N}_j(t_n) &= \mbox{ fixed concentration gradient from component j at }t_n [kg/m^3/s]\nonumber\\
                   &= \frac{dC(t_n)}{dr}\Bigg|_{r=r_j}\nonumber\\ 
                   &= \frac{C_k(r_{k-1/2},t_{n-1}) - C_j(r_{j-1/2}, t_n)}{r_{k-1/2} - r_{j-1/2}}
\label{neumann_dr}
\intertext{where}
r_{j-1/2} &= r_{j} - \frac{r_{j} - r_i}{2}\nonumber\\
r_{k-1/2} &= r_{k} - \frac{r_{k} - r_j}{2}.\nonumber
\end{align}


To support parent components that utilize the Cauchy boundary condition, the 
degradation rate model assumes that the fluid velocity is constant across the cell 
as is the concentration. Thus, 

\begin{align}
\mathcal{C}_j(t_n) &= \mbox{ fixed concentration flux from component j at }t_n [kg/m^2/s]\nonumber\\
                   &= -D\frac{\partial C(t_n)}{\partial r}\Bigg|_{r=r_j} + v_zC(t_n)\Bigg|_{r=r_j} \\
\label{deg_rate_cauchy}
\end{align}

%<++> &= \mbox{ <++> }[<++>] \nonumber\\
% mixed
\subsection{Calculation of Mass Transfer}
The MixedCell model can operate in four modes, each dictating the method by 
which mass transfer across the inner boundary is calculated.  Those modes 
utilize the source term, Dirichlet, and Neumann interfaces to model prescribed, 
flow, advective flow, or diffusive flow.
Once calculated, that material object is removed from the 
internal component ($m_{ij} = -\dot{m}_i$) such that its internal state can be 
queried accurately in future timesteps.  

For the case in which source term is used on the inner boundary, the mass 
transferred from component $i$ to component $j$ in timestep $t_{n}$ is simply

\begin{align} 
m_{ij} &=  \mathcal{S}_i(t_n) \nonumber
\intertext{where}
\mathcal{S}_i(t_n) &= \mbox{ source term provided by component i at }t_n [kg].
\end{align}

For the case in which Dirichlet is used, the mass transferred is determined by 
advection. One dimensional mass flux due to advection is the speed of flowing 
water, $v_z$, scaled by the concentration of contaminants fixed by the Dirichlet 
boundary condition, $C$, all integrated over the porous, degraded area perpendicular to 
the flow, $\theta dxdy$,

\begin{align}
  m_{ij} &= \int_{t_{n-1}}^{t_n}\int_0^y\int_0^x\theta d v_z \mathcal{D}_i(t_n) dxdydt \label{mixed_adv}
\intertext{which, for the Cyder components, becomes }
m_{ij} &= \theta d v_z C 2rl (t_n - t_{n-1})\nonumber\\
\intertext{where}
\mathcal{D}_i(t_n) &= \mbox{ fixed C from component i at }t_n [kg/m^3]\nonumber\\
r &= \mbox{ radius of the cylinder }[m]\nonumber\\
l &= \mbox{ length of the cylinder }[m].\nonumber
\end{align}

For the case in which Neumann is chosen, the mass transfer is taken to be 
dispersive, 

\begin{align}
  m_{ij} &= \int_{t_{n-1}}^{t_n}\int_0^y\int_0^x -D \theta d \mathcal{N}_i(t_n) dxdydt \label{mixed_adv}\\
         &= \int_{t_{n-1}}^{t_n}\int_0^y\int_0^x -D \theta d \frac{\partial C}{\partial z}\Bigg|_{z=r_j} dxdydt \nonumber
\intertext{which, for the Cyder components, becomes }
m_{ij} &= -D \theta d \frac{\partial C}{\partial z}\Bigg|_{z=r_j} 2rl(t_n - t_{n-1}).\nonumber
\end{align}

\subsection{Boundary Interfaces}
The source term of available contaminants is all mass in the available degraded fluid,
\begin{align}
\mathcal{S}_j(t_n) &= m_{df}(t_n). 
\end{align}
The desired boundary conditions can be expressed in terms of $m_{df}$. First, the 
Dirichlet boundary condition is 
\begin{align}
\mathcal{D}_j(t_n) &= C_j(t_n)\nonumber\\ 
 &= \frac{m_{df}(t_n)}{V_{df}(t_n)}.
\label{dirichlet_mixed}
\end{align}

From this boundary condition in combination with global advective velocity 
data, porosity data,  and elemental dispersion coefficient data, all other 
boundary conditions can be found. The Neumann boundary condition generated at 
the external boundary of cell $j$ relies on up to date data from cell $k$ and 
on internal state data from the previous time step, such that 

\begin{align}
\mathcal{N}_j(t_n)&= \frac{dC(t_n)}{dr}\Bigg|_{r=r_j}\nonumber\\ 
                  &= \frac{C_k(r_{k-1/2},t_{n-1}) - C_i(r_{j-1/2}, t_n)}{r_{k-1/2} - r_{j-1/2}}
\label{neumann_mixed}
\intertext{where}
r_{j-1/2} &= r_{j} - \frac{r_{j} - r_{i}}{2}.\nonumber\\
r_{k-1/2} &= r_{k} - \frac{r_{k} - r_{j}}{2}.\nonumber
\end{align}

This expression for the concentration gradient can also be used in the Cauchy 
boundary condition, which relies on the advective velocity and concentration 
profile as well as the concentration gradient,

\begin{align}
v_z C_0 &= \frac{dC(t_n)}{dr}\Big|_{r=r_{j}} + v_{z}C_j(t_n).
\label{cauchy_mixed}
\end{align}


% lumped
\subsection{Calculation of Mass Transfer}

The LumpedNuclide model requires a specified internal concentration, so the 
Dirichlet boundary condition is queried at the internal boundary of the lumped 
parameter nuclide model. To calculate the resulting mass transfer over a 
timestep, the response function is applied and a linear concentration profile 
is made across the cell. The concentration profile combined with 
information about the initial state and the water volume in the cell, can be 
integrated over the volume to arrive at a resulting mass in the cell,

\begin{align}
m_j(t_n) &= \int_0^V \theta C(t_n, r) dV \\\label{lp_mass}
         &= \int_{r_i}^{r_j} \int_0^{2\pi} \int_0^h \theta C(t_n, r) r dr d\phi dh\nonumber \\
         &= 2\pi l\theta \int_{r_i}^{r_j} C(t_n, r)rdr \nonumber\\
         &= 2\pi l\theta \int_{r_i}^{r_j}\left( \frac{C_j(t_n) - C_i(t_n)}{r_j - r_i}r^2 + C_j(t_n)r \right) dr\nonumber
\intertext{such that}
m_j(t_n) &= 2\pi l \theta \left[ \frac{C_j(t_n) - C_i(t_n)}{3\left(r_j - r_i\right)}r^3 + C_j(t_n-1) \frac{r^2}{2}\right]_{r_i}^{r_j} \nonumber\\
         &= 2\pi l \theta \left[ \frac{C_j(t_n) - C_i(t_n)}{3\left(r_j - r_i\right)}(r_j-r_i)^3 + C_j(t_n) \frac{(r_j-r_i)^2}{2}\right]\nonumber\\ 
         &= 2\pi l \theta (r_j-r_i)^2 \left[ \frac{C_j(t_n) - C_i(t_n)}{3} + \frac{C_j(t_n)}{2} \right]\nonumber\\ 
         &= 2\pi l \theta (r_j-r_i)^2 \left[ \frac{5C_j(t_n)}{6} - \frac{C_i(t_n)}{3} \right]. 
\end{align}

Using this expression for $m_j$, the necessary mass transfer from $m_i$ is 
simply

\begin{align}
m_{ij} &= m_j(t_n) - m_j(t_{n-1}).
\end{align}

\subsection{Boundary Interfaces}
The external source term and boundary conditions are found exactly similarly to the method by 
which the Degradation Rate model finds external boundary conditions in Section 
\ref{sec:dr_bc}. This method is entirely based on contained mass and component 
volume.  
