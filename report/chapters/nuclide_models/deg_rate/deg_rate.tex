\section{Degradation Rate Radionuclide Transport Model}\label{sec:deg_rate}

The materials that constitute the engineered and natural barriers in a saturated 
repository environment degrade over time. One abstract model of the nuclide 
release behavior of each of the engineered barrier components in the repository 
is to represent the available source term at the outer boundary as solely a 
function of the degradation rate of that component.

For the case in which all engineered barrier components are represented by 
degradation rate models, the source term at the outermost edge will be a 
function of the original central source and the degradation rates of the 
components. This results in a description of the outer edge source terms, 
from cell $i$ to cell $j$ 

\begin{align}
\dot{m}_{i,j}(t) &= f_iM_i(t)
\label{deg_rate}
\end{align}

For a situation as in \Cyclus, with discrete timesteps, this becomes :
\begin{align}
m_{i,j}(t_n) = f_i(t_n,\cdots)
m_0(t_n) = m_0(t_0) - \sum_{i=0}^n m_{0,1}(t_n)\\
\label{source}
\end{align}


\begin{align}
\dot{m}_{i,j}(t) &= f_i(t)M_i(t)
\label{deg_rate}
\end{align}

\begin{align}
\dot{m}_{i,j}(t) &= f_i(\cdots)M_i(t)
\label{deg_rate}
\end{align}
