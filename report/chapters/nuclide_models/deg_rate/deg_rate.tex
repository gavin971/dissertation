\section{Degradation Rate Radionuclide Transport Model}\label{sec:deg_rate}

The materials that constitute the engineered and natural barriers in a saturated 
repository environment degrade over time. One abstract model of the nuclide 
release behavior can be based solely on a fractional degradation rate. This 
model incorporates and source term made available on the inner boundary into its 
available mass and defines the resulting boundary conditions at the outer boundary 
as solely a function of the degradation rate of that component.

For the case in which all engineered barrier components are represented by 
degradation rate models, the source term at the outermost edge will be solely
a function of the original central source and the degradation rates of the 
components. This results in the following expression for the mass transfer, 
$m_{ij}(t)$, from cell $i$ to cell $j$ at time $t$ :

\begin{align}
\dot{m}_{ij}(t) &= f_i(\cdots)m_i(t)
\intertext{where}
\dot{m}_{ij} &= \mbox{ the rate of mass transfer from i to j }[kg/s]\nonumber\\
f_i &= \mbox{ fractional degradation rate }[1/s] \nonumber\\
m_i &= \mbox{ mass in cell i }[kg] \nonumber\\
t &= \mbox{ time  }[s] \nonumber
\label{deg_rate}
\end{align}

For a situation as in \Cyclus, with discrete timesteps, we can assume the 
timesteps are small enough to assume a constant rate $m_{ij}$ over the course of 
the timestep. Equation \eqref{deg_rate} is integrated over the timestep to give 
the mass transferred per timestep

\begin{align}
m_{ij}^n &= \int_{t_{n-1}}^{t_n}\dot{m}_{ij}(t')dt' 
\intertext{assuming a constant transfer rate}
         &= \int_{t_{n-1}}^{t_n} f_i(\cdots)m_i^{n-1}dt'.\\
         &= f_i(\cdots)m_i^{n-1}\left(t_n - t_{n-1}\right).
\label{discrete}
\end{align}

The concentration boundary condition must also be defined at the outer boundary 
to support parent components that utilize the Dirichlet boundary condition. For 
the degradation model, which incorporates no diffusion or advection, the 
concentration at the boundary is the average concentration in the saturated pore 
volume,
\begin{align}
C_{ij}^n &= \\frac{m_i^n}{V_{vi}}\\
 &= \mbox{ <++> }[<++>] \nonumber\\
<++> &= \mbox{ <++> }[<++>] \nonumber\\
<++> &= \mbox{ <++> }[<++>] \nonumber\\
<++> &= \mbox{ <++> }[<++>] \nonumber\\
\label{<++>}
\end{align}




<++> &= \mbox{ <++> }[<++>] \nonumber\\
