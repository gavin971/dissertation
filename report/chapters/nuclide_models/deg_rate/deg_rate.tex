\section{Degradation Rate Radionuclide Transport Model}\label{sec:deg_rate}

The materials that constitute the engineered and natural barriers in a saturated 
repository environment degrade over time. One abstract model of the nuclide 
release behavior of each of the engineered barrier components in the repository 
represents the available source term at the outer boundary as solely a 
function of the degradation rate of that component.

For the case in which all engineered barrier components are represented by 
degradation rate models, the source term at the outermost edge will be solely
a function of the original central source and the degradation rates of the 
components. This results in the following expression for the mass transfer, 
$m_{i,j}(t)$, from cell $i$ to cell $j$ at time $t$ :

\begin{align}
m_{i,j}(t) &= f_i(\cdots)m_i(t)
\label{deg_rate}
\end{align}

For a situation as in \Cyclus, with discrete timesteps, $t_n$, this becomes :

\begin{align}
m_{i,j}^n = f_i(\cdots)m_i^n.
\label{discrete}
\end{align}

The remaining mass in cell i at the end of timestep $t_n$ is the mass of the 
cell at the beginning of timestep $t_{n+1}$. In general, the mass balance 
equation for component $k$ is simply the sum over time of incoming and outgoing 
mass. No component has more than one external component ($l$), but some have 
many internal components ($j$, children)

\begin{align}
m_k^{n+1} &= m_k^0 + \sum_{i=0}^n\left[ \sum_{j=0}^{children}\left( m_{j,k}^i\right) - m_{k,l}^i\right]
\label{deg_rate}
\end{align}

For the central source cell, which has no children

\begin{align}
m_0^{n+1} = m_0^0 - \sum_{i=0}^n m_{0,1}^n.
\label{source}
\end{align}



\begin{align}
m_{i,j}^n &= f_i(\cdots)m_i^n
\label{deg_rate}
\end{align}
