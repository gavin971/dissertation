The interfaces between the models are essential to the understanding of the 
models themselves. The interfaces define boundary conditions in a number of 
forms based on information available internally to the component implementation. 

In a saturated, reducing environment, contaminants are transported by 
dispersion and advection. It is customary to define the combination 
of molecular diffusion and mechanical
mixing as the dispersion tensor, $D$, such that the mass conservation equation 
becomes \cite{schwartz_fundamentals_2003, wang_introduction_1982, van_genuchten_analytical_1982}:

    \begin{align}
      J &= J_{dis} + J_{adv}\nonumber\\
      &= -\theta(D_{mdis} + \tau D_m)\nabla C + \theta vC\nonumber\\ 
      &= -\theta D\nabla C + \theta vC \nonumber\\ 
      \intertext{which, for uniform flow in $\hat{k}$, is}
      &=\left(-\theta D_{xx} \frac{\partial C}{\partial x}
             \right)\hat{\imath}
             + \left( -\theta D_{yy} \frac{\partial C}{\partial y}
            \right)\hat{\jmath}
            + \left( -\theta D_{zz} \frac{\partial C}{\partial z}
             + \theta v_zC 
            \right)\hat{k},
      \label{unidirflow}
      \intertext{where}
      J_{dis} &= \mbox{ Total Dispersive Mass Flux }[kg/m^2/s]\nonumber\\
      J_{adv} &= \mbox{ Advective Mass Flux }[kg/m^2/s]\nonumber\\
      \tau &= \mbox{ Toruosity }[-] \nonumber\\
      \theta &= \mbox{ Porosity }[\%] \nonumber\\
      D_m &= \mbox{ Molecular diffusion coefficient }[m^2/s]\nonumber\\
      D_{mdis} &= \mbox{ Coefficient of mechanical dispersivity}[m^2/s]\nonumber\\
      D &= \mbox{ Effective Dispersion Coefficient }[m^2/s]\nonumber\\
      C &= \mbox{ Concentration }[kg/m^3]\nonumber\\
      v &= \mbox{ Fluid Velocity in the medium }[m/s].\nonumber
    \end{align}

Solutions to this equation can be categorized by their boundary conditions and 
those boundary conditions serve as the interfaces between components in the 
Cyder library of nuclide transport models.

  \begin{figure}[htp!]
    \begin{center}
      \def\svgwidth{\textwidth}
      \input{./interfaces/flow.eps_tex}
    \end{center}
    \caption{The boundaries between components (in this case, waste form (wf) and 
      waste package (wp) components) are robust interfaces defined by 
    Source Term, Dirichlet, Neumann, and Cauchy boundary conditions.}
    \label{fig:flow}
  \end{figure}

In addition to a specified source term (the zeroth type boundary condition, 
perhaps), the first, specified-head or Dirichlet type boundary conditions define a specified species 
concentration on some section of the boundary of the representative volume, 

    \begin{align}
      C(\vec{r},t) = C_0(\vec{r},t)\hspace{1mm}\mbox{ for } \vec{r} \in 
      \Gamma.
    \end{align}

The second type, specified-flow or Neumann type boundary conditions describe a full set of 
concentration gradients at the boundary of the domain,

    \begin{align}
      \frac{\partial C(\vec{r},t)}{\partial r} &= \theta D\vec{J}(t) \hspace{1mm}\mbox{ for } 
      \vec{r} \in \Gamma
      \intertext{where}
      \vec{r} &= \mbox{ position vector }\nonumber\\
      \Gamma &= \mbox{ domain boundary }\nonumber\\
      \vec{J}(t) &= \mbox{ solute mass flux } [kg/m^2\cdot s].\nonumber
    \end{align}
    

The third, head-dependent mixed boundary condition or Cauchy type, defines a solute 
flux along a boundary,

    \begin{align}
      -D\frac{\partial C}{\partial z} + v_zC &= v_zC(\vec{r},t) 
      \hspace{1mm}\mbox{ for }\vec{r} \in \Gamma.
    \end{align}  

The spatial concentration throughout the volume is sufficient to fully describe 
implementation of the following nuclide transport models within Cyder. This is 
supported by the implementation in which vertical advective velocity is uniform 
throughout the system and in which parameters such as the dispersion coefficient 
are known for each component. Since this is the case in Cyder, description of 
the Dirichlet condition is sufficient to fully define calculation of the Neumann 
and Cauchy type conditions.


