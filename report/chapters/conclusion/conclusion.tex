\chapter{Conclusions}\label{ch:conclusion}
\section{Contributions}

This work has provided a flexible code for rapid medium fidelity calculation of 
generic repository performance in the context of fuel cycle analysis.  Capable 
of thermal transport, hydrologic contaminant transport, and integration 
within a fuel cycle simulation code, \Cyder is the first of its kind.  

In addition to implementing fundamental modeling capabilities, \Cyder has been 
designed to accommodate the development of advanced capabilities in the future.

In this work, key conceptual components and modeling methods for geologic 
radioactive waste disposal were identified as part of a literature review, 
dominant physics of thermal and radionuclide transport were identified by 
conducting sensitivity analyses with detailed codes. Accordingly, a basic set 
of abstracted models were developed and implemented within the \Cyder code. 

A set of basic capabilities within the \Cyder library have been developed and 
validated and an assortment of advanced features, data, testing, and plotting 
capabilities are functional.  The \Cyder source code in which these models are 
implemented  is made freely available to interested researchers and potential 
model developers \cite{huff_cyder_2013}. In addition to the source code and 
supporting publications, the \Cyder code is well commented and produces 
clickable, browsable automated documentation with each build. That 
documentation is also available online.

The application programming interface to this software library is intentionally 
general, facilitating the incorporation of the models presented here within 
external software tools in need of a multicomponent disposal system simulator. 

Furthermore, this work contributes to an expanding ecosystem of computational 
models available for use with the \Cyclus fuel cycle simulator. This hydrologic 
nuclide transport library, by virtue of its capability to modularly integrate 
with the \Cyclus fuel cycle simulator has laid the foundation for integrated 
disposal option analysis in the context of fuel cycle options. 

\section{Suggested Future Work}
It is hoped that \Cyder will benefit from continued development and use. Future 
development efforts will likely be led by developer use cases, but are likely 
to include a number of advanced features that have the potential to extend the 
capabilities of this tool in significant ways. 

Initially, further validation of these models should include full benchmarks 
against the \gls{GDSM} results including biosphere conversion of the released 
source term.  Furthermore, thermal benchmarks against recent \gls{UFD} work for 
various design concepts would similarly improve the understanding of the range 
of validity for the thermal model. 

Thermal analyses in these results have been used to assess thermal performance 
of a repository after emplacement. However, dynamic, thermal capacity limited 
fuel cycle analyses concerning the variation of necessary cooling times among 
repository concepts and fuel cycles should be conducted using the capacity 
determination capability arrived at with this model.  

Additional advanced capabilities should include the incorporation of fracture 
enabled transport in a radionuclide transport model. This feature would improve 
analyses of geologic host media such as granite which exhibit significant 
cracking. Similarly, incorporation of a biosphere model in the far field would 
substantively benefit the calculation of fuel cycle metrics related to human and 
environmental effects and will support myriad expected use cases of the tool.

Additional radionuclide transport models, thermal transport models, and 
supporting data will enrich the capabilities of this code. 

