\section{Radionuclide Transport Base Cases}\label{sec:nuclide_base_cases}
\subsection{No Release Problem Specification}
The no release base cases tested basic null containment transport behavior of 
all the radionuclide transport models at each component interface. This test 
neglected thermal transport and capacity estimation to simplify validation. 

The problem design includes : 
\begin{itemize}
\item{A source facility providing one waste stream per timestep}
\item{A legislated repository capacity of 5 1kg waste streams}
\item{A waste form Component} 
\item{A waste package Component}
\item{A buffer Component}
\item{A far field Component}
\end{itemize}

\subsubsection{Degradation Rate Model}
The Degradation Rate model should not release contaminants if the degradation 
rate is 0. Thus, four simulations were run to demonstrate the containment 
behavior at the Waste Form, Waste Package, Buffer, and Far Field interfaces. 

\begin{table}
\centering
\footnotesize{
\begin{tabularx}{\textwidth}{|X|c|c|r|r|}
  \multicolumn{5}{c}{\textbf{Degradation Rate Model No Release Contaminant Transport}}\\
  \hline
  \textbf{Case}  &  \textbf{Component} &  \textbf{Degradation Rate} & \textbf{Expected 10 yrs} & \textbf{Actual 10 yrs}\\
  \textbf{ID}    & \textbf{[Type]} &  \textbf{$[yr^{-1}]$}  &  $[\%]$  & $[\%]$\\
  \hline
  DRI     &  WF    &  0   & 1 & 1\\
          &  WP    &  0.1 & 0 & 0 \\
          &  BUFF  &  0.1 & 0 & 0 \\
          &  FF    &  0.1 & 0 & 0\\
  \hline
  DRII    &  WF    &  0.1 & 0 & 0\\
          &  WP    &  0   & 1 & 1\\
          &  BUFF  &  0.1 & 0 & 0\\
          &  FF    &  0.1 & 0 & 0\\
  \hline
  DRIII   &  WF    &  0.1 & 0 & 0\\
          &  WP    &  0.1 & 0 & 0\\
          &  BUFF  &  0   & 1 & 1\\
          &  FF    &  0.1 & 0 & 0\\
  \hline
  DRIV    &  WF    &  0.1 & 0 & 0\\
          &  WP    &  0.1 & 0 & 0\\
          &  BUFF  &  0.1 & 0 & 0\\
          &  FF    &  0   & 1 & 1\\
  \hline
\end{tabularx}
\caption[Degradation rate model no release problem results.]{Results from demonstration cases for non-release from 0-degradation Degradation Rate modeled Components.}
\label{tab:dr_no_release}
}
\end{table}


\subsubsection{Mixed Cell Model}

\begin{table}
\centering
\begin{tabular}{|l|c|c|r|r|}
  \hline
  \multicolumn{5}{c}{\textbf{Mixed Cell Model No Release Contaminant Transport}}\\
  \hline
  \textbf{Case}  &  \textbf{Component} &  \textbf{Degradation Rate} & \textbf{Expected 10 yrs} & \textbf{Actual 10 yrs}\\
  \textbf{ID}    & \textbf{[Type]} &  \textbf{$[yr^{-1}]$}  &  $[\%]$  & $[\%]$\\
  \hline
  MCI     &  WF    &  0   & 1\\
          &  WP    &  0.1 & 0\\
          &  BUFF  &  0.1 & 0\\
          &  FF    &  0.1 & 0\\
  \hline
  MCII    &  WF    &  0.1 & 0\\
          &  WP    &  0   & 1\\
          &  BUFF  &  0.1 & 0\\
          &  FF    &  0.1 & 0\\
  \hline
  MCIII   &  WF    &  0.1 & 0\\
          &  WP    &  0.1 & 0\\
          &  BUFF  &  0   & 1\\
          &  FF    &  0.1 & 0\\
  \hline
  MCIV    &  WF    &  0.1 & 0\\
          &  WP    &  0.1 & 0\\
          &  BUFF  &  0.1 & 0\\
          &  FF    &  0   & 1\\
  \hline
\end{tabular}
\caption{<+Caption text+>}
\label{tab:<+label+>}
\end{table}<++>


\subsubsection{Lumped Parameter Model}

\begin{table}
\centering
\begin{tabular}{|l|c|c|r|}
  \hline
  \multicolumn{4}{c}{\textbf{Lumped Parameter Model No Release Contaminant Transport}}\\
  \hline
  \textbf{Case}  &  \textbf{Component} &  \textbf{PFM Param} & \textbf{Expected 10 yrs} & \textbf{Actual 10 yrs}\\
  \textbf{ID}    & \textbf{[Type]} &  \textbf{$[yr^{-1}]$}  &  $[\%]$  & $[\%]$\\
  \hline
  DRI     &  WF    &  0   & 1\\
          &  WP    &  0.1 & 0\\
          &  BUFF  &  0.1 & 0\\
          &  FF    &  0.1 & 0\\
  \hline
  DRII    &  WF    &  0.1 & 0\\
          &  WP    &  0   & 1\\
          &  BUFF  &  0.1 & 0\\
          &  FF    &  0.1 & 0\\
  \hline
  DRIII   &  WF    &  0.1 & 0\\
          &  WP    &  0.1 & 0\\
          &  BUFF  &  0   & 1\\
          &  FF    &  0.1 & 0\\
  \hline
  DRIV    &  WF    &  0.1 & 0\\
          &  WP    &  0.1 & 0\\
          &  BUFF  &  0.1 & 0\\
          &  FF    &  0   & 1\\
  \hline
\end{tabular}
\caption{<+Caption text+>}
\label{tab:<+label+>}
\end{table}<++>


\subsubsection{One Dimensional Advecitive Dispersive Model}

\begin{table}
\centering
\begin{tabular}{|l|c|c|r|r|}
  \hline
  \multicolumn{5}{c}{\textbf{One Dimensional PPM Model No Release Contaminant Transport}}\\
  \hline
  \textbf{Case}  &  \textbf{Component} &  \textbf{Porosity} & \textbf{Expected 10 yrs} & \textbf{Actual 10 yrs}\\
  \textbf{ID}    & \textbf{[Type]} &  \textbf{$[yr^{-1}]$}  &  $[\%]$  & $[\%]$\\
  \hline
  DRI     &  WF    &  0   & 1\\
          &  WP    &  0.1 & 0\\
          &  BUFF  &  0.1 & 0\\
          &  FF    &  0.1 & 0\\
  \hline
  DRII    &  WF    &  0.1 & 0\\
          &  WP    &  0   & 1\\
          &  BUFF  &  0.1 & 0\\
          &  FF    &  0.1 & 0\\
  \hline
  DRIII   &  WF    &  0.1 & 0\\
          &  WP    &  0.1 & 0\\
          &  BUFF  &  0   & 1\\
          &  FF    &  0.1 & 0\\
  \hline
  DRIV    &  WF    &  0.1 & 0\\
          &  WP    &  0.1 & 0\\
          &  BUFF  &  0.1 & 0\\
          &  FF    &  0   & 1\\
  \hline
\end{tabular}
\caption{<+Caption text+>}
\label{tab:<+label+>}
\end{table}<++>


\subsection{Basic Transport}

\subsubsection{Degradation Rate Model}

\begin{table}
\centering
\begin{tabularx}{\textwidth}{|X|c|c|r|r|}
  \hline
  \multicolumn{5}{c}{\textbf{Degradation Rate Model Basic Contaminant Transport}}\\
  \hline
  \textbf{Case}  &  \textbf{Component} &  \textbf{Degradation Rate} & \textbf{Expected 10 yrs} & \textbf{Actual 10 yrs}\\
  \textbf{ID}    & \textbf{[Type]} &  \textbf{$[yr^{-1}]$}  &  $[\%]$  & $[\%]$\\
  \hline
  DRI     &  WF    &  0   & 1 & <++> \\ 
          &  WP    &  0.1 & 0 & <++> \\ 
          &  BUFF  &  0.1 & 0 & <++> \\ 
          &  FF    &  0.1 & 0 & <++> \\ 
  \hline
  DRII    &  WF    &  0.1 & 0 & <++> \\ 
          &  WP    &  0   & 1 & <++> \\ 
          &  BUFF  &  0.1 & 0 & <++> \\ 
          &  FF    &  0.1 & 0 & <++> \\ 
  \hline
  DRIII   &  WF    &  0.1 & 0 & <++> \\ 
          &  WP    &  0.1 & 0 & <++> \\ 
          &  BUFF  &  0   & 1 & <++> \\ 
          &  FF    &  0.1 & 0 & <++> \\ 
  \hline
  DRIV    &  WF    &  0.1 & 0 & <++> \\ 
          &  WP    &  0.1 & 0 & <++> \\ 
          &  BUFF  &  0.1 & 0 & <++> \\ 
          &  FF    &  0   & 1 & <++> \\ 
  \hline
\end{tabularx}
\caption[Degradation rate model clay basic transport problem results.]{Release was 
tested for various degradation rates in each component.}
\label{tab:dr_no_release}
\end{table}


\subsubsection{Mixed Cell Model}


\begin{table}
\centering
\footnotesize{
\begin{tabularx}{\textwidth}{|X|c|c|r|r|}
  \multicolumn{5}{c}{\textbf{Degradation Rate Model No Release Contaminant Transport}}\\
  \hline
  \textbf{Case}  &  \textbf{Component} &  \textbf{Degradation Rate} & \textbf{Expected 10 yrs} & \textbf{Actual 10 yrs}\\
  \textbf{ID}    & \textbf{[Type]} &  \textbf{$[yr^{-1}]$}  &  $[\%]$  & $[\%]$\\
  \hline
  DRI     &  WF    &  0   & 1\\
          &  WP    &  0.1 & 0\\
          &  BUFF  &  0.1 & 0\\
          &  FF    &  0.1 & 0\\
  \hline
  DRII    &  WF    &  0.1 & 0\\
          &  WP    &  0   & 1\\
          &  BUFF  &  0.1 & 0\\
          &  FF    &  0.1 & 0\\
  \hline
  DRIII   &  WF    &  0.1 & 0\\
          &  WP    &  0.1 & 0\\
          &  BUFF  &  0   & 1\\
          &  FF    &  0.1 & 0\\
  \hline
  DRIV    &  WF    &  0.1 & 0\\
          &  WP    &  0.1 & 0\\
          &  BUFF  &  0.1 & 0\\
          &  FF    &  0   & 1\\
  \hline
\end{tabularx}
\caption{<+Caption text+>}
\label{tab:<+label+>}
}
\end{table}<++>


\subsubsection{Lumped Parameter Model}

\begin{table}
\centering
\begin{tabularx}{\textwidth}{|X|c|c|r|r|}
  \hline
  \multicolumn{5}{c}{\textbf{Degradation Rate Model No Release Contaminant Transport}}\\
  \hline
  \textbf{Case}  &  \textbf{Component} &  \textbf{Degradation Rate} & \textbf{Expected 10 yrs} & \textbf{Actual 10 yrs}\\
  \textbf{ID}    & \textbf{[Type]} &  \textbf{$[yr^{-1}]$}  &  $[\%]$  & $[\%]$\\
  \hline
  DRI     &  WF    &  0   & 1\\
          &  WP    &  0.1 & 0\\
          &  BUFF  &  0.1 & 0\\
          &  FF    &  0.1 & 0\\
  \hline
  DRII    &  WF    &  0.1 & 0\\
          &  WP    &  0   & 1\\
          &  BUFF  &  0.1 & 0\\
          &  FF    &  0.1 & 0\\
  \hline
  DRIII   &  WF    &  0.1 & 0\\
          &  WP    &  0.1 & 0\\
          &  BUFF  &  0   & 1\\
          &  FF    &  0.1 & 0\\
  \hline
  DRIV    &  WF    &  0.1 & 0\\
          &  WP    &  0.1 & 0\\
          &  BUFF  &  0.1 & 0\\
          &  FF    &  0   & 1\\
  \hline
\end{tabularx}
\caption{<+Caption text+>}
\label{tab:<+label+>}
\end{table}<++>


\subsubsection{One Dimensional Advecitive Dispersive Model}


\begin{table}
\centering
\footnotesize{
\begin{tabularx}{\textwidth}{|X|c|c|r|r|}
  \multicolumn{5}{c}{\textbf{One Dimensional PPM Rate Model No Release Contaminant Transport}}\\
  \hline
  \textbf{Case}  &  \textbf{Component} &  \textbf{Porosity} & \textbf{Expected 10 yrs} & \textbf{Actual 10 yrs}\\
  \textbf{ID}    & \textbf{[Type]} &  \textbf{$[\%]$}  &  $[\%]$  & $[\%]$\\
  \hline
  1DI     &  WF    &  0   & 1 & <++> \\ 
          &  WP    &  0.1 & 0 & <++> \\ 
          &  BUFF  &  0.1 & 0 & <++> \\ 
          &  FF    &  0.1 & 0 & <++> \\ 
  \hline
  1DII    &  WF    &  0.1 & 0 & <++> \\ 
          &  WP    &  0   & 1 & <++> \\ 
          &  BUFF  &  0.1 & 0 & <++> \\ 
          &  FF    &  0.1 & 0 & <++> \\ 
  \hline
  1DIII   &  WF    &  0.1 & 0 & <++> \\ 
          &  WP    &  0.1 & 0 & <++> \\ 
          &  BUFF  &  0   & 1 & <++> \\ 
          &  FF    &  0.1 & 0 & <++> \\ 
  \hline
  1DIV    &  WF    &  0.1 & 0 & <++> \\ 
          &  WP    &  0.1 & 0 & <++> \\ 
          &  BUFF  &  0.1 & 0 & <++> \\ 
          &  FF    &  0   & 1 & <++> \\ 
  \hline
\end{tabularx}
\caption[One-Dimensional PPM Model No Release Cases]{Each cases that was run to 
evaluate non release performance in the One-Dimensional PPM model parameterized 
a barrier component to release no contaminants.}
\label{tab:1d_no_release}
}
\end{table}

