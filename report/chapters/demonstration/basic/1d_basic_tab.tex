
\begin{table}
\centering
\footnotesize{
\begin{tabularx}{\textwidth}{|X|c|c|r|r|}
  \multicolumn{5}{c}{\textbf{One Dimensional PPM Rate Model No Release Contaminant Transport}}\\
  \hline
  \textbf{Case}  &  \textbf{Component} &  \textbf{Porosity} & \textbf{Expected 10 yrs} & \textbf{Actual 10 yrs}\\
  \textbf{ID}    & \textbf{[Type]} &  \textbf{$[\%]$}  &  $[\%]$  & $[\%]$\\
  \hline
  1DI     &  WF    &  0   & 1 & <++> \\ 
          &  WP    &  0.1 & 0 & <++> \\ 
          &  BUFF  &  0.1 & 0 & <++> \\ 
          &  FF    &  0.1 & 0 & <++> \\ 
  \hline
  1DII    &  WF    &  0.1 & 0 & <++> \\ 
          &  WP    &  0   & 1 & <++> \\ 
          &  BUFF  &  0.1 & 0 & <++> \\ 
          &  FF    &  0.1 & 0 & <++> \\ 
  \hline
  1DIII   &  WF    &  0.1 & 0 & <++> \\ 
          &  WP    &  0.1 & 0 & <++> \\ 
          &  BUFF  &  0   & 1 & <++> \\ 
          &  FF    &  0.1 & 0 & <++> \\ 
  \hline
  1DIV    &  WF    &  0.1 & 0 & <++> \\ 
          &  WP    &  0.1 & 0 & <++> \\ 
          &  BUFF  &  0.1 & 0 & <++> \\ 
          &  FF    &  0   & 1 & <++> \\ 
  \hline
\end{tabularx}
\caption[One-Dimensional PPM Model No Release Cases]{Each cases that was run to 
evaluate non release performance in the One-Dimensional PPM model parameterized 
a barrier component to release no contaminants.}
\label{tab:1d_no_release}
}
\end{table}
