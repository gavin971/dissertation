\section{Nuclide Model Benchmarks}\label{sec:nuclide_benchmarks}

As described in Section \ref{sec:nuclide_models}, hydrologic contaminant 
transport in Cyder is implemented with four interchangeable  methods in a 
modular software design. These modeling options alternately optimize speed and 
fidelity in representations of barrier components within the repository concept 
(i.e. waste form, waste package, buffer, near field geology, and far field 
geology)\cite{huff_hydrologic_2013}.  Simplistic models include a congruent 
release component degradation model and a mixed cell control volume model. For 
systems in which the flow can be assumed constant, a medium fidelity lumped 
parameter dispersion model is implemented. Also implemented is a Leij et al. 
solution to the advection dispersion equation for Cauchy boundary condition 
\cite{leij_analytical_1991, van_genuchten_analytical_1982}.  

Analyses in Table \ref{tab:nuclide_bench_tab} were conducted to compare the 
performance of these radionuclide transport models with more detailed results from the 
Clay \gls{GDSE}. 


\begin{table}[ht!]
\centering
\footnotesize{
  \begin{tabularx}{\textwidth}{|X|l|l|r|}
\multicolumn{4}{c}{\textbf{Radionuclide Transport Benchmark Cases}}\\
\hline
\textbf{Parameter} & \textbf{Symbol} & \textbf{Units} & \textbf{Value Range} \\
\hline
Hydraulic & & & \\
Reference & & & \\
Diffusivity& $\alpha_{h,ref}$& $[m^2s]$ & $10^{-15} - 10^{-8}$ \\
\hline
Hydraulic & & & \\
Conductivity& $K_{h}$& $[m \cdot s^{-1}]$ & $10^{-13} - 10^{-3}$ \\
\hline
Advective  & & & \\
Water & & & \\
Velocity & $v_{adv}$ & $[m\cdot s^{-1}]$ & $2\times10^{-16}-2\times10^{-12}$ \\
\hline
Sorption \& & & & Reducing - \\
Behavior & $K_{d,i}$& $[m^3\cdot kg^{-1}]$ & Oxidizing \\
\hline
Solubility &  & & Reducing -\\
Limitation & $C_{sol,i}$ & $[kg\cdot m^{-3}3]$& Oxidizing \\
\hline
WF& & & \\
Degradation& & & \\
Rate& $f_{wf}$ & [month$^{-1}$]& $0.0001-0.9$ \\
\hline
\end{tabularx}
\caption{The sensitivity analyses conducted in this work covered a range of 
thermal and hydrologic parameters in the context of canonical fuel cycle choices.}
}
\label{tab:nuclide_bench_tab}
\end{table}




\subsection{Case I : Diffusion Coefficient and Inventory Sensitivity}
\subsection{Case II : Vertical Advective Velocity and Diffusion Coefficient Sensitivity}
\subsection{Case III : Solubility Sensitivity}
\subsection{Case IV : Sorption Sensitivity}
\subsection{Case V : Waste Form Degradation Rate and Inventory Sensitivity}
\subsection{Case VI : Waste Package Failure Time and Diffusion Coefficient Sensitivity}
