\section{Solubility Coefficient Sensitivities}

The dissolution behavior of a solute in an aqueous solutions is called its 
solubility. This behavior is limited by the solute's solubility limit, described  
by an equilibrium constant that depends upon temperature, water chemistry, and 
the properties of the element. The solubility constant for ordinary solutes, 
$K_s$ gives units of concentration, $[kg/m^3]$, and can be determined 
algebraically by the law of mass action which gives the partitioning at 
equilibrium between reactants and products.  For a reaction
\begin{align}
  cC + dD &= yY + zZ,
  \intertext{where}
  c,d,y,z  &= \mbox{ amount of respective constituent }[mol]\nonumber\\
  C,D  &= \mbox{ reactants }[-]\nonumber\\
  Y,Z  &= \mbox{ products }[-]\nonumber,
  \intertext{the law of mass action gives}
  K &= \frac{(Y)^y(Z)^z}{(C)^c(D)^d}
  \intertext{where}
  (X)  &= \mbox{ the equilibrium molal concentration of X }[mol/m^3]\nonumber\\
  K  &= \mbox{ the equilibrium constant }[-].\nonumber
  \label{massaction}
\end{align}
The equillibrium constant for many reactions are known, and can be found in 
chemical tables. Thereafter, the solubility constraints of a solution at 
equilibrium can be found algebraically.  In cases of salts that  dissociate in 
aqueous solutions, this equilibrium constant is called the salt's solubility 
product $K_{sp}$.

This equillibrium model, however, is only appropriate for dilute situations, and 
nondilute solutions at  partial equilibrium must be treated with an activity 
model by substituting the activities of the constituents  for their molal 
concentrations,
\begin{align}
  [X] &= \gamma_x(X)
  \intertext{where}
  [X]  &= \mbox{ activity of X }[-]\nonumber\\
  \gamma_x  &= \mbox{ activity coefficient of X}[-]\nonumber\\
  (X)  &= \mbox{ molal concentration of X}[mol/m^3]\nonumber
  \intertext{such that}
  IAP &= \mbox{ Ion Activity Product }[-].\nonumber\\
      &= \frac{[Y]^y[Z]^z}{[C]^c[D]^d}\\
  \label{IAP}
\end{align}
The ratio between the IAP and the equillibrium constant $(IAP/K)$ quantifiesn
the departure from equilibrium of a solution.  This information is useful during 
the transient stage in which a solute is first introduced to a solution. When 
$IAP/K<1$, the solution is undersaturated with respect to the products. When, 
conversely, $IAP/K>1$, the solution is oversaturated and precipitation of solids 
in the volume will occur. 

Two models of mass balance applicable to a control volume incorporating 
solubility limitation include the Ahn and Hedin models.

Forty values of solubility coefficient multiplier were used. 
These values were $1E^{-9}$ through $5E^{-10}$. This multiplier
adjusted the fiducial values of solubility for each element, so 
the relative solubility between elements was preserved.

