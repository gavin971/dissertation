\section{Thermal Conductivity Sensitivity Study}\label{sec:conductivity}
The thermal conductivity, $K_{th}$ of geologic repository host media impacts 
the speed of transport, and therefore the time evolution of thermal energy 
deposition, in the medium. 

In the creation of the \gls{STC} database, the thermal conductivity was varied 
across a broad domain for each isotope, $i$, package spacing, $s$, limiting 
radius $r_{calc}$, and thermal diffusivity $\alpha_{th}$, considered.  By 
varying the thermal conductivity of the repository model from 0.1 to 4.5
$[W\cdot m^{-1} \cdot K^{-1}]$, this sensitivity analysis succeeds in capturing the domain of 
thermal conductivities witnessed in high thermal conductivity salt deposits as 
well as low thermal conductivity clays.

\begin{figure}[htbp!]
\begin{center}
%\includegraphics[width=0.7\textwidth]{./chapters/thermal_sensitivity/conductivity/Cm242Kth_alpha_low.eps}
\end{center}
\caption[$K_{th}$ Sensitivity for Low $\alpha_{th}$]{Increased thermal conductivity decreases thermal energy deposition 
(here represented by \gls{STC}) in the near field (here $r_{calc} = 0.5m$).}
\label{fig:Cm242Kth_alpha_low}
\end{figure}


Figure \ref{fig:Cm242Kth_alpha_low} shows the trend, visible for all isotopes, 
that increased thermal conductivity of a medium decreases thermal energy 
deposition in the near field. This indicates, then that thermal conductivity is 
an important parameter for repository geolgic medium selection. The effect is 
accentuated by high thermal diffusivities, as seen in 
Figure \ref{fig:Cm242Kth_alpha_high}

\begin{figure}[htbp!]
\begin{center}
%\includegraphics[width=0.7\textwidth]{./chapters/thermal_sensitivity/conductivity/Cm242Kth_alpha_high.eps}
\end{center}
\caption[$K_{th}$ Sensitivity for High $\alpha_{th}$]{Increased thermal conductivity decreases thermal energy deposition 
(here represented by \gls{STC}) in the near field (here $r_{calc} = 0.5m$).}
\label{fig:Cm242Kth_alpha_high}
\end{figure}


