\chapter{Introduction}\label{ch:introduction}

% The scope of the work includes software and analysis.

The scope of this work includes implementation of a software library of medium 
fidelity models to comprehensively represent various long-term disposal system 
concepts for nuclear material. This software library will be integrated with a 
computational fuel cycle systems analysis platform in order to inform repository 
performance metrics with respect to candidate fuel cycle options. By abstraction 
of more detailed models, this work will capture the dominant physics of 
radionuclide and heat transport phenomena affecting repository performance in 
various geologic media and as a function of arbitrary spent fuel composition. 

\section{Motivation} 

% There needs to be a system level approach to repository analysis which is both 
% modular and efficient.

The development of sustainable nuclear fuel cycles is a key challenge as the use 
of nuclear power expands internationally. While fuel cycle performance may be 
measured with respect to a variety of metrics, waste management metrics are of 
particular importance to the goal of sustainability. Since disposal options are 
heavily influenced by upstream fuel cycle decisions, a relevant analysis of 
potential geological disposal and engineered barrier solutions therefore 
requires a system level approach that is both modular and efficient. 

% The approach must also be modular to support combinatiorically complex 
% decision space 

As the merits of numerous combinatoric fuel cycle possibilities are 
investigated, a top-level simulation tool capable of modular substitution of 
various fuel cycle facility, repository, and engineered barrier components is 
needed. The modular waste package and repository models resulting from this work 
will assist in informing current technology choices, identifying important 
parameters contributing to key waste disposal metrics, and highlighting the most 
promising waste disposal combinations with respect to metrics chosen by
the user. Specifically, such models will support efforts underway in focusing 
domestic research and development priorities as well as computational tools 
under development which quantify these metrics and demonstrate the merits of 
different fuel cycle alternatives. 

% The modeling approach must be efficient in order for many scenarios to be 
% compared quickly

System level fuel cycle simulation tools must facilitate efficient sensitivity 
and uncertainty analysese as well as simulation of a wide range of fuel cycle 
alternatives.  Efficiency is achieved with models at a level of detail which 
successfully captures significant aspects of the underlying physics while 
acheiving a calculation speed in accordance with use cases requiring repeated 
simulations. Often termed abstraction, the process of simplifying while 
maintaining the salient features of the underlying physics is the method by 
which used fuel disposal system models are developed in this work. 

% The fuel cycle parameters that may be varied are numerous and coupled to the 
% back end.

Independent fuel cycle parameters of particular interest in fuel cycle systems 
analysis have been those related to the front end of the fuel cycle. Deployment 
decisions concerning reactor types, fast to thermal reactor ratios, and burnup 
rates can all be independently varied in fuel cycle simulation codes in such a 
way as to inform domestic policy decisions going forward. Some of these 
parameters are coupled, however, to aspects of the back end of the fuel cycle.  
For example, the appropriate fast reactor ratio is significantly altered by the 
chosen method and magnitude of domestic spent fuel reprocessing (or not).

% Back end parameters are interesting.

However, independent variables representing decisions concerning the back end
of the fuel cycle are of increasing interest as the United States further
investigates repository alternatives to Yucca Mountain.  Parameters such as the
repository geology, tunnel design, and appropriate loading strategies and
schedule are all independent variables up for debate. All of these
parameters are coupled with decisions about the fuel cycle. 

% So we need a full systems analysis package, incorporating a repository. 

Thus, while independent parameters can be chosen and varied
within a fuel cycle simulation, some parameters are coupled in such a way as to
require full synthesis with a systems analysis code that appropriately
determines the isotpic mass flows into the repository, their appropriate
conditioning, densities, and other physical properties.  

% %   * repository * geology * reducing/oxidizing * salt * granite * clay * deep 
% boreholes (srsly) * design * tunnel widths?  * tunnel lengths?  * distance 
% between tunnels?  * depths * distance from water table * loading strategies * 
% cooling pad timing * interim storage timing (i.e. between fuel cycle stages) * 
% utilizing short lived ILW facilities * tunnel ventilation * packing 
% optimization * partitioning/separations * separation efficiency * of MAs * of 
% Pu * of Iodine * etc.  * separation strategy * isotope decisions * MAs?  * Pu?  
% * etc.?  * chemistry decisions * aqueous?  * pyro?  * does it matter?  * 
% transmutation * Advanced Reactors * deployment timing * Pu or MA content of 
% fuels * types of reactors * high burnup * MOX * various Fast Reactors * 
% various conversion/breeding ratios * Integral Fast Reactor *  * wave reactor * 
% candle reactor * Thermal Reactors * initial UO_2 enrichment * maximum burnup * 
% Pu / MA content of fuels

%         * Thorium ?  * Reactor Type * PWR * BWR * VVER * Accelerator Driven 
%         Systems * target MA content

%         * burnup efficiency * power share of various installed reactor types * 
%         conditioning * waste packaging * forms * liquids * glass vitrification 
%         * cement * bitumen * solids * steel * combined material canisters * 
%         dual purpose transport and storage canisters * overpack 
%         characteristics * etc?  * densities * impacted by density-dependent 
%         release rates * volumes * incineration/compaction (solid waste) * 
%         evaporation/filtration/ion exchange (liquid waste) * engineered 
%         barriers * likelihood of human intrusion ?  %

\subsection{Future Fuel Cycle Options}

% DOE is interested in many fuel cycle options. 

Domestically, nuclear power expansion is motivated by the research, development, 
and demonstration roadmap being pursued by the United States Department of 
Energy Office of Nuclear Energy (DOE-NE) which seeks to ensure that nuclear 
energy remain a viable domestic energy option \cite{doe_nuclear_2010}. 

% DOE is considering 3 main options, each of which pose different waste 
% management challenges . . .

As the DOE-NE seeks to develop technologies and strategies to support a 
sustainable future for nuclear energy, various fuel cycle strategies and 
corresponding disposal system options are being considered.  Specifically, the 
domestic fuel cycle option space under current consideration is described in 
terms of three distinct fuel cycle categories with the monikers Once Through, 
Full Recycle, and Modified Open. Each category presents unique disposal system 
siting and design challenges. Systems analyses for evaluating these options must 
be undertaken in order to inform a national decision to deploy a comprehensive 
fuel cycle system by 2050 \cite{doe_nuclear_2010}. 

% Once through presents capacity issues . . . 

The Once-Through Cycle category includes fuel cycles similar to the continuation 
of the business as usual case in the United States.
Such fuel cycles neglect reprocessing and present challenges associated with 
high volumes of minimally treated spent fuel streams.  In a business as usual 
scenario, conventional power reactors comprise the majority of nuclear energy 
production and fuel takes a single pass through a reactor before it is 
classified as waste and disposed of. In the open cycle, no reprocessing is 
pursued, but research and development of advanced fuels seek to reduce waste 
volumes. Calculations from the Electric Power Research Institute indicate that 
without an increase in the stautory footprint limit of the repository, such a 
cycle will generate a volume of spent fuel that will necessitate the siting of 
two or more federal  geological repositories to accomodate spent fuel 
\cite{kessler_room_2006}.  %% KDHFLAG WHEN? HOW LONG?

% Full Recycle presents the issue of variable waste streams. . .

A Full Recycle option, on the other hand, requires the research, development, 
and deployment of partititioning, transmutation, and advanced reactor technology 
for the reprocessing of used nuclear fuel.  In this scheme, conventional 
once-through reactors will be phased out in favor of fast reactor and so called 
Generation IV reactor technologies which  demonstrate transmutation capacity and 
greater fuel efficiency. All fuel in the Full Recycle strategy will be 
reprocessed. It may be reprocessed using  an accelerator driven system or by 
cycling through an advanced fast reactor. Such fuel may undergo partitioning, 
the losses from which will require waste treatment ultimate disposal in a 
repository. Thus, a repository under the Full Recycle scenario must support 
highly variable waste stream composition during transition periods, as well as 
myriad waste forms and packaging associated with isolation of differing waste 
streams.

% Modified Open presents both problems. . . 

Finally, the Modified Open Cycle category of options includes a variety of fuel 
cycle options that fall between once through and fully closed. Advanced fuel 
cycles such as deep burn and small modular reactors will be considered within 
the Modified Open set of fuel cycle options as will partial recycle options. 
Partitioning and reprocessing strategies, however, will be limited to simplified 
chemical separations and volitilization under this scheme. This scheme presents
a dual challenge in which spent fuel volumes and composition will both vary 
dramatically among various possibilities within this scheme 
\cite{doe_nuclear_2010} .

% Various waste streams require various WFs and WPs 

Clearly, the myriad waste streams resulting from potential fuel cycles present 
an array of corresponding waste disposition, packaging, and engineered barrier 
system options. A comprehensive analysis of the disposal system, dominant 
physics models must therefore be developed for these subcomponents.  Differing 
spent fuel composition, partitioning, transmutation, and chemical processing 
decisions upstream in the fuel cycle demand differing performance and loading 
requirements of waste forms and packaging. The capability to model thermal and 
radionuclide transport phenomena through, for example, vitrified glass as well 
as ceramic waste forms with with various loadings for arbitrary isotopic 
compositions is therefore required.  

\subsection{Future Waste Disposal System Options}

% DOE is thinking about various geologies

In addition to reconsideration of the domestic fuel cycle policy, the uncertain 
future of the \gls{YMR} has driven the expansion of the option space of 
potential repository geologies to include, at the very least, granite, 
clay/shale, salt, and deep borehole concepts \cite{nutt_used_2010}. 

% Various waste forms, packages, etc. are being considered.

In accordance with various fuel cycle options, corresponding waste form, waste 
package, and other engineered barrier systems are being considered. 
Specifically, current considerations include ceramic (e.g.  Uranium Oxide), 
glass (e.g.  borosilicate glasses), and metallic (e.g.  hydride fuels) waste 
forms. Waste packages may be copper, steel, or other alloys. Similarly, buffer 
and backfill materials vary from the crushed salt recommended for a salt 
repository to bentonite or concrete in other lithologies. % KDHFLAG this 
paragraph needs explication

% Various geologies, WFs, WPs, EBSs have various physics

The physical, hydrogeologic, and geochemical mechanisms which dictate 
radionuclide and heat transport vary between the geological and engineered 
containment systems in the domestic option space.  Therefore, in support of the 
system level simulation effort, models must be developed which capture the 
salient physics of these geological options and quantify associated disposal 
metrics and benefits.  Furthermore, modular linkage between subcomponent process 
modules and the repository environmental model must acheive a cohesively 
integrated disposal system model. 


\subsubsection{Thermal Modeling Needs}
% repository loading limits
% optimization of layout
% necessary decay cooling time before emplacement.
The decay heat from nuclear material presents a significant heat source within a 
repository. In order to arrive at loading strategies which comply with   thermal 
limits in the engineered barrier system and the geological medium , a thermal 
modeling capability must be included in the repository model. Such a model is 
also necessary to inform material and hydrogeologic phenomena which effect 
radionuclide transport and are thermally coupled. 

Since the partitioning and transmutation of heat generating radionuclides within  
some fuel cycles will alter the heat evolution of the repository 
\cite{swift_applying_2010}. To distinguish  between the repository heat 
evolution associated with various fuel cycles, a systems analysis model must 
capture the decay heat behavior of dominant heat contributors.  Plutonium, 
Americium, and their decay daughters dominate decay heat contribution within 
used nuclear fuels. Other contributing radionuclides include Cesium, Strontium, 
and Curium \cite{piet_which_2007}. 

Thermal limits within a used nuclear fuel disposal system are waste form, 
package, and geology dependent. Heat generation from the waste form and 
transport through the engineered barrier system and host environment constrains 
waste form loadingss and package loadings as well as placing requirements on the 
size, design, and loading strategy in a potential geological repository.

Thermal limits of various waste forms have their technical basis in the 
temperature dependence of isolation integrity of the waste form.  The waste form 
alteration, degradation, and dissolution behavior as a function of decay heat 
constrains loading density within the waste form. 
 
Thermal limits of various engineered barrier systems similarly have a technical 
basis in the temperature dependent dissolution rate of the materials from whence 
they are constructed.  

Thermal limits of the geologic environment on can be based on the mechanical 
integrity of the rock.  minerologic, hydrologic and geochemical phenomena. The 
isolating characteristics of a geological environment are most sensitive to t  
and geochemical effects of thermal loading. Thus, heat load constraints are 
typically chosen to control hydrologic and geochemical response to thermal 
loading. In the United States, current regulations stipulate thermal limits in 
order to passively steward the repository's hydrologic and geochemical integrity 
against radionuclide  release for the upcoming 10,000 years.

The two heat load constraints which primarily determined the heat-based SNF 
capacity limit in the Yucca Mountain Repository design, for example, are 
specific to unsaturated tuff. These are given here as an example of the type of 
regulatory constraints which this model will seek to capture for various 
geologies. 

The first Yucca Mountain heat load constraint intended to promote constant 
drainage, thereby preventing repository flooding and subsequent contaminated 
water flow through the repository. It requires that the minimum temperature in 
the tuff between drifts be no  more than the boiling temperature of water which, 
at the altitude in question, is $96^{\circ}C$. For a repository with homogenous 
waste composition in parallel drifts, this constraint limits the temperature 
exactly halfway between adjacent drifts, where the temperature is at a minimum.

The second constraint intended to prevent high rock temperatures that could 
induce fractures and alteration of the crystalline rock. It stated that no part 
of the rock reach a temperature above $200^{\circ}C$, and was effectively a 
limit on the temperature at the drift wall, where the rock temperature is a 
maximum.  

Analgous constraints for a broader set of possible geological environments will 
depend on heat transport properties and geochemical behaviors of the rock matrix 
as well as its hydrogeologic state.  Such constraints will affect the  
repository drift spacing, waste package spacing, and repository footprint among 
other parameters. 

%Clay repositories should have a ~70 degrees C limit, because temps higher than 
%100 degrees can cause irreversible minerological damage.
%From ANDRA:
%``In order to remain in an operational range in which phenomena are known and, 
%thus, reduce any damage to the argillite, the objective is to restrict 
%argillite temperature to these values. Basically, it means that the thermal 
%dimensioning of the cells and the architecture of the C waste repository zone 
%aim to restrict the temperature to 90°C at the interface “disposal cell – 
%argillite” and to ensure that the temperature will be below 70°C, in the 
%geological medium on the cell boundary, before a thousand years, which provides 
%a safety margin with respect to thermal effects.''

In addition to development of a concept of heat transport within the repository 
in order to meet heat load limitations, it is also necessary to model 
temperature gradients in the repository in  order to support modeling of 
thermally dependent hydrologic and material phenomena.  As mentioned above, 
waste form corrosion processes, waste form
dissolution rates, diffusion coefficients, and the mechanical integrity of 
engineered barriers and geologic environment are coupled with temperature 
behavior.  \cite{andra_argile:_2005}. %andra, clay, evaluation, page 195)

%Only a coarse time resolution will likely be necessary to capture that coupling 
%however, since time evolution of repository heat is
%such that thermal coupling can typically be neglected for long time scales

\subsubsection{Radiotoxicity and Source Term Modeling Needs}

Domestically, the Nuclear Regulatory Commission has defined a limit on  human 
exposure due to the repository. This regulation places important limitations on 
capacity, design, and loading techniques for repository concepts under 
consideration. Repository concepts developed in this work must therefore 
quantify radionuclide transport through the geological environment in order to 
calculate repository capacity and other benefit metrics. 

The exposure limit set by the NRC is based on a `reasonably maximally exposed 
individual.' That is to say, the limiting case is a person who lives, grows 
food, drinks water and breathes air 18 km downstream from the repository. The 
Yucca Mountain Repository \gls{EPA} regulations limit total dose from the 
repository to 15 mrem/yr, and limit dose from drinking water to 4 mrem/yr.  
Predictions of that dose rate depend on an enormous variety of factors, most 
important of which is the primary pathway for release. In the \gls{YMR} primary 
pathway of radionuclides from an accidental release will be from cracking aged 
canisters. Subsequently, transport of the radionuclides to the water table 
requires that the radionuclides come in contact with water and travel through 
the rock the water table. This results in contamination of drinking water 
downstream.  

Source term is a measure of the quantity of a radionuclide released into the 
environment whereas radiotoxicity is a measure of the hazardous effect of that 
particular radionuclide upon human ingestion or inhalation.  In particular, 
radiotoxicity is measured in terms of the volume of water dilution required to 
make it safe to ingest. Studies of source term and radiotoxicity therefore make 
probabilistic assessments of radionuclide release, transport, and human 
exposure.  

Importantly, due to the long time scale and intrinsic uncertainties required in 
such probabilistic assessments it is in general not advisable to base any 
maximum repository capacity estimates on source term.
In order to give informative values for the risk associated with transport of 
particular radionuclides, for example, it is necessary to make highly uncertain  
predictions concerning waste form degredation, water flow, and other parameters 
during the long repository evolution time scale.  However, source term remains a 
useful metric for the comparison of alternative separations and fuel cycle 
scenarios.

%The probabilistic nature of these assessments mean a direct dependence of 
%source term on repository capacity can be difficult to arrive at.  

A generalized metric of probablistic risk is fairly difficult to arrive at. For 
example, the \gls{PEI} metric from Berkeley (ref. 
\cite{bouvier_comparison_2007}) is a multifaceted function of spent fuel 
composition, waste conditioning, vitrification method, and radionuclide 
transport through the repository walls and rock.  Also, it makes the assumption 
that the waste canisters have been breached at $t=0$. Furthermore, reported in 
$m^3$, PEI is a measure of radiotoxicity in the environment in the event of 
total breach. While informative, this model on its own does not completely 
determine a source-term limited maximum repository capacity.  Additional waste 
package failure and exposed individual radiotoxicity constraints must be 
incorporated into it.


\subsection{Domestic Research and Development Program}

The DOE-NE Fuel Cycle Technology (FCT) program has three groups of relevance to 
this effort.  These are the \gls{UFD}, the \gls{SWF}, and \gls{SA} campaigns.  
The \gls{UFD} campaign is conducting the \gls{RDD} related to the storage, 
transportation, and disposal of radioactive wastes generated under both the 
current and potential advanced fuel cycles.  The SWF campaign is conducting 
\gls{RDD} on potential waste forms that could be used to effectively isolate the 
wastes that would be generated in advanced fuel cycles.  The \gls{SWF} and
\gls{UFD} campaigns are developing the fundamental tools and information base 
regarding the performance of waste forms and geologic disposal systems.  The 
\gls{SA} campaign is developing the overall fuel cycle simulation tools and 
interfaces with the other FCT campaigns, including \gls{UFD}.  

This effort will interface with those campaigns to develop the higher level
dominant physics representations for use in fuel cycle system analysis tools.
Specifically, this work will leverage upon conceptual framework development and
primary data collection underway within the Used Fuel Disposition Campaign as
well as work by Radel, Wilson, Bauer et. al. to model repository behavior as a
function of the contents of the waste.  It will then incorporate dominant
physics process models into the \Cyclus computational fuel cycle analysis




\section{Methodology} 

% Scope

In this work, concise dominant physics models suitable for system level fuel 
cycle codes will be developed from comparison of analytical models with more 
detailed repository modeling efforts. The ultimate objective of this effort is 
to develop a software library capable of assessing a wide range of combinations 
of fuel cycle alternatives, potential waste forms, repository design concepts, 
and geological media. 

% Categorization

Categorization and characterization of physical mechanisms by which radionuclide 
and thermal transport take place within the materials and media under 
consideration will first be undertaken. In this way, the domain of applicability 
for which subprocesses may be generalized will be assessed. In so doing, a 
preliminary set of combinations of fuel cycles, waste forms, repository designs, 
and geologies can be chosen which covers a foundational subspace of the 
parametric domain. 

% Abstraction = Analytic Solutions + Regression Approximation

In general, such concise models are a combination of two components: 
semi-analytic mathematical models that represent a simplified description of the 
most important physical phenomena, and semi-empirical models that reproduce the 
results of detailed models.  By combining the complexity of the analytic models 
and regression against numerical experiments, variations can be limited between 
two models for the same system.  Different approaches will be compared in this 
work, with final modeling choices balancing the accuracy and efficiency of the 
possible implementations.  

% For example, Geology.

Specifically, geological models will focus on the hydrogeology and thermal 
physics which dominate radionuclide transport and heat response in candidate 
geologies as a function of radionuclide release and heat generation over long 
time scales from waste packages.  Dominant transport mechanism (advection or 
diffusion) and disposal site water chemistry (redox state) will provide primary 
differentiation between the different geologic media under consideration. In 
addition, the concise models will be capable of roughly adjusting release 
pathways according to the characteristics of the natural system (both the host 
geology and the site in general) and the engineered system (such as package 
loading arrangements, tunnel spacing, and engineered barriers).

% Abstraction in repository environment

The abstraction process in the development of a geological environment model 
will employ the comparison of semi-analytic thermal and hydrogeologic models and 
analytic regression of rich code results from more detailed models as well as 
existing empirical geologic data. Such results and data will be derived 
primarily from the \gls{UFD} campaign \glspl{GDSM} and  data as well as European 
efforts such as the RED-IMPACT asessment and \gls{ANDRA} Dossier efforts 
\cite{von_lensa_red-impact_2008, andra_argile:_2005, clayton_generic_2011} . 

% Example, heat from WPs along repo drifts.

For example, analytic models and semi-empirical models are available (i.e. 
specific temperature integrals (ref. \cite{li_methodology_2006}) and specific 
temperature change index (ref. \cite{radel_effect_2007})) which approximate 
thermal response from heat generation in waste packages as linear along 
repository drifts. These analyses arrive at a thermal evolution over time at any 
location in the rock by superpositioning and integration. Subsequently, their 
results can be converted easily into conventional line loading and areal power 
density metrics.  Such analytic models will first be assessed to determine 
likely parameters upon which thermal response will rely (e.g. tunnel spacing, 
radionuclide inventories, etc.).

% Regression.

Thereafter, a regression analysis concerning those parameters will be undertaken 
with available detailed models (e.g. 2D and 3D finite differencing, finite 
element, and thermal performance assessment codes) to further characterize the 
parametric dependence of thermal loading in a specific geology.  

% Incorporation of empirical data

Finally, the thermal behavior of a repository model so developed will depend on 
empirical data (e.g.  heat transfer coefficients, hydraulic conductivity). 
Determination of representative values to make available within the dominant 
physics model will rely on existing empirical data concerning the specific 
geologic environment being modeled (i.e. salt, clay/shale, and granite). 

% Abstraction for WFs, WPs, EBS, etc. will be the same

A similar process will be followed for radionuclide transport models.  The 
abstraction process in the development of waste form, package, and engineered 
barrier system models will be analgous to the abstraction process of repository 
environment models. Concise models will result from employing the comparison of 
semi-analytic models of those systems with regression analysis of rich code in 
combination with existing empirical material data.

% Example, radionuclide release

For instance, in the case of radionuclide release from waste packages, analytic 
models of radionuclide release (e.g.  congruent or solubility limited) will 
first be assessed to determine likely parameters upon which radionuclide release 
will rely (e.g.  radionuclide concentration, water flow rates, etc.) 
\cite{kawasaki_congruent_2004}.  Again, regression analysis concerning those 
parameters will be undertaken with available detailed models to further 
characterize the parametric dependence of radionuclide release from specific 
waste packages.  Finally, the radionuclide release model so developed will 
depend on empirical data (e.g. the waste form dissolution rate).  Determination 
of representative values to make available within the dominant physics model 
will rely on existing empirical data concerning the specific waste form  
materials being modeled (i.e. long time scale extrapolations of known glass 
degradation rates).  

% Coupling is confusing

Coupling effects between components will have to be considered carefully.  In 
particular, given the important role of temperature in the system, thermal 
coupling between the models for the engineered system and the geologic system 
may be important. Thermal dependence of radionuclide release and transport as 
well as package degradation will necessarily be analyzed to determine the 
magnitude of coupling effects in the system.


\section{Outline}

% Summarize document

The following chapter will present a literature review which organizes and 
reports upon previous relevant work. It will focus upon current analytical and 
computational modeling of radionuclide and heat transport through various waste 
forms, engineered barrier systems, and geologies of interest. It will also 
address previous efforts in generic geology repository modeling and the state of 
the art of repository modeling and integration within current systems analysis 
tools. 

% modeling paradigm

Chapter \ref{ch:paradigm} will detail the computational paradigm of the \Cyclus 
systems analysis platform as well as the components within the repository 
system. Models
representing waste form, waste package, buffer, backfill, and engineered barier 
systems will be defined by their interfaces and their relationships as 
interconnected modules, distinctly defined, but coupled. This modular paradigm 
allows exchange  of technological options (i.e. borosilicate glass and concrete 
waste forms) for comparison but also exchange of equivalent models with varying 
levels of detail.


% Categorization of current codes and physics

Chapter \ref{ch:categorization} will categorize and characterize detailed 
computational models of radionuclide and heat transport available for regression 
analysis. Specificially, detailed codes in current use are categorized according 
to the physics which they model, the disposal system components with which they 
are concerned, and the level of detail and computational methodology with which 
they capture physical phenomena. 

% Work on generic repository model

Chapter \ref{ch:current} will detail the analytical and regression analysis 
undertaken and forthcoming to acheive a generic repository model for the chosen 
base repository type. A concise, dominant physics geological repository model of 
the base case disposal environment will be developed. Informed by semi-analytic 
mathematical models representing important physical phenomena, existing detailed 
computational efforts characterizing these repository environments will be 
appropriately simplified to create concise computational models. This 
abstraction will capture fundamental physics of thermal, hydrogeologic, and 
radionuclide transport phenomena while remaining sufficiently detailed to 
illuminate behavioral differences between each of the geologic systems under 
consideration.  Verification and validation of abstracted models will be 
conducted through iterative benchmarking against more detailed repository 
models.

% Future work proposal

Chapter \ref{ch:future} will summarize the conclusions reached concerning the 
appropriate analytical and detailed models to utilize in the process of 
abstraction for radionuclide and heat transport through various components of 
the disposal system. Categorization of models and determination of the coverage 
within the option space domain will also be summarized. Finally, remaining 
future work and expected contributions to the field will be summarized. 

%%%%%%%% %%%%%%%% %%%%%%%% %%%%%%%% %%%%%%%% %%%%%%%% %%%%%%%% %%%%%%%%
%%%%%%%% %%%%%%%% %%%% These chapters may be saved for the thesis . . .  
%%%%%%%% %%%%%%%% %%%%%%%% %%%%%%%% %%%%%%%% %%%%%%%% %%%%%%%% %%%%%%%% 

% Chapter \ref{ch:ebs} will adapt existing models and data to the development of 
% concise dominant physics waste form, waste package, and other engineered 
% barriers (i.e., bentonite or cementitious materials) models appropriate for 
% treatment of key radionuclides within the waste streams.  Material/barrier 
% degradation, radionuclide release, and radionuclide transport, and thermal 
% processes and effects will be included, as necessary, in the concise 
% representations that will be developed for subsequent use in the system-level 
% architecture. A range of waste forms, waste package materials, and other 
% engineered barrier materials (buffer, backfill) under consideration by the 
% DOE-NE FCT program (SWF and UFD campaigns) will be evaluated. The concise 
% dominant physics models will include appropriate load limiting factors of the 
% stabilizing medium and waste packaging including such as waste composition, 
% chemical form, and heat generation.
% 
% Chapter \ref{ch:extension} will discuss the future work necessary to extend 
% developed models to comprehensively cover the potential disposal system option 
% space. The path forward for extension of the geological base case model to 
% cover all five geologic concepts of interest (clay/shale, granite, salt, and 
% deep boreholes) will be discussed. Similarly, gaps in waste form and 
% engineered barrier system models and data will be addressed and a plan for 
% data and model coverage for that options space will be described.

