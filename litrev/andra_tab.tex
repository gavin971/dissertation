\begin{table}
\centering
\footnotesize{
\begin{tabular}{|l|l|}
  \multicolumn{2}{c}{\textbf{Detailed Nuclide Transport Models Used in the ANDRA analysis.}}\\
\hline
Models &                                          Codes\\
\hline
Hydrogeology and “particle tracking”    & Connectflow (NAMMU component, 3D modelling,\\
in continuous porous media              &  finite elements).\\
                                        &  Geoan (3D modelling, finite differences).\\
                                        &  Porflow (3D modelling, finite differences).\\
Hydrogeology and “particle tracking”    &  Connectflow (NAMMU component, 3D modelling,\\
in discrete fracture networks.          &  finite elements).\\
                                        &  FracMan (generation of discrete fracture networks) and\\
                                        &  MAFIC (hydraulic resolution of the networks, 3D,\\
                                        &  finite elements).\\
Transport in continuous porous media.   &  PROPER (COMP-23 component, modelling in\\
                                        &  segments of the engineered barrier, finite differences).\\
                                        &  Goldsim (volume modelling of engineered barriers).\\
                                        &  Porflow.\\
Transport in discrete fracture networks.&  PROPER (FARF-31 component, 1D modelling 1D\\
                                        &  stream tube concept).\\
                                        &  PathPipe (conversion of networks of tubes for transport)\\
                                        &  and Goldsim (modelling in networks of 1D pipes).\\
\hline
\end{tabular}
\caption[Particle Transport Codes Used in ANDRA Assessment]{Similar to the Total System Performance Assessment, ANDRA's analyses are a coupled mass of many codes. Table reprouced from Argile Dossier 2005 \cite{andra_argile:_2005}}
\label{tab:andra}
}
\end{table}
