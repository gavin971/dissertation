\chapter{Literature Review}
The following literature review addresses five areas of current research integral to the work at hand. The contribution of computational nuclear fuel cycle simulation tools to sensitivity analyses of repository performance metrics is first summarized. A review of current computational repository models follows, including both standalone and those incorporated into nuclear fuel cycle simulation tools. Special focus is paid to the availability of supporting data and algorithms informing geochemical and hydrogeological transport on long time scales and in various geologies. Next, an overview is presented of available waste form performance models applicable to likely advanced fuel cycle waste streams, and finally, work is reviewed that concerns the need for simplified first order, physics based models of fuel cycle processes within the context of top level simulation. 
\section{Sensitivity Analyses of Repository Performance}
Comprehensive fuel cycle sensitivity analyses with an emphasis on used fuel disposition and waste management have been conducted by Li, Piet, Wilson, and Ahn. Primary repository performance metrics of interest have been heat, source term, and other metrics of envrionmental impact assessment.
%%%%%%%%%%%%%%%%%%%%%%%%%%%%%%%%%%%%%%%%%%%%
% This is where CapacityNotes was inserted %
%%%%%%%%%%%%%%%%%%%%%%%%%%%%%%%%%%%%%%%%%%%%

\section{Capacity Dependence on Heat Load}
\subsection{Heat Load Constraints}
\paragraph{}
Two heat load constraints primarily determine the heat-based SNF capacity limit in the Yucca Mountain Repository design. Thermal limits in the current design are intended to passively steward the repository's integrity against radionuclide release for the upcoming 10,000 years.
\paragraph{}
The first constraint intends to prevent repository flooding and subsequent contaminated water flow through the repository. It requires that the minimum temperature in the granite tuff between drifts be no more than the boiling temperature of water which, at the altitude in question, is $96^{\circ}C$. This is effectively a limit on the temperature halfway between adjacent drifts, where the temperature will be at a minimum.
\paragraph{}
The second constraint intends to prevent high rock temperatures that induce fractures and would increase leach rates. It states that no part of the rock reach a temperature above $200^{\circ}C$, and is effectively a limit on the temperature at the drift wall, where the rock temperature is a maximum. 
\subsection{Heat Load Metrics}
\paragraph{}
Line loading ($t/m$) and areal power density ($W/km^2$) are two common metrics for describing the fullness of the repository. While these metrics are informative for mass capacity and power capacity respectively, they fail to reflect differences in thermal behavior due to varying SNF compositions. A closer look at the isotopics of the situtation has proven much more applicable to thermal performance studies of the repository, and the preferred method in the current literature relies on specific temperature integrals.
\paragraph{}
\textbf{Specific Temperature Integrals} model the thermal source as linear along the repository drifts, similar to the line loading and areal power density metrics. However, a temperate integral takes account of heat transfer behavior in the rock, includes the effects of myriad SNF compositions, and gives the thermal integration over time for any specific location within the rock. Man-Sung Yim calls this the Specific Temperature Increase method\cite{YimSTI} though other researchers have other names for this method. Tracy Radel calls her temperature metric at a point in the rock the Specific Temperature Change.\cite{Radel}
\paragraph{}
Heat flux from the drifts can be expressed as the superposition of the linear heat flux contributions of all the radionuclides in the waste. Each radionuclide contributes in proportion to its decay heat generation and its weight fraction of the SNF. With information about isotopic composition of the SNF, the Specific Temperature Increase can determine the maximum thermal capacity of the repository in terms of tonnes/m. The length based accounting in $\frac{t}{m}$ is converted to $\frac{t}{Repository}$ by multiplication with the total emplacement tunnel length of the repository, 67 km.
\subsection{Current Wisdom}
\paragraph{}
The statutory limit of once-through, thermal PWR waste is 70,000 tonnes SNF. That is to say, the statutory line load limit is approximately 1.04 tonnes/m for 67km of planned emplacement tunnels (with 81 meters between drifts). The Office of Civilian Radioactive Waste Management Science and Engineering Report gives this basic ``statutory limit", but suggests an inherent design flexibility that could allow for expansion. The ``full inventory" Yucca Mountain design alternative gives a maximum repository capacity of 97,000 tonnes. In addition, the current design for the repository has flexibility for ``additional repository capacity" which would give a 119,000 tonne capacity at 1.04 tonnes/m.\cite{OCRWMSciEng} 
\paragraph{}
Specific Temperature Change analysis by Radel, Wilson, et al. find a maximum thermal capcity of 1.09 tonnes/m for commercial SNF (at an ELF of $49 GWd_{th}/m$).\cite{RadelWilson} 
\paragraph{}
Elongation of cooling times has the potential to expand the capacity of the repository. `Cooling time' refers to delaying complete loading of the repository. Longer cooling times allow high heat, short lived isotopes to decay to lower activity before they begin to heat the repository. Much of the benefit to repository capacity comes from the advantage that the cooling time allows a decrease in the space between emplacement drifts. Aged SNF has lower heat flux and so, the drift spacing can be decreased from 81 to 70 meters. A study by Man-Sung Yim and colleagues at North Carolina state found that for a representative commercial SNF composition a cooling time of 75 years allows for over 100 MTU SNF disposal without expanding the Yucca Mountain footprint.\cite{YimGlobal}
\paragraph{}
Similarly, age based fuel mixing also allows for decreases in drift spacing. In aged based fuel mixing, aged (long cool time) SNF is loaded in a mixture with young SNF. This age based fuel mixing has been shown to achieve a $48\%$ increase in the repository capacity as constrained by heat load.\cite{Nicholson} This factor uses a fiducial default footprint of $4.6 km^2$ used in the NRC TSPA. The reported $48\%$ increase in capacity results in total repository capacity of 103,600 tonnes.\cite{TSPA}
\paragraph{}
In addition to variable drift spacing, other modifications to repository layout have had promising results in terms of heat-limited repository capacity. The Electric Power Research Institute (EPRI) in their Room at the Mountain study found that with redesign of the repository an increased capacity of at least $400\%$ (295 tonnes once-through SNF) and up to $900\%$ (663 tonnes) could be expected to be achieved. Proposed design changes include decreased spacing between drifts, a larger areal footprint, vertical expansion into second and third levels of repository space, and hybrid solutions involving combinations of these ideas. In particular, EPRI suggests either an expansion of the footprint with redesign of the current Upper Block line load design plan or a multi-level plan that repeats the footprint and line load design of the current plan.\cite{EPRI}
\begin{centering}
\begin{tabular}{l}
\\
\begin{tabular}{l|c|c|r}
Author&Max. Capacity&Footprint&Details\\
&$tonnes$&$km^2$&\\
\hline
&&&\\
OCRWM&$70,000$&$4.65$&``statutory case''\\
&$97,000$&$6$&``full inventory case''\\
&$119,000$&$~7$&``additional case''\\
\hline
&&&\\
Yim, M.S.&$75,187$&$4.6$&SRTA code\\
&$76,493$&$4.6$&STI method\\
&$95,970$&$4.6$&$63$m drift spacing\\
&$82,110$&$4.6$&75 yrs. cooling\\
\hline
&&&\\
Nicholson, M.&$103,600$&$4.6$&drift spacing\\
\hline
&&&\\
EPRI&&&\\
&$63,000$&$6.5$&Base Case CSNF\\
option 1&$126,000$&$13$&expanded footprint\\
option 2&$189,000$&$6.5$&multi-level design\\
option 3&$189,000$&$6.5$&grouped drifts\\
options 2+3&$252,000$&$6.5$&hybrid\\
options 1+(2or3) &$378,000$&$13$&hybrid\\
options 1+2+3 &$567,000$&$13$&hybrid\\
\hline

\end{tabular}
\end{tabular}
\end{centering}

\subsection{Related Questions}
\begin{itemize}
\item{} What is the thermal dependence on wasteform characteristics?
%\begin{itemize}
%\item{Matthern}
%\item{VISION}
%\end{itemize}
\item{} What thermal effect does separation have?
%\begin{itemize}
%\item{Wigeland}
%\item{Djokic}
%\item{Matthern}
%\item{Ahn}
%\item{Yim}
%\end{itemize}
\item{} What kind of costs are related to drift-spacing redesign, footprint expansion, age based mixing, extended cooling times, etc..? 
\end{itemize}
\section{Capacity Dependence on Source Term}
\subsection{Radiotoxicity and Source Term Constraints}
\paragraph{}
The exposure limit set by the NRC is based on a `reasonably exposed individual.' That is to say, the limiting case is a person who lives, grows food, drinks water and breathes air 18 km downstream from the repository. The Yucca Mountain Repository legistlative regulations limit total dose from the repository to 15 mrem/yr, and limit dose from drinking water to 4 mrem/yr. Predictions of that dose rate depend on an enormous variety of factors. The primary pathway of radionuclides from an accidental release will be from cracking aged canisters. Transport of the radionuclides to the water table requires that the leakages come in contact with water and travel through the rock the water table. This results in contamination of drinking water downstream. 
\subsection{Radiotoxicity and Source Term Metrics}
\paragraph{}
Source term is a measure of the quantity of a radionuclide released into the environment and radiotoxicity is a measure of the hazardous effect of that particular radionuclide upon human ingestion. In particular, radiotoxicity is measured in terms of the volume of water dilution required to make it safe to ingest. Studies of source term and radiotoxicity therefore make probabilistic assessments of radionuclide release, transport, and human exposure. The probabilistic nature of these assessments mean a direct dependence of source term on repository capacity can be difficult to arrive at. In order to give informative values for the risk associated with transport of particular radionuclides, for example, studies make hundred thousand year predictions about waste form degredation, water flow, etc. 
\paragraph{}
A generalized metric of probablistic risk is fairly difficult to arrive at. The Peak Environmental Impact metric from Berkeley \cite{AhnGlobal}, for example, is a complicated function of spent fuel composition, waste conditioning, vitrification method, and radionuclide transport through the repository walls and rock. Also, it makes the assumption that the waste canisters have been breached at $t=0$. Furthermore, reported in $m^3$, PEI is a measure of radiotoxicity in the environment in the event of total breach. While informative, this model on its own does not completely determine a source-term limited maximum repository capacity. Additional waste package failure and exposed individual radiotoxicity constraints must be incorporated into it.
\subsection{Current Wisdom}
It has been shown with a great degree of uncertainty ($~50\%$) that the peak dose rate from a full, base case repository will occur in 300,000 years, and will reach no greater than 2.6 mrem/yr.\cite{Jun}
\paragraph{}
The Total System Performance Assessment on the other hand conservatively predicts a peak mean dose rate in the first million years (where the peak occurs at approximately 300,000 years) to be on the order of 150 mrem/yr.\cite{TSPA} 
\paragraph{}
Due to the incredible time scale and intrinsic uncertainties required in the probabilistic assessment it is in general not advisable to base any maximum repository capacity estimates on source term. However, source term is a useful metric for the comparison of alternative separations and fuel cycle scenarios.

\subsection{Factors Affecting Source Term}
A source term model will incorporate waste package failure rate, nuclide dissolution rate, and advective transfer rate. Waste package failure rate depends on near field environmental factors such as pH and humidity as well as decay heat and radiative damage anticipated from the contained waste. In turn, the nuclide release rate from the waste package depends on the character of the waste form matrix, treatment of water flow, nuclide solubility and the elemental diffusion constant. Similarly, advective transfer through the granite tuff also depends on water flow, nuclide solubility, and nuclide diffusion, but is employed in the context of the hydrogeology of the rock.   
\paragraph{Waste package failure rate} varies between models. While some employ a simulation code called EBSFAIL, a part of the EBSPAC module used in the TSPA code, other models incorporate their own hydrogeologic approximations of canister degredation, and still others assume immediate waste canister failure in order to focus on dissolution and transfer. 
\newline
\begin{centering}
\begin{table}
\begin{tabular}[h!bt]{l|r|r|r}
Model&WP Failure Mode&Waste Form&Time at first failure\\
\hline
TSPA&EBSFAIL&&$300,000$ years\\
\hline
Ahn 2003&Instantaneous Failure&Borosilicate Glass&$t=0$\\
\hline
Ahn 2007& &CSNF $UO_2$ matrix &$T_f=75,000$ years\\
& &Borosilicate Glass &$T_f=75,000$ years\\
& & Naval $UO_2$ matrix &$T_f=75,000$ years\\
\hline
Jun&EBSFAIL&&$300,000$ years\\
\end{tabular}
\end{table}
\end{centering}
\paragraph{Nuclide dissolution rate} can also be understood as nuclide release rate from the waste packages, and is the rate of mass transfer of a nuclide from its waste matrix into the saturation water. Models calculating  nuclide dissolution rate have assumed waste package failure insofar as the water is assumed to saturate the waste matrix. The mode of water flowthrough heavily effects nuclide dissolution rate and is treated differently in various models. While some, inspired by the TSP assessment, assume water moves through the waste packages at a constant volumetric rate (`flowthrough model'), others adopt less conservative assumptions incorporating weather based predictions of hydrogeologic activity. as directly proportional to the nuclide concentration in some cases,  
\paragraph{Advective transfer rate} through the granite tuff is dependent upon diffusion through the rock as well as water speed, etc. The diffusion coefficient varies per nuclide and is heavily dependent upon the concentration of that nuclide in the flowthrough water. This is just Fick's First Law. 
\begin{equation}
J = -D\frac{\delta\phi}{\delta x}
\end{equation}
\paragraph{Source term dependence on concentration} has a significant effect on potential repository capacity. Sensitivity to concentration complicates the viability of alternate loading schemes as well as waste separation scenarios. Radionuclide concentration has been shown to be proportional to the waste package loading configuration.\cite{AhnConfig2,AhnConfig}

The dissolution rate and the advective transfer rate through the granite tuff of a nuclide are typically be taken to depend directly upon the diffusion coefficient of that nuclide, which for some nuclides is heavily dependent upon the concentration of that nuclide relative to the flowthrough water. 

\subsection{Ahn Models (\cite{Ahn2004, Ahn2007})}
\paragraph{Basic Flowthrough Path}
Waste canisters are modelled as compartments of waste matrix surrounded by a buffer layer which is in turn surrounded by layers of near field rock and far field rock. Water is introduced to the system at a constant rate, and encounters an array of failed waste packages (at $t=0$ in the 2004 model, and at $T_f=75,000$ years in the 2007 model). The water immediately begins dissolving the waste matrix. Nuclides with higher solubilities are preferentially dissolved and treated with a `congruent release' model discussed below. Nuclides with lower solubilities are transported through the buffer with the alternative `solubility limited release model. The water flow begins at one waste package and travels through the matrix and buffer space to the next waste package, contacting each waste package consecutively and then flowing on into the near field. In this way, the water is increasingly contaminated as its path through the waste packages proceeds.  
\paragraph{Congruent Release Model} 
Nuclides with a high solubility coefficient are modeled with the congruent release model. Nuclides of this type include most of the fission products, but not the actinides. This model states that the release from the waste packages is congruent with the dissolution of the waste matrix and is transported through the rock by advective transfer with the water that flows through the waste packages.  
\paragraph{Solubility Limited Release Model}
Nuclides with lower solubility coefficients are modeled with the solubility limited release model. Solubility values are assumed from TSPA for this model, and a solubility of $~5\times 1^{-2} [mol/m^3]$ are taken to be `low.' Elements in this `low' category include the toxic actinides such as Zr, Nb, Sn, Th, and Ra. This model suggests that a dominant mode of dissolution of the nuclide into the flowthrough water is dominated instead by the diffusion coefficient, which is largely dependent upon the concentration gradient between the waste matrix and the water. The mass balance driving nuclide release takes the form:
\begin{equation}
\dot{m_i}=8\epsilon D_eS_iL\sqrt{\frac{Ur_0}{\pi D_e}}
\end{equation}
where $\epsilon$, U, $r_0$, and L are the geometric and hydrogeologic factors porosity, water velocity, waste package radius, and waste package length, repsectively. $D_e$ is the diffusion coefficient ($m^2/yr$) of the element \emph{e} and $S_i$ is the isotope's solubility ($kg/m^3$).

\subsection{Li Model\cite{Jun}}
\paragraph{Basic Flowthrough Path}
As a function of time, water enters the Engineered Barrier System and corrodes the waste packages. These fail and from the failed waste packages nuclides are released according to advective transfer. Further transportation through the near and far field rock medium is modeled in two modes, one representing the Unsaturated Zone, and one representing the Saturated Zone.
\paragraph{Waste package failure and nuclide release} are modeled with two TSPA code modules called EBSFAIL and EPSREL. The waste package failure rate is determined from EBSFAIL which incorporates waste form chemistry, humidity, oxidation, etc and upon contact from water begins the degradation process. The results of EBSFAIL become the input to EBSREL which models corresponding nuclide release from those failed waste packages. Mass balance governing the nuclide release rate in this model allows advective transfer to dominate and takes the form:
\begin{equation}
\dot{m_i}=w_{li}(t)-w_{ci}{t}-m_i\lambda_i+m_{i-1}\lambda_{i-1}\nonumber
\end{equation}
In this expression, $w_{li}(t)$ is the rate $[mol/yr]$ of isotope \emph{i} leached into the water. It is a function of water flow rate, chemistry, and isotope solubility. $m_i$ describes the mass of isotope \emph{i}, and $\lambda_i$ describes its decay constant. Finally, $w_{ci}(t)$ describes the advective transfer rate $[mol/yr]$ of the isotope \emph{i}. This model defines $w_{ci}$ as:
\begin{equation}
w_{ci}(t)=C_i(t)q_{out}(t)
\end{equation} 
where $q_out$ is the volumetric flow rate of the water $[m^3/yr]$, and $C_i = m_i/V_{wp}$ in $[mol/m^3]$. These assumptions fail to take into account any differences in the varying solubilities of the isotopes, but are quite sensitive to the concentration of an isotope \emph{i} in the waste package volume. 
\paragraph{The Unsaturated Zone} lies between the repository and the water table. This model describes transport time in the porous rock as:
\begin{equation}
T_a= \frac{X_u}{U_u}R_{du}
\end{equation}
where X is length of the unsaturated zone, U is the pore velocity of the water $[m/yr]$ and $R_{iu}$ is the retardation factor for the isotope \emph{i}. The retardation factor is poorly described here, but in general denotes migration distance of the solute over that of the solvent. Presumably, this factor incorporates isotope specific characteristics and is independent of or weakly dependent on concentration. 
\paragraph{The Saturated Zone} lies below the water table. At this point, the nuclide transport is taken to be completely advective, nuclide independent, and congruent with the volumetric flow of the water within the water table. 

\subsection{Related Questions}
\begin{itemize}
\item{} More detail on development of TSPA modules/solubility values?
\item{} Effect of wasteforms on source term? 
%\begin{itemize}
%\item{Djokic}
%\item{VISION}
%\item{Wigeland}
%\end{itemize}
\item{} More studies making probabilistic risk assessments of source term?
\end{itemize}
%\section{Economic Benefits of Reprocessing}
%\subsection{Subtleties}
%Safety increased by recycling
%Some recycling is a proliferation concern
%temporary storage an option.
%Infinite
%Different Fuel Cycle Scenarios
%\subsection{Current Wisdom, Back of The Envelope}
%EPRI Room At The Mountain Cost Estimates
%\paragraph{}
%To make a back of the envelope set of calculations, we'll make use of the current projected price of Yucca Mountain (96 billion?). Dividing by 70,000 tonnes, we have a \$/tonne value for the repository.
%In order to determine the amount saved in each scenario, we'll multiply the amount of additional SNF fit into the repository by that \$/tonne cost. This will give a very rough estimate of the Yucca Mountain equivalent expenditures avoided in each scenario. 

\section{Capacity Dependence on Heat Load}
\subsection{Heat Load Constraints}
\paragraph{}<++>
\subsection{Independent Fuel Cycle Parameters}
Independent fuel cycle parameters of particular interest to academics have been those related to the front end of the fuel cycle. Deployment decisions concerning reactor types, Fast to Thermal reactor ratios, and burnup rates can all be independently varied in fuel cycle simulation codes in such a way as to inform domestic policy decisiong going forward. A Some of these parameters are coupled, however, to aspects of the back end of the fuel cycle. For example, the appropriate fast reactor ratio is significantly altered by the chosen method and magnitude of domestic spent fuel reprocessing (or not).

However, independent variables representing decisions concerning the back end of the fuel cycle are of increasing interest as the United States further investigates repository alternatives to Yucca Mountain. Parameters such as the repository geology, tunnel design, and appropriate loading strategies and schedule are all independent variables up for debate. That said, some of these parameters are coupled with decisions about the fuel cycle. 

The point then, is that while independent parameters can be chosen and varied within a fuel cycle simulation, some parameters are coupled in such a way as to require full synthesis with a systems analysis code that appropriately determines the isotpic mass flows into the repository, their appropriate conditioning, densities, and other physical properties.  
%
%   * repository
%      * geology
%         * reducing/oxidizing
%         * salt
%         * granite
%         * clay
%         * deep boreholes (srsly)
%      * design
%         * tunnel widths?
%         * tunnel lengths?
%         * distance between tunnels?
%         * depths
%         * distance from water table
%      * loading strategies
%         * cooling pad timing
%         * interim storage timing (i.e. between fuel cycle stages)
%         * utilizing short lived ILW facilities
%         * tunnel ventilation
%         * packing optimization
%   * partitioning/separations
%      * separation efficiency 
%         * of MAs
%         * of Pu
%         * of Iodine
%         * etc.
%      * separation strategy
%         * isotope decisions
%            * MAs?
%            * Pu?
%            * etc.?
%         * chemistry decisions
%            * aqueous?
%            * pyro?
%            * does it matter?
%   * transmutation
%      * Advanced Reactors
%         * deployment timing
%         * Pu or MA content of fuels
%         * types of reactors
%            * high burnup
%            * MOX
%            * various Fast Reactors
%               * various conversion/breeding ratios
%               * Integral Fast Reactor
%               *  
%            * wave reactor
%            * candle reactor
%      * Thermal Reactors
%         * initial UO_2 enrichment
%         * maximum burnup 
%         * Pu / MA content of fuels
%         * Thorium ?
%         * Reactor Type
%            * PWR
%            * BWR 
%            * VVER
%      * Accelerator Driven Systems
%         * target MA content
%         * burnup efficiency
%      * power share of various installed reactor types
%   * conditioning
%      * waste packaging
%         * forms
%            * liquids
%               * glass vitrification
%               * cement
%               * bitumen
%            * solids
%               * steel
%               * combined material canisters
%               * dual purpose transport and storage canisters
%               * overpack characteristics
%            * etc?
%         * densities
%            * impacted by density-dependent release rates
%         * volumes
%      * incineration/compaction (solid waste)
%      * evaporation/filtration/ion exchange (liquid waste)
%      * engineered barriers
%      * likelihood of human intrusion ?
%
\subsection{Repository Performance Metrics}
On the back end, dependent variables are interesting as well. 
\subsubsection{Criticality Safety}
\subsubsection{Source Term}
% Undisrupted Scenario
% Disrupted Scenario
% RadioToxicity Index (Sv)
\subsubsection{Peak Dose (Sv/y)}
\subsubsection{Integrated Dose (Sv)}
\subsubsection{Maximum Environmental Release (Sv/y)}
\subsubsection{Integrated Envrionmental Release (Sv)}
\subsubsection{Necessary Repository Volume}
% Under various waste classification scenarios?
\subsubsection{Thermal Load}
\subsubsection{Cost}
\subsection{Current Methodologies}<++>

\section{Repository Models}
% The total system performance assessment is one type. Repository modules incorporated into VISION and whatnot are another type. 
% Things to ask about each of them include:
% Which geologies do they model?
% How long do they take to run? 
% Are they proprietary? 
% How well validated are they?
\subsection{Stand Alone Models}
% TSPA - Think about Stephen L. Turner Document ( lessons learned ).
% TSPA for wipp?
% Sweden?
% France? 
% Figure out which repositories acutally exist in the world. There's certainly a repository for each. 
\subsubsection{FSCNE}

\subsubsection{A$^3$MCNP}

\subsubsection{SCANS 1A}

\subsubsection{ReFREP}
Refrep is a near-field model.  A. Hautojarvi and T. Vieno Model For A Spent Fuel Technical Research Centre of Finland (VTT) Repository.\cite{gaps}
\subsection{Models Incorporated into Systems Analysis Codes}
\subsubsection{NUWASTE} 
Nuclear Waste Technical Review Board code that determines many metrics about the fuel cycle according to various parameters. \cite{NuwastePres} 

\subsubsection{VISION}
% VISION
VISION's repository model conducts decay calculations and tracks upwards of 83 isotopes of interest in the nuclear fuel cycle. \cite{VISION} Are there wasteform models in VISION?
\subsubsection{DANESS}<++>
% DANESS
\subsubsection{COSI}<++>
% COSI
\subsubsection{DYMOND}<++>
% DYMOND ?
\subsubsection{NFCSim}<++>
\subsubsection{CAFCA}<++>
\subsubsection{SMAFS}<++>
\subsubsection{NFCSS}<++>
% get rest of this list from MLDG paper.
\section{Geochemical Migration Models}
% Some mathematical models describing aqueous isotope transport in geologic settings are available absent of Engineered Barier Systems and the like. These solely model host rock transport. 
% ``PHREEQC version 2 is a computer program written in the C programming language that is designed to perform a wide variety of low-temperature aqueous geochemical calculations. PHREEQC is based on an ion-association aqueous model and has capabilities for (1) speciation and saturation-index calculations; (2) batch-reaction and one-dimensional (1D) transport calculations involving reversible reactions, which include aqueous, mineral, gas, solid-solution, surface-complexation, and ion-exchange equilibria, and irreversible reactions, which include specified mole transfers of reactants, kinetically controlled reactions, mixing of solutions, and temperature changes; and (3) inverse modeling, which finds sets of mineral and gas mole transfers that account for differences in composition between waters, within specified compositional uncertainty limits.''
% GTMCHEM — a simulation environment which “deterministically” models the one-dimensional migration of radionuclides through the geosphere up to the biosphere. Focusing on scenario and parametric uncertainty, we show that mean predicted maximum doses to humans on the earth's surface due to 1–129, and uncertainty bands around those predictions, are larger when scenario uncertainty is properly assessed and propagated. .
% Things to ask about each of them include:
% Which geologies do they model?
% How long do they take to run? 
% Are they proprietary? 
% How well validated are they?
\section{Waste Form Models}
\subsection{TAD Canisters}
% The canisters proposed for transportation, aging and disposal are called TAD canisters. 
% They are two concentric cylinders of steel and alloy22 inside and out respectively. 
\subsection{Borosilicate Glass}
% Current borosilicate glass: Includes processing chemicals from original separations, with U/Pu removed, but minor actinides and Cs/Sr remaining
% Potential borosilicate glass: No minor actinides and/or no Cs/Sr; Mo may be removed to increase glass loading of radionuclides; it has alower volumetric heat rate
\subsection{Glass Ceramic}
% Glass Ceramic:  This is glass-bonded sodalite from Echem processing of EBR-II, and from potential future Echem processing of oxide fuels o Metal Alloy: This includes subcategories
\subsection{Metal Alloy}
% Metal alloy from Echem: Includes cladding as well as noble metals that did not dissolve in the Echem dissolution
% Metal alloy from aqueous reprocessing:  Includes undissolved solids and transition metal fission products
\subsection{Advanced Ceramic}
% Advanced Ceramic: An advanced waste form that includes iodine volatilized during chopping, which is then gettered during head-end processing of used fuels
\subsection{Separated Streams}
% Other:  Examples include radionuclides removed from other waste forms (e.g., Cs/Sr, I, C), as well as new waste forms such as a salt waste form
\subsection{Classes A, B, and C waste}
% Lower Than High Level Waste (LTHLW): Includes Classes A, B, and C
\subsection{GTCC LTHLW}
% Greater Than Class C (GTCC)




