\chapter{Literature Review}
The following literature review addresses four areas of current research integral to the work at hand. First, a review of the contribution of computational nuclear fuel cycle simulation tools to sensitivity analyses of repository performance metrics for various fuel cycle parameters. A review of current computational repository models both standalone and those incorporated into nuclear fuel cycle simulation tools, follows. Special focus is paid to the availability and parametric regimes of supporting data and algorithms informing geochemical and hydrogeological transportmodels on long time scales and in various geologies. Finally, a gap analysis demonstrates the reange of available waste form performance models applicable to used fuel streams likely to result from various advanced fuel cycles.   
\section{Sensitivity Analyses of Repository Performance}
Comprehensive fuel cycle sensitivity analyses with an emphasis on used fuel disposition and waste management have been conducted by Li, Piet, and ≈

\subsection{Independent Fuel Cycle Parameters}
%
%   * repository
%      * geology
%         * reducing/oxidizing
%         * salt
%         * granite
%         * clay
%         * deep boreholes (srsly)
%      * design
%         * tunnel widths?
%         * tunnel lengths?
%         * distance between tunnels?
%         * depths
%         * distance from water table
%      * loading strategies
%         * cooling pad timing
%         * interim storage timing (i.e. between fuel cycle stages)
%         * utilizing short lived ILW facilities
%         * tunnel ventilation
%         * packing optimization
%   * partitioning/separations
%      * separation efficiency 
%         * of MAs
%         * of Pu
%         * of Iodine
%         * etc.
%      * separation strategy
%         * isotope decisions
%            * MAs?
%            * Pu?
%            * etc.?
%         * chemistry decisions
%            * aqueous?
%            * pyro?
%            * does it matter?
%   * transmutation
%      * Advanced Reactors
%         * deployment timing
%         * Pu or MA content of fuels
%         * types of reactors
%            * high burnup
%            * MOX
%            * various Fast Reactors
%               * various conversion/breeding ratios
%               * Integral Fast Reactor
%               *  
%            * wave reactor
%            * candle reactor
%      * Thermal Reactors
%         * initial UO_2 enrichment
%         * maximum burnup 
%         * Pu / MA content of fuels
%         * Thorium ?
%         * Reactor Type
%            * PWR
%            * BWR 
%            * VVER
%      * Accelerator Driven Systems
%         * target MA content
%         * burnup efficiency
%      * power share of various installed reactor types
%   * conditioning
%      * waste packaging
%         * forms
%            * liquids
%               * glass vitrification
%               * cement
%               * bitumen
%            * solids
%               * steel
%               * combined material canisters
%               * dual purpose transport and storage canisters
%               * overpack characteristics
%            * etc?
%         * densities
%            * impacted by density-dependent release rates
%         * volumes
%      * incineration/compaction (solid waste)
%      * evaporation/filtration/ion exchange (liquid waste)
%      * engineered barriers
%      * likelihood of human intrusion ?
%
\subsection{Repository Performance Metrics}<++>
\subsubsection{Criticality Safety}
\subsubsection{Source Term}
% Undisrupted Scenario
% Disrupted Scenario
% RadioToxicity Index (Sv)
\subsubsection{Peak Dose (Sv/y)}
\subsubsection{Integrated Dose (Sv)}
\subsubsection{Maximum Environmental Release (Sv/y)}
\subsubsection{Integrated Envrionmental Release (Sv)}
\subsubsection{Necessary Repository Volume}
% Under various waste classification scenarios?
\subsubsection{Thermal Load}
\subsubsection{Cost}
\subsection{Current Methodologies}<++>

\section{Geochemical Migration Models}
% Some mathematical models describing aqueous isotope transport in geologic settings are available absent of Engineered Barier Systems and the like. These solely model host rock transport. 
% Things to ask about each of them include:
% Which geologies do they model?
% How long do they take to run? 
% Are they proprietary? 
% How well validated are they?
\section{Repository Models}
% The total system performance assessment is one type. Repository modules incorporated into VISION and whatnot are another type. 
% Things to ask about each of them include:
% Which geologies do they model?
% How long do they take to run? 
% Are they proprietary? 
% How well validated are they?
\subsection{Stand Alone Models}
% TSPA - Think about Stephen L. Turner Document ( lessons learned ).
% TSPA for wipp?
% Sweden?
% France? 
% Figure out which repositories acutally exist in the world. There's certainly a repository for each. 
\subsubsection{FSCNE}

\subsubsection{A$^3$MCNP}

\subsubsection{SCANS 1A}

\subsubsection{ReFREP}
Refrep is a near-field model.  A. Hautojarvi and T. Vieno Model For A Spent Fuel Technical Research Centre of Finland (VTT) Repository.\cite{gaps}
\subsection{Models Incorporated into Systems Analysis Codes}
\subsubsection{VISION}
% VISION
\subsubsection{DANESS}<++>
% DANESS
\subsubsection{COSI}<++>
% COSI
\subsubsection{DYMOND}<++>
% DYMOND ?
\subsubsection{NFCSim}<++>
\subsubsection{CAFCA}<++>
\subsubsection{SMAFS}<++>
\subsubsection{NFCSS}<++>
% get rest of this list from MLDG paper.
\section{Waste Form Models}
\subsection{TAD Canisters}
% The canisters proposed for transportation, aging and disposal are called TAD canisters. 
% They are two concentric cylinders of steel and alloy22 inside and out respectively. 
\subsection{Borosilicate Glass}
% Current borosilicate glass: Includes processing chemicals from original separations, with U/Pu removed, but minor actinides and Cs/Sr remaining
% Potential borosilicate glass: No minor actinides and/or no Cs/Sr; Mo may be removed to increase glass loading of radionuclides; it has alower volumetric heat rate
\subsection{Glass Ceramic}
% Glass Ceramic:  This is glass-bonded sodalite from Echem processing of EBR-II, and from potential future Echem processing of oxide fuels o Metal Alloy: This includes subcategories
\subsection{Metal Alloy}
% Metal alloy from Echem: Includes cladding as well as noble metals that did not dissolve in the Echem dissolution
% Metal alloy from aqueous reprocessing:  Includes undissolved solids and transition metal fission products
\subsection{Advanced Ceramic}
% Advanced Ceramic: An advanced waste form that includes iodine volatilized during chopping, which is then gettered during head-end processing of used fuels
\subsection{Separated Streams}
% Other:  Examples include radionuclides removed from other waste forms (e.g., Cs/Sr, I, C), as well as new waste forms such as a salt waste form
\subsection{Classes A, B, and C waste}
% Lower Than High Level Waste (LTHLW): Includes Classes A, B, and C
\subsection{GTCC LTHLW}
% Greater Than Class C (GTCC)




