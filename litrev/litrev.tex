\chapter{Literature Review}\label{ch:litrev}

The following literature review addresses five areas of current 
research integral to the work at hand. The contribution of 
computational nuclear fuel cycle simulation tools to sensitivity 
analyses of repository performance metrics is first summarized. A 
review of analytical models of nuclide transport follows, after which 
follows a review of analytical models of heat transport. An overview
of current detailed computational models, available data and 
algorithms characterizing nuclide transport follows, including both 
standalone and those incorporated into nuclear fuel cycle simulation 
tools. Finally, a review of current computational models of heat 
transport in the waste disposal system context is given.  Special 
focus is paid to the availability of supporting data and algorithms
informing geochemical and hydrogeological transport on long time 
scales and in various geologies. 

% Next, an overview is presented of available waste form performance 
% models, and finally, work is reviewed that concerns the need for 
% simplified first order, physics based models of fuel cycle processes 
% within the context of top level simulation.  


\section{Repository Capabilities within Systems Analysis Tools}
\label{sec:SA_repos}

%%%%%%%%%%%%%%%% Systems Analysis Repository Capabilities %%%%%%%%%%%%%%%%%%%%%%%%
% The total system performance assessment is one type. Repository 
% modules incorporated into VISION and whatnot are another type.  
% Things to ask about each of them include:
% Which geologies do they model?
% How long do they take to run?  Are they proprietary?  How well 
% validated are they?

%% This section comes from 2011 CFP Narrative 3067, only lightly edited %%
Current top-level simulators largely disregard the waste disposal 
phase of fuel cycle analysis. Choosing instead to report metrics such 
as mass or volumes of accumulated spent nuclear fuel, these analyses 
fail to
address the impact of those waste streams on the performance of the 
geologic disposal system.  To fully inform the decision making 
process, metrics that depend on the performance of the geologic 
disposal system will be necessary. 

A model for repository capacity was developed for the VISION fuel 
cycle simulator \cite{yacout_visionverifiable_2006}
\cite{radel_repository_2007} and recent efforts on the NUWASTE 
simulator \cite{ abkowitz_nuclear_2010} have made some progress in 
addressing this deficiency, but despite a proliferation of 
sophisticated fuel cycle simulators, similar efforts are lacking in 
this regard. 

%% What follows has not been edited since the CFP narrative %%
The DOE-NE Fuel Cycle Technology (FCT) program has three groups of 
relevance to this effort.  These are the FCT Used Fuel Disposition 
(UFD), the Separations/Waste Form (SWF), and System Analysis (SA) 
campaigns.  The UFD campaign is conducting the RD\&D related to the 
storage, transportation, and disposal of radioactive wastes generated 
under both the current and potential advanced fuel cycles.  The SWF 
campaign is conducting RD\&D on potential waste forms that could be 
used to effectively isolate the wastes that would be generated in 
advanced fuel cycles.  The SWF and UFD campaigns are developing the 
fundamental tools and information base regarding the performance of 
waste forms and geologic disposal systems.  The SA campaign is 
developing the overall fuel cycle simulation tools and interfaces with 
the other FCT campaigns, including UFD. 

This effort will interface with those campaigns todevelop the 
higher-level dominant physics representations for use in fuel cycle 
system analysis tools.  Specifically, this work will leverage upon 
conceptual framework development and primary data collection underway 
within the Used Fuel Disposition Campaign as well as work by Radel, 
Wilson, Bauer et. al. to model repository behavior as a function of 
the contents of the waste.
It will then incorporate dominant physics process models into the 
\Cyclus computational fuel cycle analysis platform.

%This work encompasses the implementation of a comprehensive software 
%library of medium fidelity models to represent the thermal behavior 
%and long-term disposal system performance of different disposal 
%system concepts in different geologic media for deployment in a 
%modular systems analysis platform. 

%%%%%%%% RED-IMPACT Table %%%%%%%%%%%%
%%%%%%%%%%%%%%%%%%%%%%%%%%%%%%%%%%%%%%

\begin{table}
  \centering
  \footnotesize{
  \begin{tabular}{|l|c|c|c|c|}
    \multicolumn{5}{c}{\textbf{International Repository Concepts}}\\
    \hline
    Geology     & Nation      & Waste Stream   & Metric    & Institution \\
    \hline 
    Granite     & Spain       & HLW            & Heat Load & Enresa  \\
    Granite     & Czech Rep.  & HLW            & Heat Load & NRI \\
    Clay        & Belgium     & HLW            & Heat Load & SCK$\cdot$CEN \\
    Salt        & Germany     & HLW            & Heat Load & GRS \\
    Granite     & Spain       & HLW            & Dose      & Enresa  \\
    Clay        & Belgium     & HLW            & Dose      & SCK$\cdot$CEN \\
    Clay        & France      & HLW            & Dose      & CEA \\
    Salt        & Germany     & HLW            & Dose      & GRS  \\
    Granite     & Czech Rep.  & ILW            & LT Dose   & NRI  \\
    Granite     & Spain       & ILW            & LT Dose   & Enresa  \\
    Clay        & Belgium     & ILW            & LT Dose   & SCK$\cdot$CEN  \\
    Granite     & Spain       & HLW/ILW/Iodine & LT Dose   & Enresa \\
    Clay        & Belgium     & HLW/ILW/Iodine & LT Dose   & SCK$\cdot$CEN \\
    \hline
  \end{tabular}
  \caption[International Repository Concepts]{International repository concepts evaluated in the RED Impact 
  Assessment.\cite{von_lensa_red-impact_2008}}
  \label{tab:red}
  }
\end{table}



%%%%%%%%%%%%%%%%%%%%%%%%%%%%%%%%%%%%%%

\clearpage

\subsection{NUWASTE} Nuclear Waste Technical Review Board code that 
determines many metrics about the fuel cycle according to various 
parameters. \cite{abkowitz_nuclear_2010} This code tracks 65 isotopes 
within material objects, discretely models individual shipping casks 
and incorporates cooling time in both dry and wet intermediate storage 
facilities.

\subsection{VISION}
VISION's repository model conducts decay calculations and tracks 
upwards of 83 isotopes of interest in the nuclear fuel cycle.  
\cite{yacout_visionverifiable_2006} 
% Are there wasteform models?
% Is there nuclide transport, source term estimate?
% Heat transport?

\subsection{DANESS}

\subsection{COSI}

\subsection{DYMOND}

\subsection{NFCSim}

\subsection{CAFCA}

\subsection{SMAFS}

\subsection{NFCSS}

\subsection{European RED-IMPACT}

\subsection{Yucca Mountain Total System Performance Assesment}
The Total System Performance Assessment (TSPA) is a very detailed 
model of the Yucca Mountain disposal system. It includes 
transportation issues and detailed emplacement timing and strategy 
models, but considers only the fuel cycle associated with the current  
U.S. reactor fleet. Casts are modeled discretely and nuclide and heat 
transport are modeled in great detail. 

\subsection{WIPP Performance Assessment Code}

\subsection{Fuel Cycle Technology, UFD Codes Under Development}

The Used Fuel Disposition campaign is currently conducting an effort 
to produce generic disposal system models for various geological 
environments. Teams from various national labs, Argonne, Lawrence 
Livermore, and Sandia are developing models of generic clay, granite,   
and salt disposal environments. Sandia is simultaneously constructing  
a borehole disposal system mode. Each generic disposal system model 
will perform detailed calculations of nuclide  and heat transport 
within its respective lithology. 

The nuclide transport calculations within the  clay model being 
developed by Argonne National Lab (ANL) are performed within the 
GoldSim simulation platform, and models a single repeatable waste 
package cell. 

Thermal modeling for the clay model is being conducted using the 
SINDA$\$G heat transport solver.  

\subsection{Repository Focused Fuel Cycle Analyses}

Focused fuel cycle sensitivity analyses emphasizing used fuel 
disposition and waste management in the Yucca Mountain Repository 
(YMR) have been conducted by Li, Piet, Wilson, and Ahn. With a focus 
on YMR capacity benefit, repository performance metrics of interest 
for these analyses were heat, source term, and more global 
envrionmental impact metrics.  Sensitivity analyses for other 
geologies were conducted concerning repository concepts relevant to 
other nations as well. See Table \ref{tab:red}.

%%%%%%%%%%%%%%%%%%%%%%%%%%%%%%%%%%%%%%
\section{Analytical Models of Nuclide Transport} 
\label{sec:analytical_nuc}

%%%%%%%%%%%%%%%% Analytical Nuclide Transport %%%%%%%%%%%%%%%%%%%%%%%%

A comprehensive model of radiotoxic source term must address nuclide 
transport through the full release pathway including waste packages, 
engineered barrier systems, and geologic media. A model of transport 
through the waste package must incorporate waste package failure rate, 
nuclide release rate via waste matrix dissolution, and advective 
transfer rate into the engineered barrier system.  Waste package 
failure rate depends on near field environmental factors such as pH 
and humidity as well as decay heat and radiative damage anticipated 
from the contained waste.  In turn, the nuclide release rate from the 
waste package depends on the character of the waste form matrix, 
treatment of water flow, nuclide solubility and the elemental 
diffusion constant.  Similarly, advective transfer through the 
engineered barrier system and into the geological medium also depends 
on water flow, nuclide solubility, and nuclide diffusion, but must be 
employed in the context of the hydrogeology of the rock.   

\subsection{Factors Affecting Source Term}

Waste package failure rate varies between models. While some employ a 
simulation code called EBSFAIL, a part of the EBSPAC module used in 
the TSPA code, other models incorporate their own hydrogeologic 
approximations of canister degredation, and still others assume 
immediate waste canister failure in order to focus on dissolution and 
transfer. 

Waste package release rate is the rate of mass transfer of a nuclide 
from its waste form into the saturation water.
The mode of water flowthrough heavily effects nuclide dissolution rate 
and is treated differently in various models. While some, inspired by 
the TSP assessment, assume water moves through the waste packages at a 
constant volumetric rate (`flowthrough model'), others adopt less 
conservative assumptions incorporating weather based predictions of 
hydrogeologic activity. as directly proportional to the nuclide 
concentration in some cases,  

Nuclide transfer rate through the lithology is  dependent upon 
diffusion as well as advection.  The diffusion coefficient varies per 
nuclide and is heavily dependent upon the concentration of that 
nuclide in the flowthrough water. Nuclide concentration in this model 
is described by Fick's First Law, 

\begin{equation}
J = -D\frac{\delta\phi}{\delta x}.
\end{equation}

Source term dependence on concentration has a significant effect on 
potential repository capacity. Sensitivity to concentration 
complicates the viability of alternate loading schemes as well as 
waste separation scenarios. Radionuclide concentration has been shown 
to be proportional to the waste package loading configuration for the 
Yucca Mountain 
case.\cite{ahn_relationship_2002,kawasaki_congruent_2004}

\subsection{Waste Form Release Models}

\subsubsection{Hedin Model (\cite{hedin_integrated_2002})}

In a saturated fractured rock matrix representative of the KBS-3 
granitic Swedish repository concept, copper canister waste packages 
contain a waste matrix, and a bentonite buffer surrounds the canisters 
within repository drift tunnels. Waste form dissolution within the 
Hedin model is a rate based model which takes place within the waste 
package void. Nuclides are released congruently until the waste form 
is completely degraded. \cite{hedin_integrated_2002} 

\subsubsection{Ahn Models (\cite{ahn_environmental_2004, 
ahn_environmental_2007})}

Waste canisters are modelled as compartments of waste matrix 
surrounded by a buffer layer which is in turn surrounded by layers of 
near field rock and far field rock. Water is introduced to the system 
at a constant rate, and encounters an array of failed waste packages 
(at $t=0$ in the 2004 model, and at $T_f=75,000$ years in the 2007 
model). The water immediately begins dissolving the waste matrix.  
Nuclides with higher solubilities are preferentially dissolved and 
treated with a `congruent release' model discussed below. Nuclides 
with lower solubilities are transported through the buffer with the 
alternative `solubility limited' release model. The water flow begins 
at one waste package and travels through the matrix and buffer space 
to the next waste package, contacting each waste package consecutively 
and then flowing on into the near field. In this way, the water is 
increasingly contaminated as its path through the waste packages 
proceeds.  

\subsubsection{Congruent Release Model} 

In the Ahn models, nuclides with a high solubility coefficient are 
modeled with the congruent release model.  Nuclides of this type 
include most of the fission products, but not the actinides. This 
model states that the release from the waste packages is congruent 
with the dissolution of the waste matrix and is transported through 
the rock by advective transfer with the water that flows through the 
waste packages.  

\subsubsection{Solubility Limited Release Model}

In the Ahn models, nuclides with lower solubility coefficients are 
modeled with the solubility limited release model.  Solubility values 
are assumed from TSPA for this model, and a solubility of $~5\times 
1^{-2} [mol/m^3]$ are taken to be `low.' Elements in this `low' 
category include the toxic actinides such as Zr, Nb, Sn, Th, and Ra.  
This model suggests that a dominant mode of dissolution of the nuclide 
into the flowthrough water is dominated instead by the diffusion 
coefficient, which is largely dependent upon the concentration 
gradient between the waste matrix and the water. The mass balance 
driving nuclide release takes the form:

\begin{equation}
\dot{m_i}=8\epsilon D_eS_iL\sqrt{\frac{Ur_0}{\pi D_e}}
\end{equation}
where $\epsilon$, U, $r_0$, and L are the geometric and hydrogeologic 
factors porosity, water velocity, waste package radius, and waste 
package length, repsectively. $D_e$ is the diffusion coefficient 
($m^2/yr$) of the element \emph{e} and $S_i$ is the isotope's 
solubility ($kg/m^3$).

In the Hedin model of the waste matrix, the amount of solute available 
within the waste package is solved for, and for nuclides with low 
solubility, the mass fraction released from the waste matrix is 
limited by a simplified description of their solubility. That is, 

\begin{align*}
  m_{1i}(t)\le v_{1i}(t)C_{sol}
\end{align*}

where the mass $m_{1i}$ $[g]$ of a nuclide, $i$ relased into the waste 
package void volume $v_1$ in $[m^3]$, at a time t, is limited by 
constant the maximum concentration, $C_{sol}$ in $[g/m^3]$ at which 
that nuclide is soluble. \cite{hedin_integrated_2002}

%%%%

\subsection{Waste Package Failure Models}

%%%%%%%%%%%%%%%%%%%%%%%%%%%%%%%%%%%%%%
%%%%% WP Failure Modes Table %%%%%%%%%
%%%%%%%%%%%%%%%%%%%%%%%%%%%%%%%%%%%%%%
\begin{table}[h!]
\centering
\footnotesize{
\begin{tabular}[h!bt]{|l|r|r|r|}
  \multicolumn{4}{c}{\textbf{Current Waste Package Failure Models}}\\
  \hline
  Model&WP Failure Mode&Waste Form&Details\\
  \hline
  TSPA&EBSFAIL&&$300,000$ years\\
  \hline
  Ahn 2003&Instantaneous Failure&Borosilicate Glass&$t=0$\\
  \hline
  Ahn 2007& &CSNF $UO_2$ matrix &$T_f=75,000$ years\\
  & &Borosilicate Glass &$T_f=75,000$ years\\
  & & Naval $UO_2$ matrix &$T_f=75,000$ years\\
  \hline
  Li&EBSFAIL&&$300,000$ years\\
  \hline
  Hedin 2003& Instantaneous & Copper KBS-3 Concept & $t_{delay} = 300$ years \\
  \hline
\end{tabular}
\label{tab:wpfail}
\caption[Current WP Failure Models]{The above represent current methods by which waste packeage 
failure rates are modeled.}
}
\end{table}

%%%%%%%%%%%%%%%%%%%%%%%%%%%%%%%%%%%%%%

\subsubsection{Continuous}

\subsubsection{Instantaneous}
 
The Hedin model of waste package failure is effectively instantaneous, 
but limited by release resistance coefficient. The release is assumed  
to occur through a hole in the waste canister that exists throughout 
the simulation, and the resistance coefficient limiting flow through 
the hole represents the magnitude of the canister flaw in combination
with the buffer-geosphere interface.  \cite{hedin_integrated_2002}

\subsubsection{Probabilistic}

%%%%

\subsection{Nuclide Transport Through Engineered Barriers}

\subsubsection{Barrier Dissolution and Failure}

\subsubsection{Transport Through EBS Matrix}


%%%%

\subsection{Hydrogeologic Transport Models}

\subsubsection{Solute Transport in Permeable Porous Media}
Clay, granite, salt, and shale can larely be characterized as 
permeable porous media.

% Solute Transport in permeable porous media
% advection - transport at the velocity of water flow
% hydraulic dispersion - transport due to anisotropies in the water 
% velocity feild. It depends on concentration of solute, darcy 
% velocity, and dispersivity.
% diffusion - is from random brownian motion, tends to homogenize 
% concentration field.
%

The equation representing solute transport in a permeable medium of 
homogenous porosity can be written
\begin{align*}
  \frac{\partial \omega C}{\partial t} & = - \nabla \cdot  (F_c + 
  F_{dc} + F_d) + m \intertext{where}
   \omega &= \mbox{ solute accessible porosity } [\%]\\
   C &= \mbox{ concentration } [kg \cdot m^{-3}]\\
   F_c &= \mbox{ convetive flow } [kg \cdot m^{-2}\cdot s^{-1}]\\
       &= qC \\
   F_{dc} &= \mbox{ dispersive flow } [kg \cdot m^{-2}\cdot s^{-1}]\\
       &= \alpha q \nabla C  \\
   F_d &= \mbox{ diffusive flow } [kg \cdot m^{-2}\cdot s^{-1}]\\
       &= D_e \nabla C\\
   m &= \mbox{ solute source } [kg \cdot m^{-2}\cdot s^{-1}].\\
\end{align*}
In the expressions above,
\begin{align*}
  q &= \mbox{ Darcy velocity } [m\cdot s^{-1}] \\ \alpha &= \mbox{ 
  dispersivity } [m] \intertext{and}
  D_e &= \mbox{ effective diffusion coefficient } [m^2\cdot s^{-1}] 
  .\\ \end{align*}
The method by which the dominant solute transport mode is determined 
for a particular porous medium is by use of the dimensionless Peclet 
number,
\begin{align*}
  Pe &= \frac{qL}{\alpha q + D_e},
  \intertext{where}
  L &= \mbox{ transport distance } [m].\\
\end{align*}

\subsubsection{Solute Transport in Fractured Media}
\paragraph{Continuum Models}
The models arrived at via this genre of approximation are appropriate 
for very fractured or very unfractured situations.

Equivalent Porous Medium (EPM) models assert that a uniformly  
fractured medium can be approximated as a fractureless matrix with an 
effective porosity high enough to account for real fracturation.
\cite{berkowitz_continuum_1988}
\cite{anderson_applied_1992}


Dual Porosity Models make up one type. This model incorporates 
advective transport in simplistic, uniform fractures and diffisive 
sorption and desorption into the stagnant (no advective transfer) 
water contained in the pores of the rock matrix
\cite{uleberg_dual_1996}
\cite{ho_dual_2000}.


Dual Permeability Models are another type. These are similar to dual
porosity models, but incorporate advective transfer within the rock 
matrix and between the rock matrix and the fracture volume.
\cite{uleberg_dual_1996}
\cite{ho_dual_2000}

\paragraph{Discrete Fracture Network Models}
Discrete fracture network models approximate that water and 
contaminants move only through the fracture network 
\cite{anderson_applied_1992}
\cite{schwartz_fundamentals_2003}.

The flow in each fracture can be approximated, as in Schwartz and 
Zhang, with the flow between two parallel plates having an aperture 
$b$, the mean fracture height \cite{schwartz_fundamentals_2003}. For a 
fracture perpendicular to gravitational acceleration, $g$, the 
hydraulic conductivity, $K$, is described according to the cubic law 
as 

\begin{align}
  K&= \frac{\rho_w g b^2}{12 \mu}
  \label{Kplates}
  \intertext{where}
  \rho_w &= \mbox{water density,}\nonumber\\
  \mu &= \mbox{viscosity}.\nonumber
\end{align}

Accordingly, the volumetric flow rate in the single fracture of width,
$w$, can be described in terms of the hydraulic head gradient, 
$\frac{\partial h}{\partial L}$, as

\begin{align}
  Q & = -Kbw\frac{\partial h}{\partial L}
  \label{Qplates}
\end{align}

Calculation of the volumetric flow rate and corresponding solute
transport in a discrete fracture network model for many non-parallel
fractures is an intensive numerical computation. However, for 
uniformly
fractured media, a fracture network can be approximated by a set of 
parallel plates fractures. 

If flow is expect in the $\theta_f$ direction, and the fractures of 
the set are spaced a distance, $s$, apart,

\begin{align}
  N &= \mbox{fracture frequency}\nonumber\\
  &= \frac{cos(\theta_f)}{s}.
  \label{fracfreq}
\end{align}

The fracture network permeability is then defined as,
\begin{align}
  k_f = \frac{b^3}{12N}.
  \label{fracperm}
\end{align}

The permeability, $k$, in an equivalent permeability model is thereby 
obtained
by the permeabilities of the fracture network, $k_f$, and the 
permeability
of the host matrix, $k_m$. Following the derivation in Schwartz and 
Zhang, in terms of the cross sectional contact areas of the matrix and 
fractures $A_m$ and $A_f$, the equivalent permeability, $k$, can be 
expressed

\begin{align}
  k = \frac{k_m + \frac{A_f}{A_m}k_f}{1+\frac{A_f}{A_m}}.
  \label{equivperm}
\end{align}

\subsubsection{Geochemical Transport Models}

%\subsubsection{Reducing Environments}

%\paragraph{Saturated Environments}

%\paragraph{Unsaturated Environments}

%\subsubsection{Oxidizing Environments}

%\paragraph{Saturated Environments}

%\paragraph{Unsaturated Environments}


% The distribution of fracture aperture sizes in a saturated fractured 
% hard rock matrix determines the appropriate model with which to 
% treat rock fracturation and subsequent nuclide transport through 
% fracturous pathways. 

% Fracturation within a saturated medium can be treated either as an 
% effective porosity, in which n_eff is suggested according to the 
% quantity of connected fractures, much like connected pores, in a 
% rock matrix. If these are of similar sizes, 

% Fracturation in grantite, tuff, and basalt have a log normal 
% distribution of of fracture apertures. Therefore, there are a few 
% large fractures but the majority are microfractures.  Therefore, the 
% equivalent porous medium approach is inappropriate. Instead we must 
% take the approach in which fractures of larger aperture are given 
% greater importance and are modeled as primary conduits to the 
% biosphere while minor fractures can be considered a part of the 
% porous medium of the saturated rock. \cite{ahn_thesis}


% For details about fracturation models in unsaturated rock, it's 
% likely best to consult YMR TSPA models.

%\paragraph{Sorption Into Matrix}

%Sorption into the rock matrix is a method by which contaminants (and 
%water? ) are removed from a fracture during flow through that 
%fracture. However, this is a reversible process \cite{ahn_thesis, pg 
%16}, which means the contaminant might be returned to the fracture 
%from the matrix with the same ''distribution coefficient`` with which 
%they entered the rock matrix . 

% One way to model sorption is by assuming a linear isotherm. What 
% does that mean?  \cite{ahn_thesis_pg20}

% The concentration of sorbed contaminant onto the fracture surface 
% can be demonstrated in terms of a rate equation (if surface 
% diffusion is neglected.) (What is surface diffusion?) 
% \cite{ahn_thesis_pg20)

% Ahn finds that including sorption, the concentration of contaminant 
% in the fracture satisfies the differential rate equation:
% 
% \begin{align*}
% R_f\frac{\partial N}{\partial t} + \nu \frac{\partial N}{\partial z} 
% - D \frac{\partial^2 D}{\partial z^2} + R_f \lambda N + \frac{q}{b} 
%   = 0, z>0 and t>0
% \end{align*}

% Also, there is sorption through the pores in the matrix. Again, we 
% can model this with a ``linear sorption isotherm. . . '' whatever 
% that is. 


%\paragraph{Diffusion Into Fracture}

%\paragraph{Advection Through Fracture}

%\paragraph{Effective Porosity}

%\paragraph{Major and Minor Fracturation}



\section{Analytical Models of Heat Transport}
\label{sec:analytical_heat}

%%%%%%%%%%%%%%%% Analytical Heat Transport %%%%%%%%%%%%%%%%%%%%%%%%

\subsection{Impact of Repository Designs}

\subsection{Heat Limits in Various Waste Packages}
% The CEA and ANDRA take 90$^\circ$C to be the maximum temperature for 
% spent fuel waste packages. The reason for this is to remain within 
% well understood limits of material evolution of waste packages and 
% the surrounding (clay) repository. \cite{argile_geo_evo}  

% Confinement capabilities of bitumen waste packages (B packages in 
% ANDRA Argile evaluation) is only confident below 70$^\circ$C. Uncide 
% the ``disposal cells'' a maximum temperature of 30$\circ$C is 
% adopted.


\subsection{Heat Limits in Various Geologies}

\subsubsection{Clay}
% Well understood behavior for agrillaceous clay is taken by the CEA 
% to occur under 90$^\circ$C, as with the waste packages in this 
% geology.
% 
% ``This low permeability of the Callovo-Oxfordian, linked to low 
% hydraulic head gradients on either side of the formation, controls 
% slow vertical water flow (Inset 3.5). The velocities of this water 
% flow are fairly different in detail because of the medium's pore 
% structure. Moreover, some pores are not connected and cannot take 
% part in the water flow. A so-called kinematic porosity is therefore 
% defined, which is a fraction of total porosity, used macroscopically 
% to calculate mean water flow velocity in the direction of the 
% hydraulic head gradient. For the Callovo-Oxfordian, this kinematic 
% porosity has been taken as being the same as the fraction of free 
% water in the rock, i.e. about 9 %, corresponding to half the total 
% porosity. The very low permeability determines average flow speeds 
% within the layer (Darcy velocity, inset 3.5) at around 3 cm per 
% 100,000 years, which corresponds to a water transfer velocity of 
% about 30 cm per 100,000 years, considering the kinematic porosity.
% \cite{argile_geo_evo}

% Water flow in a porous medium is dominated by darcy flow.  This 
% would be a great place to describe darcy's law! 

\subsubsection{Granite}

\subsubsection{Salt}
Response of a salt repository to heat has a significant mechanical 
component. Bulk heating of a salt repository matrix causes coalescing  
of the salt surrounding the heat source. In the case of a nuclear 
waste repository, this phenomenon increases isolation capability of 
the salt. A heat limit, then, is difficult to characterize, but 
evolution of the heat in a salt environment is of great importance to 
nuclide transport modeling. 

A model of temperature dependent salt coalescent behavior is is order. 

% WIPP, system model
% Germany, GRS concept

\subsubsection{Shale}


\paragraph{Unsaturated Tuff}

Two heat load constraints primarily determine the heat-based SNF 
capacity limit in the Yucca Mountain Repository design, which is 
located in unsaturated tuff. Thermal limits in that design are 
intended to passively steward the repository's integrity against 
radionuclide release for the upcoming 10,000 years.

The first constraint intends to prevent repository flooding and 
subsequent contaminated water flow through the repository. It requires 
that the minimum temperature in the granite tuff between drifts be no 
more than the boiling temperature of water which, at the altitude in 
question, is $96^{\circ}C$. This is effectively a limit on the 
temperature halfway between adjacent drifts, where the temperature 
will be at a minimum.

The second constraint intends to prevent high rock temperatures that 
induce fractures and would increase leach rates. It states that no 
part of the rock reach a temperature above $200^{\circ}C$, and is 
effectively a limit on the temperature at the drift wall, where the 
rock temperature is a maximum.  

\subsection{Tuff}

%%%%%%%%%%%%%%%%%%%%%%%%%%%%%%%%%%%%%%
%%%%% Heat Load Analytical Models Table %%%%%%%%%
%%%%%%%%%%%%%%%%%%%%%%%%%%%%%%%%%%%%%%
 \begin{table}
    \centering
    \footnotesize{
    \begin{tabular}{|l|c|c|l|}
      \multicolumn{4}{c}{\textbf{Models of Heat Load for Various Geologies}}\\
      \hline
      Source & Nation & Geology & Methodology \\  
      (Who) & (Where) & (What) & (How) \\  
      \hline
      Enresa \cite{von_lensa_red-impact_2008}           & Spain       & Granite       &  CODE\_BRIGHT  \\ 
      NRI   \cite{von_lensa_red-impact_2008}            & Czech Rep.  & Granite       &  Specific Temperature Integral   \\
      ANDRA \cite{andra_granite:_2005}                  & France      & Granite       &  3D Finite Element CGM code   \\
      SKB \cite{ab_long-term_2006}                      & Sweden      & metagranite   &  Forsmark / Laxemar Site \\
                                                        &             &               &  Descriptive Model (SDM)\\
      SCK$\cdot$CEN   \cite{von_lensa_red-impact_2008}  & Belgium     & Clay          &  Specific Temperature Integral   \\ 
      ANDRA \cite{andra_argile:_2005}                   & France      & Argile Clay   &  3D Finite Element CGM code   \\
      NAGRA \cite{johnson_project_2002, johnson_calculations_2002}  & Switzerland  & Opalinus Clay &  3D Finite Element CGM code \\
      GRS \cite{von_lensa_red-impact_2008}              & Germany     & Salt          &  HEATING (3D finite difference)   \\ 
      NCSU(Li)   \cite{li_examining_2007}               & USA         & Yucca Tuff    &  Specific Temperature Integral \\        
      NCSU(Nicholson) \cite{nicholson_thermal_2007}     & USA         & Yucca Tuff    &  SRTA and COSMOL codes\\
      Radel \& Wilson \cite{radel_repository_2007}      & USA         & Yucca Tuff    &  Specific Temperature Change \\ 
      \hline
    \end{tabular}
    \caption[Models for Heat Transport for Various Geologies]{Methods by which to calculate heat 
    load are independent of geology. Maximum heat load constraints, however, vary among host formations. }
    \label{tab:heat}
    }
  \end{table}

%%%%%%%%%%%%%%%%%%%%%%%%%%%%%%%%%%%%%%


The statutory limit of once-through, thermal PWR waste is 70,000 
tonnes SNF. That is to say, the statutory line load limit is 
approximately 1.04 tonnes/m for 67km of planned emplacement tunnels 
(with 81 meters between drifts). The Office of Civilian Radioactive 
Waste Management Science and Engineering Report gives this basic 
``statutory limit", but suggests an inherent design flexibility that 
could allow for expansion. The ``full inventory" Yucca Mountain design 
alternative gives a maximum repository capacity of 97,000 tonnes. In 
addition, the current design for the repository has flexibility for 
``additional repository capacity" which would give a 119,000 tonne 
capacity at 1.04 tonnes/m.\cite{ doe_yucca_2002}

Specific Temperature Change analysis by Radel, Wilson, et al. find a 
maximum thermal capcity of 1.09 tonnes/m for commercial SNF (at an ELF 
of $49 GWd_{th}/m$).\cite{radel_effect_2007} 

Elongation of cooling times has the potential to expand the capacity 
of the repository. `Cooling time' refers to delaying complete loading 
of the repository. Longer cooling times allow high heat, short lived 
isotopes to decay to lower activity before they begin to heat the 
repository. Much of the benefit to repository capacity comes from the 
advantage that the cooling time allows a decrease in the space between 
emplacement drifts. Aged SNF has lower heat flux and so, the drift 
spacing can be decreased from 81 to 70 meters. A study by Man-Sung Yim 
and colleagues at North Carolina state found that for a representative 
commercial SNF composition a cooling time of 75 years allows for over 
100 MTU SNF disposal without expanding the Yucca Mountain 
footprint.\cite{li_examining_2007}

Similarly, age based fuel mixing also allows for decreases in drift 
spacing. In aged based fuel mixing, aged (long cool time) SNF is 
loaded in a mixture with young SNF. This age based fuel mixing has 
been shown to achieve a $48\%$ increase in the repository capacity as 
constrained by heat load.\cite{nicholson_thermal_2007} This factor 
uses a fiducial default footprint of $4.6 km^2$ used in the NRC TSPA.  
The reported $48\%$ increase in capacity results in total repository 
capacity of 103,600 tonnes.\cite{williams_contract_2001}

In addition to variable drift spacing, other modifications to 
repository layout have had promising results in terms of heat-limited 
repository capacity. The Electric Power Research Institute (EPRI) in 
their Room at the Mountain study found that with redesign of the 
repository an increased capacity of at least $400\%$ (295 tonnes 
once-through SNF) and up to $900\%$ (663 tonnes) could be expected to 
be achieved. Proposed design changes include decreased spacing between 
drifts, a larger areal footprint, vertical expansion into second and 
third levels of repository space, and hybrid solutions involving 
combinations of these ideas. In particular, EPRI suggests either an 
expansion of the footprint with redesign of the current Upper Block 
line load design plan or a multi-level plan that repeats the footprint 
and line load design of the current plan.\cite{kessler_room_2006}

\begin{table}
  \centering
      \footnotesize{
      \begin{tabular}{|l|c|c|r|}
          \multicolumn{4}{c}{\textbf{Yucca Mounting Footprint Expansion Calculations}}\\
          \hline
          Author&Max. Capacity&Footprint&Details\\
          &$tonnes$&$km^2$&\\
          \hline
          &&&\\
          OCRWM&$70,000$&$4.65$&``statutory case''\\
          &$97,000$&$6$&``full inventory case''\\
          &$119,000$&$~7$&``additional case''\\
          \hline
          &&&\\
          Yim, M.S.&$75,187$&$4.6$&SRTA code\\
          &$76,493$&$4.6$&STI method\\
          &$95,970$&$4.6$&$63$m drift spacing\\
          &$82,110$&$4.6$&75 yrs. cooling\\
          \hline
          &&&\\
          Nicholson, M.&$103,600$&$4.6$&drift spacing\\
          \hline
          &&&\\
          EPRI&&&\\
          &$63,000$&$6.5$&Base Case CSNF\\
          option 1&$126,000$&$13$&expanded footprint\\
          option 2&$189,000$&$6.5$&multi-level design\\
          option 3&$189,000$&$6.5$&grouped drifts\\
          options 2+3&$252,000$&$6.5$&hybrid\\
          options 1+(2or3) &$378,000$&$13$&hybrid\\
          options 1+2+3 &$567,000$&$13$&hybrid\\
          \hline
        \end{tabular}
        \caption[Yucca Mountain Footprint Expansion Calculations]{Various analyses based on heat 
        load limited repository designs have resulted in footprint expansion calculations of the 
        YMR.} 
        }
      \end{table}

\subsection{Specific Temperature Integral}

Line loading ($t/m$) and areal power density ($W/km^2$) are two common 
metrics for describing the fullness of the repository. While these 
metrics are informative for mass capacity and power capacity 
respectively, they fail to reflect differences in thermal behavior due 
to varying SNF compositions.  A closer look at the isotopics of the 
situtation has proven much more applicable to thermal performance 
studies of the repository, and the preferred method in the current 
literature relies on specific temperature integrals.


Specific Temperature Integrals model the thermal source as linear 
along the repository drifts, similar to the line loading and areal 
power density metrics. However, a temperate integral takes account of 
heat transfer behavior in the rock, includes the effects of myriad SNF 
compositions, and gives the thermal integration over time for any 
specific location within the rock.  Man-Sung Yim calls this the 
Specific Temperature Increase method\cite{li_specific_2008} though 
other researchers have other names for this method. Tracy Radel calls 
her temperature metric at a point in the rock the Specific Temperature 
Change.\cite{radel_repository_2007}

In a repository with linear drifts, the Heat flux from the drifts can 
be expressed as the superposition of the linear heat flux 
contributions of all the radionuclides in the waste. Each radionuclide 
contributes in proportion to its decay heat generation and its weight 
fraction of the SNF. With information about isotopic composition of 
the SNF, the Specific Temperature Increase can determine the maximum 
thermal capacity of the repository in terms of tonnes/m. The length 
based accounting in $\frac{t}{m}$ is converted to 
$\frac{t}{Repository}$ by multiplication with the total emplacement 
tunnel length of the repository.  In the case of Yucca Mountain, this 
was 67 km.


\section{Detailed Computational Models of Nuclide Transport}
\label{sec:detailed_nuclide}

%%%%%%%%%%%%%%%% Detailed Nuclide Transport %%%%%%%%%%%%%%%%%%%%%%%%


\section{Detailed Computational Models of Heat Transport}
\label{sec:detailed_heat}

%%%%%%%%%%%%%%%% Detailed Heat Transport %%%%%%%%%%%%%%%%%%%%%%%%


%%%%%%%%%%%%%%%%%%%% Line of Awesomeness %%%%%%%%%%%%%%%%%%%%%%%%%%%%%%%


%%%%%%%%%%%%%%%%%%%%%%%%%%%%%%%%%%%%%%%%%%%%
% This is where CapacityNotes was inserted %
%%%%%%%%%%%%%%%%%%%%%%%%%%%%%%%%%%%%%%%%%%%%

\subsubsection{Li Model\cite{li_methodology_2006}}
As a function of time, water enters the Engineered Barrier System and 
corrodes the waste packages.  These fail and from the failed waste 
packages nuclides are released according to advective transfer.  
Further transportation through the near and far field rock medium is 
modeled in two modes, one representing the Unsaturated Zone, and one 
representing the Saturated Zone.

\paragraph{Waste package failure and nuclide release} are modeled with 
two TSPA code modules called EBSFAIL and EPSREL. The waste package 
failure rate is determined from EBSFAIL which incorporates waste form 
chemistry, humidity, oxidation, etc and upon contact from water begins 
the degradation process. The results of EBSFAIL become the input to 
EBSREL which models corresponding nuclide release from those failed 
waste packages. Mass balance governing the nuclide release rate in 
this model allows advective transfer to dominate and takes the form:

\begin{equation}
\dot{m_i}=w_{li}(t)-w_{ci}{t}-m_i\lambda_i+m_{i-1}\lambda_{i-1}\nonumber
\end{equation}

In this expression, $w_{li}(t)$ is the rate $[mol/yr]$ of isotope 
\emph{i} leached into the water.  It is a function of water flow rate, 
chemistry, and isotope solubility. $m_i$ describes the mass of isotope 
\emph{i}, and $\lambda_i$ describes its decay constant. Finally, 
$w_{ci}(t)$ describes the advective transfer rate $[mol/yr]$ of the 
isotope \emph{i}. This model defines $w_{ci}$ as:

\begin{equation}
w_{ci}(t)=C_i(t)q_{out}(t)
\end{equation}

where $q_out$ is the volumetric flow rate of the water $[m^3/yr]$, and 
$C_i = m_i/V_{wp}$ in $[mol/m^3]$. These assumptions fail to take into 
account any differences in the varying solubilities of the isotopes, 
but are quite sensitive to the concentration of an isotope \emph{i} in 
the waste package volume.  

\paragraph{The Unsaturated Zone} lies between the repository and the 
water table. This model describes transport time in the porous rock 
as:
\begin{equation}
T_a= \frac{X_u}{U_u}R_{du}
\end{equation}
where X is length of the unsaturated zone, U is the pore velocity of 
the water $[m/yr]$ and $R_{iu}$ is the retardation factor for the 
isotope \emph{i}. The retardation factor is poorly described here, but 
in general denotes migration distance of the solute over that of the 
solvent. Presumably, this factor incorporates isotope specific 
characteristics and is independent of or weakly dependent on 
concentration.  

\paragraph{The Saturated Zone} lies below the water table. At this 
point, the nuclide transport is taken to be completely advective, 
nuclide independent, and congruent with the volumetric flow of the 
water within the water table. 



\subsection{Independent Fuel Cycle Parameters}
Independent fuel cycle parameters of particular interest in fuel cycle 
systems analysis have been those related to the front end of the fuel 
cycle. Deployment decisions concerning reactor types, fast to thermal 
reactor ratios, and burnup rates can all be independently varied in 
fuel cycle simulation codes in such a way as to inform domestic policy 
decisiong going forward. A Some of these parameters are coupled, 
however, to aspects of the back end of the fuel cycle. For example, 
the appropriate fast reactor ratio is significantly altered by the 
chosen method and magnitude of domestic spent fuel reprocessing (or 
not).

However, independent variables representing decisions concerning the 
back end of the fuel cycle are of increasing interest as the United 
States further investigates repository alternatives to Yucca Mountain.  
Parameters such as the repository geology, tunnel design, and 
appropriate loading strategies and schedule are all independent 
variables up for debate. That said, some of these parameters are 
coupled with decisions about the fuel cycle. 

The point then, is that while independent parameters can be chosen and 
varied within a fuel cycle simulation, some parameters are coupled in 
such a way as to require full synthesis with a systems analysis code 
that appropriately determines the isotpic mass flows into the 
repository, their appropriate conditioning, densities, and other 
physical properties.  

%
%   * repository
%      * geology
%         * reducing/oxidizing
%         * salt
%         * granite
%         * clay
%         * deep boreholes (srsly)
%      * design
%         * tunnel widths?
%         * tunnel lengths?
%         * distance between tunnels?
%         * depths
%         * distance from water table
%      * loading strategies
%         * cooling pad timing
%         * interim storage timing (i.e. between fuel cycle stages)
%         * utilizing short lived ILW facilities
%         * tunnel ventilation
%         * packing optimization
%   * partitioning/separations
%      * separation efficiency * of MAs
%         * of Pu
%         * of Iodine
%         * etc.
%      * separation strategy
%         * isotope decisions
%            * MAs?
%            * Pu?
%            * etc.?
%         * chemistry decisions
%            * aqueous?
%            * pyro?
%            * does it matter?
%   * transmutation
%      * Advanced Reactors
%         * deployment timing
%         * Pu or MA content of fuels
%         * types of reactors
%            * high burnup
%            * MOX
%            * various Fast Reactors
%               * various conversion/breeding ratios
%               * Integral Fast Reactor
%               *  * wave reactor
%            * candle reactor
%      * Thermal Reactors
%         * initial UO_2 enrichment
%         * maximum burnup * Pu / MA content of fuels
%         * Thorium ?
%         * Reactor Type
%            * PWR
%            * BWR * VVER
%      * Accelerator Driven Systems
%         * target MA content
%         * burnup efficiency
%      * power share of various installed reactor types
%   * conditioning
%      * waste packaging
%         * forms
%            * liquids
%               * glass vitrification
%               * cement
%               * bitumen
%            * solids
%               * steel
%               * combined material canisters
%               * dual purpose transport and storage canisters
%               * overpack characteristics
%            * etc?
%         * densities
%            * impacted by density-dependent release rates
%         * volumes
%      * incineration/compaction (solid waste)
%      * evaporation/filtration/ion exchange (liquid waste)
%      * engineered barriers
%      * likelihood of human intrusion ?
%

%%%%%%%%%%%%%%%%%%%%%%%%%%%%%%%%%%%%%%
%% Source Term in Many Geos Table %%%%
%%%%%%%%%%%%%%%%%%%%%%%%%%%%%%%%%%%%%%
  \begin{table}[h!]
    \centering
    \footnotesize{
    \begin{tabular}{|l|c|c|l|}
      \multicolumn{4}{c}{\textbf{Models of Source Term for Various Geologies}}\\
      \hline
      Source & Nation & Geology & Methodology \\  
      (Who) & (Where) & (What) & (How) \\  
      \hline
      Enresa \cite{von_lensa_red-impact_2008}           & Spain       & Granite                   &  GoldSim Proprietary Framework\\ 
                                                        &             &                           & $^{129}I$ primary contributor \\
      SCK$\cdot$CEN   \cite{von_lensa_red-impact_2008}  & Belgium     & Clay                      & Features, events, processes\\
                                                        &             &                           & $^{129}I$ primary contributor \\
      GRS \cite{von_lensa_red-impact_2008}              & Germany     & Salt                      & Systematic Performance Asessment \\
                                                        &             &                           & $^{135}$Cs, $^{129}$I, $^{226}$Ra, $^{229}$Th \\
      Ahn \cite{ahn_environmental_2004, ahn_environmental_2007} & USA     & Yucca Tuff            & Solubility Limited Release \& \\ 
                                                        &             &                           & Congruent Release  \\
      NCSU(Nicholson) \cite{li_methodology_2006}        & USA         & Yucca Tuff                & TSPA codes EBSREL and EBSFAIL  \\ 
      WIPP                                              & USA         & Salt                      & ?  \\
      NAGRA \cite{johnson_project_2002, johnson_calculations_2002}  & Switzerland & Opalinus Clay & TAME code  \\
      ANDRA \cite{andra_argile:_2005}                   & France      & Argile Clay               & Very detailed CEA code  \\
                                                        &             &                           & Mostly homogeneous medium \\
                                                        &             &                           & $^{129}I$ primary contributor \\
      ANDRA \cite{andra_granite:_2005}                  & France      & Granite                   &  Very detailed CEA code  \\
                                                        &             &                           &  Involves fracturation of medium \\
                                                        &             &                           & $^{129}I$ primary contributor \\
      SKB \cite{ab_long-term_2006}                      & Sweden      & Forsmark                  &  HYDRASTAR solute transport\\
                                                        &             & Laxemar                   &  FracMan for fracturation\\
      \hline
    \end{tabular}
    \caption[Models of Source Term for Various Geologies]{Methods by which to 
    evaluate source term dependence of waste package failure, transport through 
    the \gls{EBS} and hydrogeologic transport. The latter two parts vary significantly among host formations. }
    \label{tab:geosource}
    }
  \end{table}

%%%%%%%%%%%%%%%%%%%%%%%%%%%%%%%%%%%%%%

%%%%%%%%%%%%%%%%%%%%%%%%%%%%%%%%%%%%%%
%%%%%% Andra Sub-Codes Table %%%%%%%%%
%%%%%%%%%%%%%%%%%%%%%%%%%%%%%%%%%%%%%%
\begin{table}
\centering
\footnotesize{
\begin{tabular}{|l|l|}
  \multicolumn{2}{c}{\textbf{Detailed Nuclide Transport Models Used in the ANDRA analysis.}}\\
\hline
Models &                                          Codes\\
\hline
Hydrogeology and particle tracking      &  Connectflow (3D finite element)\\
in continuous porous media              &  Geoan (3D finite differences).\\
                                        &  Porflow (3D finite differences).\\
Hydrogeology and particle tracking      &  Connectflow (3D finite elements).\\
in discrete fracture networks.          &  FracMan (discrete fracture networks) and\\
                                        &  MAFIC (3D finite elements).\\
Transport in continuous porous media.   &  PROPER (finite differences),\\
                                        &  Goldsim (control volumes), and\\ 
                                        &  Porflow (control volumes?).\\
Transport in discrete fracture networks.&  PROPER (1D stream tube concept).\\
                                        &  PathPipe (networks of tubes)\\
                                        &  and Goldsim (networks of 1D pipes).\\
\hline
\end{tabular}
\caption[Particle Transport Codes Used in ANDRA Assessment]{Similar to the Total System Performance Assessment, ANDRA's analyses are a coupled mass of many codes. Table reprouced from Argile Dossier 2005 \cite{andra_argile:_2005}}
\label{tab:andra}
}
\end{table}

%%%%%%%%%%%%%%%%%%%%%%%%%%%%%%%%%%%%%%


\subsection{Models Incorporated into Systems Analysis Codes}

% get rest of this list from MLDG paper.
%\section{Geochemical Migration Models}
% Some mathematical models describing aqueous isotope transport in 
% geologic settings are available absent of Engineered Barier Systems 
% and the like. These solely model host rock transport.  ``PHREEQC 
% version 2 is a computer program written in the C programming 
% language that is designed to perform a wide variety of 
% low-temperature aqueous geochemical calculations. PHREEQC is based 
% on an ion-association aqueous model and has capabilities for (1) 
% speciation and saturation-index calculations; (2) batch-reaction and 
% one-dimensional (1D) transport calculations involving reversible 
% reactions, which include aqueous, mineral, gas, solid-solution, 
% surface-complexation, and ion-exchange equilibria, and irreversible 
% reactions, which include specified mole transfers of reactants, 
% kinetically controlled reactions, mixing of solutions, and 
% temperature changes; and (3) inverse modeling, which finds sets of 
% mineral and gas mole transfers that account for differences in 
% composition between waters, within specified compositional 
% uncertainty limits.''
% GTMCHEM — a simulation environment which “deterministically” models 
% the one-dimensional migration of radionuclides through the geosphere 
% up to the biosphere. Focusing on scenario and parametric 
% uncertainty, we show that mean predicted maximum doses to humans on 
% the earth's surface due to 1–129, and uncertainty bands around those 
% predictions, are larger when scenario uncertainty is properly 
% assessed and propagated. .
% Things to ask about each of them include:
% Which geologies do they model?
% How long do they take to run?  Are they proprietary?  How well 
% validated are they?

%\section{Waste Form Models}

%\subsection{TAD Canisters}
% The canisters proposed for transportation, aging and disposal are 
% called TAD canisters.  They are two concentric cylinders of steel 
% and alloy22 inside and out respectively. 

%\subsection{Borosilicate Glass}
% Current borosilicate glass: Includes processing chemicals from 
% original separations, with U/Pu removed, but minor actinides and 
% Cs/Sr remaining
% Potential borosilicate glass: No minor actinides and/or no Cs/Sr; Mo 
% may be removed to increase glass loading of radionuclides; it has 
% alower volumetric heat rate

%\subsection{Glass Ceramic}
% Glass Ceramic:  This is glass-bonded sodalite from Echem processing 
% of EBR-II, and from potential future Echem processing of oxide fuels 
% o Metal Alloy: This includes subcategories

%\subsection{Metal Alloy}
% Metal alloy from Echem: Includes cladding as well as noble metals 
% that did not dissolve in the Echem dissolution
% Metal alloy from aqueous reprocessing:  Includes undissolved solids 
% and transition metal fission products

%\subsection{Advanced Ceramic}
% Advanced Ceramic: An advanced waste form that includes iodine 
% volatilized during chopping, which is then gettered during head-end 
% processing of used fuels

%\subsection{Separated Streams}
% Other:  Examples include radionuclides removed from other waste 
% forms (e.g., Cs/Sr, I, C), as well as new waste forms such as a salt 
% waste form

%\subsection{Classes A, B, and C waste}
% Lower Than High Level Waste (LTHLW): Includes Classes A, B, and C

%\subsection{GTCC LTHLW}
% Greater Than Class C (GTCC)




