As the United States \gls{DOE} simultaneously considers alternative fuel cycles 
and waste disposal options, an integrated fuel cycle and generic disposal system 
analysis tool is increasingly necessary for informing domestic nuclear spent 
fuel policy. A generic repository model capabable of illuminating the distinct 
dominant physics of candidate repository lithologies, designs, and engineering 
components will provide an interface between the \gls{UFD} and \gls{SA} Campaign 
goals. Repository metrics such as necessary repository footprint and peak annual 
dose are affected by heat and nuclide release characteristics specific to 
variable spent fuel compositions associated  with alternative fuel cycles. 
Computational tools capable of simulating the dynamic, heterogeneous spent fuel 
isotopics resulting from transition scenarios and alternative fuel cycles cycles 
are, however, lacking in repository modeling  options. This work proposes to 
construct such a generic repository model appropriate for systems analysis. By 
emphasizing modularity and speed, the work at hand seeks to  provide a tool 
which captures the dominant physics of detailed repository analysis within the 
\gls{UFD} Campaign and can be robustly and flexibly integrated within fuel cycle 
simulation tools such as the \Cyclus tool developed at the University of 
Wisconsin - Madison.


\glsresetall
