\begin{wbepi}{David C.~Makinson (1965)}
It is customary for authors of academic books to include in their prefaces statements such as this: ``I am indebted to ... for their invaluable help; however, any errors which remain are my sole responsibility.'' Occasionally an author will go further. Rather than say that if there are any mistakes then he is responsible for them, he will say that there will inevitably be some mistakes and he is responsible for them....

Although the shouldering of all responsibility is usually a social ritual, the admission that errors exist is not --- it is often a sincere avowal of belief. But this appears to present a living and everyday example of a situation which philosophers have commonly dismissed as absurd; that it is sometimes rational to hold logically incompatible beliefs.
\end{wbepi}

Above is the famous ``preface paradox,'' a placeholder, \textbf{\textsc{Do Not 
Believe}}.

