\chapter{Summary and Future Work}\label{ch:future}

\section{Summary}

\subsection{Motivation}

Most fuel cycle simulators are lacking in repository analysis. 

The \gls{UFD} campaign needs an interface with the \gls{SA} campaign.

\subsection{Current Work}

Abstraction of current detailed repository models in combination with the 
current \Cyclus framework has generated a fundamental modeling concept and 
demonstrated capability for modeling the nuclide transport and heat evolution of 
a generic geological repository.

\section{Future Work}

\subsection{Sensitivity Analysis}

The sensitivity analysis underway to arrive at simplified dominant physics 
relationships between the various input parameters of the simulation. 

\subsubsection{Nuclide Transport}

These are the independent parameters I'm interested in.

A TABLE, perhaps. Including parametric domain.

This is how I'll vary them.

Show that these provide a largely complete set of input variables to define 
nuclide transport. Discuss variables that are being neglected for simplicity, 
perhaps to be added later. 

\subsubsection{Heat Evolution}

These are the independent parameters I'm interested in.

A TABLE, perhaps. Including parametric domaie.

This is how I'll vary them.

Show that these provide a largely complete set of input variables to define 
heat evolution. Discuss variables that are being neglected for simplicity, 
perhaps to be added later. 

\subsection{Mathematical Model Abstraction}

Using the results of these sensitivity analyses, regression analysis will be 
performed to develop simplified parametric dependencies for all independent 
variables for each model. 

\subsection{Computational Model Development}

\subsubsection{Base Case}

Clay, bentonite backfill, no evacuation disturbed zone, some handfull of waste 
packages and waste forms.

\paragraph{TAD Canisters} The canisters proposed for transportation, aging and 
disposal are called TAD canisters.  They are two concentric cylinders of steel 
and alloy22 inside and out respectively. 


\paragraph{Borosilicate Glass} Current borosilicate glass: Includes processing 
chemicals from original separations, with U/Pu removed, but minor actinides and 
Cs/Sr remaining Potential borosilicate glass: No minor actinides and/or no 
Cs/Sr; Mo may be removed to increase glass loading of radionuclides; it has 
alower volumetric heat rate


\paragraph{Glass Ceramic} Glass Ceramic:  This is glass-bonded sodalite from 
Echem processing of EBR-II, and from potential future Echem processing of oxide 
fuels o Metal Alloy: This includes subcategories


\paragraph{Metal Alloy} Metal alloy from Echem: Includes cladding as well as 
noble metals that did not dissolve in the Echem dissolution Metal alloy from 
aqueous reprocessing:  Includes undissolved solids and transition metal fission 
products


\paragraph{Advanced Ceramic} Advanced Ceramic: An advanced waste form that 
includes iodine volatilized during chopping, which is then gettered during 
head-end processing of used fuels


\paragraph{Separated Streams} Other:  Examples include radionuclides removed 
from other waste forms (e.g., Cs/Sr, I, C), as well as new waste forms such as 
a salt waste form

\paragraph{Classes A, B, and C waste} Lower Than High Level Waste (LTHLW): 
Includes Classes A, B, and C

\paragraph{GTCC LTHLW}  Greater Than Class C (GTCC)


\subsubsection{Demonstration}

Show that the complete model behaves in agreement with the more detailed model 
on which it was based. Else, iterate through sensitivity analyses, model 
abstraction, and computational development until the model is validated. 



\subsubsection{Extension}

This disposal system will at this point be extended to include a fleet of 
predefined model implementations to represent some canonical waste forms, 
packages, buffers, and clay types.  

\subsubsection{Fuel Cycle Analysis}

Show some various fuel cycles have different repository needs and metrics. 
Specifically, compare a closed fuel cycle, an open one, and at least one 
modified fuel cycle. 




% UFD developing metrics for the fct program option screening these and perhaps 
% other metrics will be included. 

