\chapter{Summary and Future Work}\label{ch:future}

\section{Summary}

\subsection{Motivation}

The need for this work has been shown by a summary of the current state of the 
art of fuel cycle simulator repository capabilities. The literature review 
concluded that most fuel cycle simulators lack repository analysis beyond basic 
mass tracking. The integrated radionuclide transport and thermal analysis to be 
pursued in this work will provide a currently unavailable tool for disposal 
system analysis. An immediate need for such a tool has been expressed in the 
\gls{UFD} campaign roadmap for this year in which an interface with the \gls{SA} 
campaign was noted as a primary goal. 

\subsection{Current Work}

Development of the \Cyclus fuel cycle simulator has generated a tool with which 
the systems analysis aspects of this work will be conducted. Development of the  
fundamental repository module concept appropriate for modular integration with 
\Cyclus has laid the foundation for a software effort which will deliver a 
library of disposal system component models capable of analyzing current 
disposal concepts of interest both domestically and internationally.

\section{Future Work}

\subsection{Demonstration Case Development}

% Demonstration case milestone

A first milestone in the development of this software will be a  proof of 
principle demonstration of the data structure and information passing schemes.  
That is, no physics will be implemented in the demonstration milestone. Rather, 
the component models will be developed in such a way that they are capable of 
passing their information passing schemes and placeholder functions will be 
generated in place of physics.

\subsubsection{Demonstration Case Concept}

  % Concept

    % Facility Interface, complete

    % Control Volumes

    % Dynamic loading

  The demonstration will produce a complete but `empty' repository model. That 
  is, on the repository scale the appropriate facility interface will be 
  implemented, which allows the repository modeled to be loaded into a \Cyclus 
  simulation. A structure of subcomponent control volumes will also be 
  implemented which can be dynamically loaded at runtime.

    % Heat Information Passing

    % Nuclide Information Passing

    % Database writing, structure

  At the subcomponent level, information passing, bookkeeping, and mass and 
  energy balances within and between the subcomponent control volumes.  
  Information passing between subcomponents concerning heat and radionuclide 
  concentration boundary conditions will be implemented. A complementary output 
  database structure will be defined and bookeeping for writing relevant 
  radionuclide and heat transport information.  

    % basic mass conservation checking

    % basic energy conservation checking

\subsubsection{Demonstration Code Development}

  % Code Development
    
  Initial code development on the base case model has begun with creation of the 
  FacilityModel subclass, GenericRepository. 

  A first phase in the demonstration milestone will be for the repository model 
  to successfully load its subcomponent models from user input (i.e.  the 
  lithology, waste form, waste package, buffer, etc.) with their corresponding 
  defining parameters.  

  A second phase will involve implementing an information exchange paradigm 
  between the subcomponents which communicates sufficient temperature and 
  contaminant concentrations at the boundaries of the control volumes for 
  solution of their internal transport calculations.

  When these aspects are implemented, a structure for the output database will 
  be defined and appropriate bookkeeping  will be implemented sufficient to 
  communicate the heat and solute evolution within the repository. 

\subsubsection{Demonstration Testing}

  % Testing

  This work will be developed with a test driven development strategy. That is, 
  before any new functionality is implemented, a suite of tests is written which 
  as closely define its necessary behavior as possible. The software is then 
  written with the goal of passing the test suite. In this way, the software 
  developed in this work is expected to be comprehensively tested in parallel 
  with its development. 


    % Send through 1 radionuclide with no filters in the volumes

    % Have each control volume release 1/3 of what it accepts. see 1/81st.

    Test problems which will help comprehensively define and confirm each unit 
    of the  demonstration functionality will include very basic information 
    passing tests as well as more complex multiple subcomponent integration 
    tests. A null test, for example,  will release a single contaminant 
    radionuclide through each subcomponent sequentially. This test will pass if  
    the bookkeeper properly writes to the database its (aphysically unhindered) 
    path through each control volume.

\subsection{Base Case Development}

% Base Case Milestone

\subsubsection{Base Case Concept}

  % Concept

    % uniform unfractured permeable porous medium

      % reducing geochemistry

      The base case concept will model a generic, isotropic, permeable porous 
      geological medium with reducing geochemistry. This model will be 
      appropriate for clay and salt geologies. The fracturation in the geologies  
      of the granite and deep borehole concepts will not be appropriately 
      modeled in the base case unless an equivalent porous medium calculation is 
      conducted external to the code. The incorporation of fracture models will 
      follow in subsequent extensions to the base case.
  
    % WF : glass 

      % alteration

      % temperature limit

    % WF : uox 

      % cladding limit

      % corrosion

      A waste form component module capable of modeling two canonical waste
      form concepts will be developed. It will be modeled with a rate based 
      dissolution model and will be equipped with a heat limit which may 
      constrict the waste form loading. Such a waste form component module will 
      be appropriate for borosilicate glass, the dissolution of which is 
      dominated by a surface alteration rate. It will be appropriate also for a 
      ceramic oxide waste form, the dissolution of which is dominated by a 
      corrosive degradation rate.

    % WP : steel

      % fractional
      
      An explicit time dependent probability distribution waste package failure 
      model will be employed that can accommodate instantaneous, constant, and 
      more complex time dependent failure probability density functions, such as  
      Weibull distribution.
      From a user-defined time dependent probability distribution (of which the 
      Weibull distribution is one), a waste package failure vector will be 
      populated, which assigns a failure time to each waste package object. As 
      the simulation progresses, these waste packages will fail discretely, 
      trigerring the initial degradation attack for the waste forms within them.

    % Buffer : Purely Diffusive Transport 
      
      % permeable porous medium

      The buffer component will be modeled as an isotropic permeable porous 
      medium which is chemically reducing and in which transport is diffusion 
      dominated  and solubility limited.  For the base case, this model will 
      involve only diffusive transport. Extensions to this model will include a 
      model for sorption as well as fracturation and advective transport.  

    % Heat limits at buffer and rock boundary 

      Heat limits in the base case will be calculated at the boundary between 
      the waste package and buffer as well as the boundary between the buffer or  
      backfill and the rock matrix.

\subsubsection{Base Case Abstraction }

  % Regression Analysis

  Abstraction will be conducted for each of the models listed above to identify  
  potential simplifying assumptions. For example, diffusive contaminant 
  transport in  permeable porous media will likely utilize the sensitivity 
  analysis results in section \ref{sec:current} to assume that the mean annual 
  dose is not sensitive to the horizontal spacing between drifts. Appropriate 
  conceptual models will employ these simplifications to streamline 
  calculations. 

\subsubsection{Base Case Code Development}

  % Code Development
  
  Building upon the empty demonstration code stucture, the development of the 
  base case model will primarily consist of populating the subcomponent control 
  volume models with appropriate physics, mass balances, rate equations, and 
  data. This step will implement the calculations in each subcomponent which 
  will provide sufficient information to determine the heat based capacity of 
  the repository for each request of material. 
  
\subsubsection{Base Case Testing, Verification, and Validation}

  % Testing
   
  % verification / validation 

  For verification and validation uring development, some comparisons to current 
  detailed models such as the \gls{UFD} \gls{GPAM} and GDSEs will be 
  incorporated into the testing framework. Additional verification and 
  validation can be expected to be conducted with respect to known benchmarks 
  such as \gls{ANDRA} and RED-IMPACT results once the model is fully functional.


\subsection{Extensions}


% Geology Extension Milestone

When the base case is established, a series of extensions to these models will 
be pursued. 

\subsubsection{Extension Concepts}

  % Concepts

    % Sorption
    
    First, additional phenomena in radionuclide transport will be added to the 
    modeling capability. Sorption, for example, will be added by incorporating 
    sorption effects to the basic diffusive solute transport model, approaching  
    a full solute transport solution such as equation \eqref{ogatabanks}. 
    Additionally, solubility limited transport will be added with a simple 
    restriction on the mixing calculations during mass balancing in the control 
    volumes.

    % Fracturation (granite)
    
    Next, additional geologies (i.e. granite and deep borehole) will be added 
    to the modeling capability by adding a dual continuum fracture model. 

    % Coalescent behavior (salt and clay)

    Another anticipated extension will adapt the radionuclide transport model 
    to incorporate the effects of heat and time driven coalescent behavior in 
    clay and salt behavior. This extension will focus on the porosity decrease 
    in those media as time and heat drive collapse around the waste packages.  
    Time dependent coalescence will be addressed first, followed by temperature  
    dependent coalescence.

    % Fast Pathways (borehole and salt)

    Another potential extension will address the issue of modeling a disruption 
    scenario with a fast advective pathway that intersects the repository.

\subsubsection{Extension Abstraction}

  % Regression Analysis


  \subsubsection{Extension Code Development}

  % Code Development

  \subsubsection{Extension Testing, Verification, and Validataion}

  % Testing



\subsection{Sensitivity Analysis}

A sensitivity analysis underway to arrive at simplified dominant physics 
relationships between the various input parameters of the simulation. 

\subsubsection{Nuclide Transport}

These are the independent parameters I'm interested in.

A TABLE, perhaps. Including parametric domain.

This is how I'll vary them.

Show that these provide a largely complete set of input variables to define 
nuclide transport. Discuss variables that are being neglected for simplicity, 
perhaps to be added later. 

\subsubsection{Heat Evolution}

These are the independent parameters I'm interested in.

A TABLE, perhaps. Including parametric domaie.

This is how I'll vary them.

Show that these provide a largely complete set of input variables to define heat 
evolution. Discuss variables that are being neglected for simplicity, perhaps to 
be added later. 

\subsection{Mathematical Model Abstraction}

Using the results of these sensitivity analyses, regression analysis will be 
performed to develop simplified parametric dependencies for all independent 
variables for each model. 

\subsection{Computational Model Development}

The development of the computational model will begin with simplistic models of  
the fundamental components of the detailing the phyical phenomena at work vs.  
the conceptual and mathematical models available, listed in order of complexity.


\subsubsection{Base Case}

Clay, bentonite backfill, no evacuation disturbed zone, some handfull of waste 
packages and waste forms.

%        File: cat_table.tex
%     Created: Tue Jul 19 11:00 AM 2011 C
% Last Change: Tue Jul 19 11:00 AM 2011 C
%
\begin{table}
  \centering
  \footnotesize{
  \begin{tabular}{|l|c|c|c|c|c|c|c|}
    \multicolumn{8}{c}{\textbf{Categorization of Phenomena}}\\
    \hline
     Phenomenon&Simplest&&&&&&Hardest\\
    \hline
     WF dissolution&instant&fractional&f(t)&f(H20)&f(T)&f(T,H20)&f(T,H20,etc.)\\
     WP dissolution&instant&fractional&f(t)&f(H20)&f(T)&f(T,H20)&\\
     WF release&instant&fractional&diffusive&advective&diff+adv&congruent&solubility limited\\
     WP release&instant&fractional&diffusive&advective&diff+adv&congruent&solubility limited\\
     Buffer failure&instant&fractional&f(t)&f(H20)&f(T)&f(T,H20)&f(T,H20,etc.)\\
     Buffer release &instant&fractional&diffusive&advective&diff+adv&congruent&solubility limited\\
     FF transport &diffusive&+fractures&+advective&congruent&+sorption&+colloids&solubility limited\\
     WF Heat&indexed&decay&&&&&\\
     WP Heat&conductive&+conv&+rad&+mass&2d&finite diff&finite element\\
     Buffer Heat&conductive&+conv&+rad&+mass&2d&finite diff&finite element\\
     FF Heat&conductive&+conv&+rad&+mass&2d&finite diff&finite element\\
    \hline
  \end{tabular}
  \caption[Categorization of Phenomena]{This table is a preliminary sketch of 
  the various categories of phenomena which will occur in the components of the  
  repository model.}
  \label{tab:cat}
  }
\end{table}





\paragraph{TAD Canisters} The canisters proposed for transportation, aging and 
disposal are called TAD canisters.  They are two concentric cylinders of steel 
and alloy22 inside and out respectively. 


\paragraph{Borosilicate Glass} Current borosilicate glass: Includes processing 
chemicals from original separations, with U/Pu removed, but minor actinides and 
Cs/Sr remaining Potential borosilicate glass: No minor actinides and/or no 
Cs/Sr; Mo may be removed to increase glass loading of radionuclides; it has 
alower volumetric heat rate


\paragraph{Glass Ceramic} Glass Ceramic:  This is glass-bonded sodalite from 
Echem processing of EBR-II, and from potential future Echem processing of oxide 
fuels o Metal Alloy: This includes subcategories


\paragraph{Metal Alloy} Metal alloy from Echem: Includes cladding as well as 
noble metals that did not dissolve in the Echem dissolution Metal alloy from 
aqueous reprocessing:  Includes undissolved solids and transition metal fission 
products


\paragraph{Advanced Ceramic} Advanced Ceramic: An advanced waste form that 
includes iodine volatilized during chopping, which is then gettered during 
head-end processing of used fuels


\paragraph{Separated Streams} Other:  Examples include radionuclides removed 
from other waste forms (e.g., Cs/Sr, I, C), as well as new waste forms such as a 
salt waste form

\paragraph{Classes A, B, and C waste} Lower Than High Level Waste (LTHLW): 
Includes Classes A, B, and C

\paragraph{GTCC LTHLW}  Greater Than Class C (GTCC)


\subsubsection{Demonstration}

Show that the complete model behaves in agreement with the more detailed model 
on which it was based. Else, iterate through sensitivity analyses, model 
abstraction, and computational development until the model is validated. 



\subsubsection{Extension}

This disposal system will at this point be extended to include a fleet of 
predefined model implementations to represent some canonical waste forms, 
packages, buffers, and clay types.  

\subsubsection{Fuel Cycle Analysis}

Show some various fuel cycles have different repository needs and metrics.  
Specifically, compare a closed fuel cycle, an open one, and at least one 
modified fuel cycle. 




% UFD developing metrics for the fct program option screening these and perhaps 
% other metrics will be included. 

