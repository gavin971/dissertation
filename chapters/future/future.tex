\chapter{Summary and Future Work}\label{ch:future}

\section{Summary}

\subsection{Motivation}

Most fuel cycle simulators are lacking in repository analysis. 

\subsection{Current Work}

Abstraction of current detailed repository models in combination with the 
current \Cyclus framework has generated a fundamental modeling concept and 
demonstrated capability for modeling the nuclide transport and heat evolution of 
a generic geological repository.

\section{Future Work}

\subsection{Sensitivity Analysis}

The sensitivity analysis underway to 


\subsubsection{Nuclide Transport}

These are the independent parameters I'm interested in.

A TABLE, perhaps. Including parametric domain.

This is how I'll vary them.

Show that these provide a largely complete set of input variables to define 
nuclide transport. Discuss variables that are being neglected for simplicity, 
perhaps to be added later. 

\subsubsection{Heat Evolution}

These are the independent parameters I'm interested in.

A TABLE, perhaps. Including parametric domaie.

This is how I'll vary them.

Show that these provide a largely complete set of input variables to define 
heat evolution. Discuss variables that are being neglected for simplicity, 
perhaps to be added later. 

\subsection{Mathematical Model Abstraction}

Using the results of these sensitivity analyses, regression analysis will be 
performed to develop simplified parametric dependencies for all independent 
variables for each model. 

\subsection{Computational Model Development}

\subsubsection{Base Case}

Clay, bentonite backfill, no evacuation disturbed zone, some handfull of waste 
packages and waste forms.

\subsubsection{Demonstration}

Show that the complete model behaves in agreement with the more detailed model 
on which it was based. Else, iterate through sensitivity analyses, model 
abstraction, and computational development until the model is validated. 


\subsubsection{Extension}

This disposal system will at this point be extended to include a fleet of 
predefined model implementations to represent some canonical waste forms, 
packages, buffers, and clay types. 

\subsubsection{Fuel Cycle Analysis}

Show some various fuel cycles have different repository needs and metrics. 
Specifically, compare a closed fuel cycle, an open one, and at least one 
modified fuel cycle. 



