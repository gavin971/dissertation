\section{Repository Modeling Paradigm}

The repository model architecture is intended to modularly permit 
exchange of disposal system subcomponents, accept arbitrary spent fuel 
streams, and enable extending modules representing new or different 
component models.

\subsection{Simulation Interface}

The interface of the repository model with the \Cyclus fuel cycle 
simulation interface is intended to be minimally restrictive, 
requiring only that the simulation supply waste stream information and 
provide a bookkeeping framework with which to record repository 
performance metrics. 

\subsubsection{Waste Stream Input}

The repository model must accept arbitrary spent fuel streams.  
Material objects resulting from the simulated fuel cycle arrive at the  
repository and are emplaced if all repository capacity limits allow 
it.

As the set of disposable material in the simulation is of heterogeneous 
composition and therefore heterogeneous heat production capability, the 
repository model will sometime need to recalculate its own capacity in terms of  
new materials.

\subsubsection{Repository Performance Metrics Calculated}

Repository performance metrics that may be calculated from the source 
term and heat data calculated by the model will cover a broad space of 
metrics of interest to sustainability goals. 

Some metrics support analyses which seek to maximize safe repository 
capacity under heat and source term limitations. Those include spatial 
dimensions, spatial dimensions per kWh or equivalent, repository 
footprint, and number of waste packages generated.

Still other metrics which may be calculated from the calculations of 
the repository model include environmental metrics such as peak 
radiotoxic dose to the public, radiotoxic fluxes released to the 
biosphere integrated over time, and the minimum managed lifetime.  
These metrics are recorded in a database flexibly defined by the repository 
model.

\subsubsection{Facility Functionality}
The repository will behave as a facility within the \Cyclus simulation 
paradigm. The fundamental facility behavior within \Cyclus involves 
participating in commodity markets. The repository will participate as 
a requester of waste commodities. During reactor operation, the 
repository will make requests to markets dealing in spent fuel streams 
according to its available capacity..

\subsection{Nested Component Concept}
The fundamental unit of information in the repository model is the 
nuclide release at each stage of containment, and the repository model 
in this work mimics reality treating them as nested elements in a 
release chain.

\subsubsection{Structure and Linkages}
Each component passes some information radially outward to the nested 
component immediately containing it and some information radially 
inward to the nested component it contains.

Most component models require external information concerning the 
water volume that has breached containment, so information concerning 
incoming water volumes is passed radially inward. 

Each component model similarly requires information about the nuclides 
released from the component it immediately contains.  Thus, nuclide 
release information is passed radially outward from the waste stream 
sequentially through each containment layer to the geosphere.

\subsubsection{Concept Generality}
The capability to allow each model to define the models within it gives this 
repository model concept the ability to model many types of repository concept 
while maintaining a simple interface with the simulation. 

\subsection{Components of the Nested System}
\subsubsection{Waste Stream}
The waste stream data object contains spent fuel isotopics over the 
course of the simulation. As nuclides are gained, lost, and transmuted within 
the spent fuel object, a history of its isotopic composition is recorded.

For waste streams that vary from each other in composition, the thermal capacity 
of the repository must be recalculated. One way to model this will be to 
recalculate the appropriate lengthwise spacing of waste packages when the heat 
generation rate of a new package is significantly different than other waste 
packages in the repository. 

\subsubsection{Waste Form}
The waste form model will calculate nuclide release due to dissolution 
of the waste form. Various heuristics by which nuclide release is modeled in 
accordance with waste form dissolution as well as the method by which 
the dissolution is modeled.

Dissolution can be instantaneous, rate based, water dependent, heat 
dependent, or coupled.

Dissolution related release can be modeled as congruent, solubility 
limited, or both. Some nuclides are immediately accessible, and some 
tend to remain in the fuel matrix. 

\subsubsection{Waste Package}
The waste package model calculates nuclide release due to waste 
pacakge failure. Waste package failure is typically modeled as 
instantaneous and complete or partial and constant. That is, a delay 
before full release, or a constantly present hole in the package.

Waste package time to failure is dependent on water contact and heat, 
but can be modeled as an average, probabilistic, or a rate.

In the case of highly deforming geologic media, such as salt, 
mechanical failure can be the primary mechanism for release from the 
waste package.

\subsubsection{Buffer}
Diffusion is the primary mechanism for nuclide transport through the 
buffer component of the repository system.  

Salt and borehole repositories may not have a buffer material.

\subsubsection{Backfill}
Similarly, diffusion is the primary mechanism for nuclide transport 
through the buffer component of the repository system.

Borehole repository concepts may not have a backfill material. 

\subsubsection{Tunnel Lining}
Much like the waste package model, nuclide release is typically 
modeled as either instantaneous and total or constant and partial. 

\subsubsection{Geological Environment}
Various hydrogeological models exist to represent travel through 
porous media, fractured porous media, crystalline media, and clays.

Sudicky and Frind give a model for fractured porous media which 
analytically captures nuclide mobility due to advection through 
fractures as well as competing retardation factors including sorption on the 
fracture walls, diffusion into stagnant fluid in the rock matrix, and 
sorption into rock matrix walls. 

