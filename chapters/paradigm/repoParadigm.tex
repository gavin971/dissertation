\section{Repository Modeling Paradigm}

The repository model architecture is intended to modularly permit 
exchange of disposal system subcomponents, accept arbitrary spent fuel 
streams, and enable extending modules representing new or different 
component models.

\subsection{Simulation Interface}

The interface of the repository model with the \Cyclus fuel cycle 
simulation interface is intended to be minimally restrictive, 
requiring only that the simulation supply waste stream information and 
provide a bookkeeping framework with which to record repository 
performance metrics. The repository model, in order to participate in the 
simulation as a facility model, must make requests for spent material up 
to its capacity. Determination of the repository capacity for various 
types of spent fuel commodities will comprise the interfacing functionality of 
the repository model. With the intention of developing the repository model in 
such a way as to be capable of interfacing with other simulation tools, however, 
calculation of metrics including expected dose rates and 
component failures will be the model's primary functionality. 

\subsubsection{Waste Stream Input}

The repository model must accept arbitrary spent fuel and high level waste
streams.  Material objects resulting from the simulated fuel cycle arrive at 
the  repository and are emplaced if all repository capacity limits allow 
it.

Since disposable material in most simulations of interest will be of variable 
composition and therefore heterogeneous heat production capability, the 
repository model will repeatedly need to recalculate its own capacity as 
new materials are offered.

\subsubsection{Repository Performance Metrics Calculated}

Repository performance metrics that may be calculated from the source 
term and heat data calculated by the model will cover many metrics of
interest to sustainability goals. Some metrics support analyses that
seek to maximize safe repository capacity under heat and source term limitations. 
Those include spatial dimensions, spatial dimensions per kWh or equivalent,
repository footprint, and number of waste packages generated.

Still other metrics that may be calculated include those being considered by 
the \gls{UFD} campaign in a Fuel Cycle Data Package task underway 
\cite{nutt_personal_2011}. Additional metrics that will be considered in this
context will likely include environmental metrics such as peak dose 
to the public, radiotoxic fluxes released to the biosphere integrated over time, 
and the minimum managed lifetime.  These metrics are recorded in a database 
flexibly defined by the repository model. 

\subsubsection{Facility Functionality}

The repository will behave as a facility within the \Cyclus simulation 
paradigm. The fundamental facility behavior within \Cyclus involves 
participating in commodity markets. The repository will participate as 
a requester of waste commodities. During reactor operation, the 
repository will make requests to markets dealing in spent fuel streams 
according to its available capacity. Possible optional intermediate storage facility 
model is available for cooling periods.

\subsection{Nested Component Concept}

The fundamental unit of information in the repository model is the 
nuclide release at each stage of containment, and the repository model 
in this work mimics reality treating them as nested elements in a 
release chain.

\subsubsection{Control Volumes}

Each component of the repository system (i.e. waste form, waste package, buffer, 
and geologic medium) is modeled as a discrete control volume. Each control 
volume performs its own mass balance at each time step and assesses its own 
internal  heat transfer and degradation phenomena separately from the other 
nested components.

Each control volume will initially be modeled as a mixed cell. That is, for 
permeable porous media, all contaminants released into the pore and fracture 
water are assumed to be uniformly distributed.  

\subsubsection{Information Passing Between Volumes}

Each component passes some information radially outward to the nested 
component immediately containing it and some information radially 
inward to the nested component it contains.

Most component models require external information concerning the 
water volume that has breached containment, so information concerning 
incoming water volumes is passed radially inward. 

Each component model similarly requires information about the radionuclides 
released from the component it immediately contains.  Thus, nuclide 
release information is passed radially outward from the waste stream 
sequentially through each containment layer to the geosphere.

\subsubsection{Concept Generality}

The capability to allow each model to define the models within it gives this 
repository model concept the ability to model many types of repository concept 
while maintaining a simple interface with the simulation. 

\subsection{Components of the Nested System}

The repository model is a collection of subcomponents which behave collectively 
to calculate repository metrics of interest. These subcomponents will be models 
of their own, and within the object oriented paradigm of the software will be 
a collection of module classes. Each component (i.e. waste form, waste 
package, buffer, lithology, etc.) will name a component superclass. Each superclass 
will be inherited by subclass models capable of representing that component in 
some level of detail specified by the model.

\subsubsection{Waste Stream}

The waste stream data object contains spent fuel isotopics over the 
course of the simulation. As radionuclides are gained, lost, and transmuted within 
the spent fuel object, a history of its isotopic composition is recorded.

For waste streams that vary from each other in composition, the thermal capacity 
of the repository must be recalculated. One way to model this will be to 
recalculate the appropriate lengthwise spacing of waste packages when the heat 
generation rate of a new package is significantly different than other waste 
packages in the repository. 

\subsubsection{Waste Form}
The waste form model will calculate nuclide release due to dissolution 
of the waste form. Various heuristics by which nuclide release is modeled in 
accordance with waste form dissolution as well as the method by which 
the dissolution is modeled.

Dissolution can be instantaneous, rate based, water dependent, heat 
dependent, or coupled.

Dissolution related release can be modeled as congruent, solubility 
limited, or both. Some radionuclides are immediately accessible, and some 
tend to remain in the fuel matrix. 

\subsubsection{Waste Package}
The waste package model calculates nuclide release due to waste 
package failure. Waste package failure is typically modeled as 
instantaneous and complete or partial and constant. That is, a delay 
before full release, or a constantly present hole in the package.

Waste package time to failure is dependent on water contact and heat, 
but can be modeled as an average, probabilistic, or a rate.

In the case of highly deforming geologic media, such as salt, 
mechanical failure can be the primary mechanism for release from the 
waste package.

\subsubsection{Buffer}
Diffusion is the primary mechanism for nuclide transport through the 
buffer component of the repository system.  

Salt, clay, and borehole repository concepts may not have a buffer 
material.

\subsubsection{Backfill}
Similarly, diffusion is the primary mechanism for nuclide transport 
through the buffer component of the repository system.

Clay concepts and borehole concepts may not have a backfill material. 


\subsubsection{Geological Environment}

The literature review introduced various hydrological models that represent
fluid and contaminant travel through permeable porous media and fractured porous 
media. These assume saturated flow and incorporate diffusive flow, advective 
flow, hydrodynamic dispersion, and equilibrium sorption. The geological 
environment control volume component will implement these models appropriately 
for each geology to provide a mass balance and to communicate concentrations to  
adjacent components.  Dirichlet boundary conditions at the surfaces of the 
control volume will allow the simulation to step through transport in the rock. 
Additional boundary condition types maybe implemented as extensions to the base 
case model.


