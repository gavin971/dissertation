\chapter{Categorization of Models}\label{ch:categorization}

The goal of this chapter is to categorize the available analytical and 
computational models into their salient characteristics according to the 
phenomena we are interested in modeling. For example, a table will be listed 
here detailing the phyical phenomena at work vs. the conceptual and 
mathematical models available, listed in order of complexity.

%        File: cat_table.tex
%     Created: Tue Jul 19 11:00 AM 2011 C
% Last Change: Tue Jul 19 11:00 AM 2011 C
%
\begin{table}
  \centering
  \footnotesize{
  \begin{tabular}{|l|c|c|c|c|c|c|c|}
    \multicolumn{8}{c}{\textbf{Categorization of Phenomena}}\\
    \hline
     Phenomenon&Simplest&&&&&&Hardest\\
    \hline
     WF dissolution&instant&fractional&f(t)&f(H20)&f(T)&f(T,H20)&f(T,H20,etc.)\\
     WP dissolution&instant&fractional&f(t)&f(H20)&f(T)&f(T,H20)&\\
     WF release&instant&fractional&diffusive&advective&diff+adv&congruent&solubility limited\\
     WP release&instant&fractional&diffusive&advective&diff+adv&congruent&solubility limited\\
     Buffer failure&instant&fractional&f(t)&f(H20)&f(T)&f(T,H20)&f(T,H20,etc.)\\
     Buffer release &instant&fractional&diffusive&advective&diff+adv&congruent&solubility limited\\
     FF transport &diffusive&+fractures&+advective&congruent&+sorption&+colloids&solubility limited\\
     WF Heat&indexed&decay&&&&&\\
     WP Heat&conductive&+conv&+rad&+mass&2d&finite diff&finite element\\
     Buffer Heat&conductive&+conv&+rad&+mass&2d&finite diff&finite element\\
     FF Heat&conductive&+conv&+rad&+mass&2d&finite diff&finite element\\
    \hline
  \end{tabular}
  \caption[Categorization of Phenomena]{This table is a preliminary sketch of 
  the various categories of phenomena which will occur in the components of the  
  repository model.}
  \label{tab:cat}
  }
\end{table}





\section{Fractures}

\subsection{Explicit Discrete Fracture Formulation}

This modeling formulation is complex and requires detailed knowledge of the 
candidate lithology. While an extensibility to include such a model will be 
retained in the repository system, it is well out of scope of the level of 
detail being sought by this nature of simulation. 

\subsection{Dual-Continuum Formulation}

The dual continuum formulation for either porosity or permeability allows a 
network of cracks to be  

It is one of the most widely employed models of fracture flow. 
\cite{diodato_compendium_1994}

\subsection{Discrete Fracture Network}
\subsection{Single Equivalent Continuum}
\subsection{Unsaturated or Multiphase Flow}
While some candidate repository geologies such as salt and granite are presumed 
to be unsaturated, the base case geology being modeled in this work is assumed 
to be saturated.


\section{Mixing}

\subsection{Precipitates}

\subsection{Dissolved Concentrations}



\section{Matter Dissolution}

\subsection{ Atomistic }

\subsection{ Molecular Dynamics } 

\subsection{ Staged }

Various stages of matter dissolution may include the initial attack, alteration, 
maturation, evolution, etc.

\subsection{ SWF/UFD/NEAMS data}

These will presumably give simplified constant rates. 


\section{Diffusion Coefficients}

\subsection{ Constant }

Tabulated.

\subsection{ Temperature Dependent : Arrenhius Relationship }

Gives D as a function of temperature and apparent activation energy of the 
medium.

\subsection{ Temperature Dependent : Stokes Einstein Relationship }

Gives D as a function of the temperature depedence of water.

\subsection{ Concentration Dependent : ? }

Maybe there is a generally known functional form.


\section{Heat}

I'll be using the lumped parameter model for everything. Maybe linking to SINDA 
or the LLNL program. 
 

. . . 

