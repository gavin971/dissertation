\chapter{Categorization of Models}\label{ch:categorization}

The goal of this chapter is to categorize the available analytical and 
computational models into their salient characteristics according to the 
phenomena we are interested in modeling. For example, a table will be listed 
here detailing the phyical phenomena at work vs. the conceptual and 
mathematical models available, listed in order of complexity.


\section{Media}

\subsection{Fractures}

\subsubsection{Explicit Discrete Fracture Formulation}

This modeling formulation is complex and requires detailed knowledge of the 
candidate lithology. While an extensibility to include such a model will be 
retained in the repository system, it is well out of scope of the level of 
detail being sought by this nature of simulation. 

\subsubsection{Dual-Continuum Formulation}

The dual continuum formulation for either porosity or permeability allows a 
network of cracks to be 

It is one of the most widely employed models of fracture flow. 
\cite{diodato_compendium_1994}

\subsubsection{Discrete Fracture Network}
\subsubsection{Single Equivalent Continuum}
\subsubsection{Unsaturated or Multiphase Flow}
While some candidate repository geologies such as salt and granite are presumed 
to be unsaturated, the base case geology being modeled in this work is assumed 
to be saturated.

\subsection{Mixing}

\subsubsection{Precipitates}

\subsubsection{Dissolved Concentrations}

\subsubsection{}<++>



