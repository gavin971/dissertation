
\begin{frame}[ctb!]
\frametitle{Specific Temperature Change Method}
\footnotesize{
Introduced by Radel, Wilson et al., the Specific Temperature Change (STC) method uses 
a linear approximation to arrive at the thermal loading density limit 
\cite{radel_repository_2007, radel_effect_2007}.  

Since the thermal response in a system with a long term transient response is strong function of the 
transient decay power, it is also a strong function of the isotopic 
composition of the waste. Thus, the time dependent temperature change, $\Delta 
T$, at the limiting radius, $r_{lim}$, can be approximated as proportional to the 
mass loading density. First, $\Delta T$ is determined for a limiting loading density 
of the particular material composition then it is normalized to a single 
kilogram of that material, $\Delta t$, the so called STC. 

\begin{align}
 \Delta T(r_{lim}) &= m \cdot \Delta t(r_{lim})
 \label{STC}
 \intertext{where}
 \Delta T &= \mbox{ Temperature change due to m }[^{\circ}K]\nonumber\\
 m &= \mbox{ Mass of heat generating material }[kg]\nonumber \\
 \Delta t &= \mbox{ Temperature change due to 1 kg }[^{\circ}K]\nonumber\\
 r_{lim} &= \mbox{ Limiting radius } [m].\nonumber
\end{align}
}
\end{frame}

\begin{frame}[ctb!]
\frametitle{Specific Temperature Change Superposition}
\footnotesize{

For an arbitrary waste stream composition, scaled curves, $\Delta t_i$, calculated in this 
manner for individual isotopes can be superimposed for each isotope to arrive at an 
approximate total temperature change.

\begin{align}
 \Delta T (r_{lim}) &\sim \sum_{i} m_i \Delta t_i(r_{lim})
 \label{superposition}
\intertext{where}
 i &= \mbox{ An isotope in the material } [-]\nonumber\\
 m_i &= \mbox{ mass of isotope i  } [kg]\nonumber\\
 \Delta t_i &= \mbox{ Specifc temperature change due to \textsl{i} } [^{\circ}K].\nonumber
\end{align}


}
\end{frame}
