
The results here provide an overview of the relative importance of thermal
parameters that that affect the repository capacity of simplified generic
disposal concept in various geologic media where conduction is the dominant
heat transfer mode. The applicability of this sensitivity analysis is thus
restricted to enclosed, backfilled concepts.  

\subsection{Parametric Domain}

Sensitivity analyses were conducted which span the parametric range of values 
generated by the reference specific temperature change database and described 
in Table \ref{tab:thermal_cases}.  

These values were selected to provide detail in the near field and at values of
$\alpha_{th}$ and $K_{th}$ in the three host media under consideration in this
work.

\subsection{Approach}

% used existing gdsms 
This analysis utilized the \gls{LLNL} semi-analytic MathCAD model utilized to 
derive supporting data and to validate behaviro within the \Cyder STC model. It 
performs detailed calculations of the conductive thermal transport in a generic 
repository concept with a gridded layout.  

It relies on the thermal diffusivity, $\alpha_{th}$ and conductivity $K_{th}$ of 
the material as well as the waste package spacing, $S$, and thermally limiting 
radius, $r_{lim}$. Finally, it relies on the \gls{STC} data calculated with the 
semi-analytic model based on the decay heat profiles of the emplace wastes, $Q$. 
The essential decay heat profiles, $Q$, were retrieved from a \gls{UFD} database 
provided by Carter et al. \cite{carter_fuel_2011}.


